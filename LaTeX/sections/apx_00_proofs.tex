\chapter{Proofs and other stuff}



%%%% ---------- Variant 2 ---------------
% \begin{proposition}\label{theorem:negativity}
% The \emph{negativity} $\mathscr{N}(\rho)$ of a state $\rho$ is given as the absolute sum of all negative eigenvalues of $\rho$: 
% \begin{equation}
%   \mathscr{N}(\rho) \equiv \frac{\norm{\rho^{\Gamma_A}}_1 - 1}{2} = \abs{\sum_{\lambda_i < 0} \lambda_i}
% \end{equation}
% \end{proposition}

% \begin{proof}
%   The proof is given in two steps, which in parts is given by Vidal \cite{Vidal_2001}:
% \begin{enumerate}
%   \item It is known, that the density matrix is hermitian: $\rho^\dagger = \rho$. For a general hermitian matrix $A$, the trace norm $\norm{A}_1 \equiv \tr \sqrt{A^\dagger A}$ is equal to the sum of the absolute eigenvalues of A.
%   This can be immediately seen by the spectral theorem:
%   \begin{equation*}
%     \tr \sqrt{A^\dagger A} = \tr \sqrt{A^2} = \tr{U\sqrt{\diag(\lambda_1, \dots)^2}U^\dagger} = \sum_i \sqrt{\lambda_i^2} .
%   \end{equation*}
%   \item The trace norm of the density matrix is given as $\norm{\rho}_1=\sum \lambda_i = \tr \rho = 1$. The partial transpose $\rho^{\Gamma_A}$ obviously also satisfies $\tr \rho^{\Gamma_A} = 1$ but might have negative eigenvalues. Since $\rho^{\Gamma_A}$ is still hermitian, the trace norm is given by
%   \begin{equation*}
%     \norm{\rho^{\Gamma_A}}_1 = \sum_i\abs{\lambda_i} = \sum_{\lambda_i \ge 0} \lambda_i + \sum_{\lambda_i < 0} \abs{\lambda_i} = \sum_i \lambda_i + 2\sum_{\lambda_i < 0} \abs{\lambda_i} = 1 + 2\sum_{\lambda_i < 0} \abs{\lambda_i} ,
%   \end{equation*}
%   where in the last step $\sum \lambda_i = \tr \rho^{\Gamma_A} = 1$ was used. The negativity can be defined as $\mathscr{N}(\rho) = \abs{\sum_{\lambda_i < 0} \lambda_i}$ and the statement is shown.
% \end{enumerate}
% \end{proof}
% \begin{remark}
%   The \emph{logarithmic negativity} \cite{Plenio_2005} relates to the negativity as follows
%   \begin{equation}
%     E_N(\rho) = \log_2 \norm{\rho^{\Gamma_A}}_1 = \log_2 \left( 2\mathcal{N}(\rho) + 1 \right)
%   \end{equation}
%   and can therefore be easily calculated by using the above \cref{theorem:negativity}. In comparison to the negativity, logarithmic negativity has additive properties \cite{Plenio_2005a}:
%   \begin{equation*}
%     E_N\left(\rho \otimes \sigma \right) = E_N(\rho) + E_N(\sigma) 
%   \end{equation*}
% \end{remark}
