\chapter{Proofs and other stuff}

\section{Negativity}

\begin{lemma}\label{lemma:trace-norm-hermitian}
  The trace norm $\norm{A}_1 \equiv \tr \sqrt{A^\dagger A}$ of a hermitian matrix $A$ is equal to the sum of the absolute eigenvalues of $A$.
\end{lemma}
\begin{proof}
  This can be immediately seen by the spectral theorem:
  \begin{equation*}
    \tr \sqrt{A^\dagger A} = \tr \sqrt{A^2} = \tr{U\sqrt{\diag(\lambda_1, \dots)^2}U^\dagger} = \sum_i \sqrt{\lambda_i^2} = \sum_i \abs{\lambda_i}.
  \end{equation*}
\end{proof}

\begin{proposition}\label{proposition:negativity}
  The \emph{negativity} $\mathscr{N}(\rho)$ of a state $\rho$ is given as the absolute sum of all negative eigenvalues of $\rho$: 
\begin{equation}
    \mathscr{N}(\rho) \equiv \frac{\norm{\rho^{\Gamma_A}}_1 - 1}{2} = \abs{\sum_{\lambda_i < 0} \lambda_i}.
\end{equation}
\end{proposition}
\begin{proof}
  The proof is in parts given by Vidal \cite{Vidal_2001}. It is known that the density matrix is hermitian: $\rho = \rho^\dagger$. Using \cref{lemma:trace-norm-hermitian}, the trace norm of the density matrix is is given as $\norm{\rho}_1=\sum \lambda_i = \tr \rho = 1$. The partial transpose $\rho^{\Gamma_A}$ obviously also satisfies $\tr \rho^{\Gamma_A} = 1$ but might have negative eigenvalues. Since $\rho^{\Gamma_A}$ is still hermitian, the trace norm is given by
  \begin{equation*}
    \norm{\rho^{\Gamma_A}}_1 = \sum_i\abs{\lambda_i} = \sum_{\lambda_i \ge 0} \lambda_i + \sum_{\lambda_i < 0} \abs{\lambda_i} = \sum_i \lambda_i + 2\sum_{\lambda_i < 0} \abs{\lambda_i} = 1 + 2\sum_{\lambda_i < 0} \abs{\lambda_i} ,
  \end{equation*}
  where in the last step $\sum \lambda_i = \tr \rho^{\Gamma_A} = 1$ was used. The negativity can be defined as $\mathscr{N}(\rho) = \abs{\sum_{\lambda_i < 0} \lambda_i}$ and the statement is shown.
\end{proof}
\begin{remark}
  The \emph{logarithmic negativity} \cite{Plenio_2005} relates to the negativity as follows
  \begin{equation}
    E_N(\rho) = \log_2 \norm{\rho^{\Gamma_A}}_1 = \log_2 \left( 2\mathcal{N}(\rho) + 1 \right)
  \end{equation}
  and can therefore be easily calculated by using the above \cref{proposition:negativity}. In comparison to the negativity, logarithmic negativity has additive properties \cite{Plenio_2005a}:
  \begin{equation*}
    E_N\left(\rho \otimes \sigma \right) = E_N(\rho) + E_N(\sigma) 
  \end{equation*}
\end{remark}

\newpage

%%%% ---------- Variant 2 ---------------
% \begin{proposition}\label{theorem:negativity}
% The \emph{negativity} $\mathscr{N}(\rho)$ of a state $\rho$ is given as the absolute sum of all negative eigenvalues of $\rho$: 
% \begin{equation}
%   \mathscr{N}(\rho) \equiv \frac{\norm{\rho^{\Gamma_A}}_1 - 1}{2} = \abs{\sum_{\lambda_i < 0} \lambda_i}
% \end{equation}
% \end{proposition}

% \begin{proof}
%   The proof is given in two steps, which in parts is given by Vidal \cite{Vidal_2001}:
% \begin{enumerate}
%   \item It is known, that the density matrix is hermitian: $\rho^\dagger = \rho$. For a general hermitian matrix $A$, the trace norm $\norm{A}_1 \equiv \tr \sqrt{A^\dagger A}$ is equal to the sum of the absolute eigenvalues of A.
%   This can be immediately seen by the spectral theorem:
%   \begin{equation*}
%     \tr \sqrt{A^\dagger A} = \tr \sqrt{A^2} = \tr{U\sqrt{\diag(\lambda_1, \dots)^2}U^\dagger} = \sum_i \sqrt{\lambda_i^2} .
%   \end{equation*}
%   \item The trace norm of the density matrix is given as $\norm{\rho}_1=\sum \lambda_i = \tr \rho = 1$. The partial transpose $\rho^{\Gamma_A}$ obviously also satisfies $\tr \rho^{\Gamma_A} = 1$ but might have negative eigenvalues. Since $\rho^{\Gamma_A}$ is still hermitian, the trace norm is given by
%   \begin{equation*}
%     \norm{\rho^{\Gamma_A}}_1 = \sum_i\abs{\lambda_i} = \sum_{\lambda_i \ge 0} \lambda_i + \sum_{\lambda_i < 0} \abs{\lambda_i} = \sum_i \lambda_i + 2\sum_{\lambda_i < 0} \abs{\lambda_i} = 1 + 2\sum_{\lambda_i < 0} \abs{\lambda_i} ,
%   \end{equation*}
%   where in the last step $\sum \lambda_i = \tr \rho^{\Gamma_A} = 1$ was used. The negativity can be defined as $\mathscr{N}(\rho) = \abs{\sum_{\lambda_i < 0} \lambda_i}$ and the statement is shown.
% \end{enumerate}
% \end{proof}
% \begin{remark}
%   The \emph{logarithmic negativity} \cite{Plenio_2005} relates to the negativity as follows
%   \begin{equation}
%     E_N(\rho) = \log_2 \norm{\rho^{\Gamma_A}}_1 = \log_2 \left( 2\mathcal{N}(\rho) + 1 \right)
%   \end{equation}
%   and can therefore be easily calculated by using the above \cref{theorem:negativity}. In comparison to the negativity, logarithmic negativity has additive properties \cite{Plenio_2005a}:
%   \begin{equation*}
%     E_N\left(\rho \otimes \sigma \right) = E_N(\rho) + E_N(\sigma) 
%   \end{equation*}
% \end{remark}




\section{Fidelity}
The \emph{fidelity} of two quantum states $\rho$ and $\sigma$ is defined as \cite[p. 409-412]{Nielsen_2010} 
\begin{equation}
  F(\rho, \sigma) = \tr \sqrt{\sqrt{\rho} \sigma \sqrt{\rho}}
\end{equation}
and can be used as a distance measurement between quantum states. It is monotonic, concave and bounded between 0 and 1. If both states are equal $\rho = \sigma$, it is clear that $F(\rho, \sigma) = 1$, by using $\sqrt{\rho}\rho\sqrt{\rho} = \rho^2$. If both states commute, i.e. they are diagonalizable in the same orthogonal basis $\{ \ket{i} \}$, 
\begin{equation*}
  \rho = \sum_i r_i \ketbra{i}; \quad \sigma = \sum_i s_i \ketbra{i},
\end{equation*}
the fidelity is given as \cite{Nielsen_2010}
\begin{equation*}
  F(\rho, \sigma) = \tr \sqrt{\sum_i r_i s_i \ketbra{i}} = \sum_i \sqrt{r_i s_i}.
\end{equation*}
This can be seen immediately by the use of the spectral theorem $\tr \sqrt{\rho} = \tr{U \sqrt{\mathrm{diag}(r_i)} U^\dagger} = \tr \diag(\sqrt{r_i})$.
Another special case is given for the fidelity of a pure state $\rho=\ketbra{\psi}$ and an arbitrary state $\sigma$ \cite{Nielsen_2010}:
\begin{equation*}
  F(\ket{\psi}, \sigma) = \tr \sqrt{\bra{\psi}\sigma\ket{\psi} \ketbra{\psi}} = \sqrt{\bra{\psi}\sigma\ket{\psi}}.
\end{equation*}
If the state $\sigma = \ketbra{\phi}$ is also pure, the fidelity reduces to
\begin{equation}
  F(\ket{\psi}, \ket{\phi}) = \abs{\braket{\psi}{\phi}} \le 1,
\end{equation}
with equality being attained if the states are the same and only differ by a phase. 