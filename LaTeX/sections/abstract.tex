\begin{abstract}
  \section*{Abstract}
  The quantum nature of gravity remains an unresolved question in modern physics, despite significant advances in both quantum theory and general relativity.
  The detection of gravitational entanglement between two massive particles has recently been considered as a crucial step towards establishing the non-classical nature of gravity.
  Some theoretical proposals suggest measuring such entanglement between the delocalized center-of-mass degrees of freedom of Schrödinger-cat states, interacting purely gravitationally in a controlled laboratory setup.
  The overall objective is to place the masses as close as possible to each other in order to enhance the gravitational coupling.
  To mitigate unwanted entanglement caused by short-range Casimir interactions at close particle separation, the use of a conductive Faraday shield between the masses has been proposed.
  This work investigates the impact of Casimir forces arising between the particles and the newly introduced shield on entanglement generation.
  It is shown that stochastic variations in the initial placement of the particles over multiple experimental runs can destroy measurable entanglement.
  In addition, the effects of thermal vibrations of the Faraday shield are analyzed, providing important insights into experimental feasibility and requirements. 
  These results contribute to ongoing efforts to design realistic experiments that probe the quantum nature of gravity.
\end{abstract}

% \begin{abstract}
% \section*{Abstract}
% The quantum nature of gravity remains an unresolved question in modern physics, despite significant advances in both quantum theory and general relativity.
% The detection of gravitational entanglement between two massive particles has recently been considered as a crucial step towards establishing the non-classical nature of gravity.
% Some theoretical proposals suggest measuring such entanglement between the delocalized center-of-mass degrees of freedom of Schrödinger-cat states, interacting purely gravitationally in a controlled laboratory setup.
% The overall objective is to place the masses as close as possible to each other in order to enhance the gravitational interaction.
% To mitigate unwanted entanglement caused by short-range Casimir interactions at close particle separation, the use of a conductive Faraday shield between the masses has been proposed.
% In this talk, I investigate the impact of Casimir forces arising between the particles and the newly introduced shield on entanglement generation.
% I show that stochastic variations in the initial placement of the particles over multiple experimental runs can destroy measurable entanglement.
% In addition, I analyze the effects of thermal vibrations of the Faraday shield and the associated entanglement dynamics. 
% \end{abstract}