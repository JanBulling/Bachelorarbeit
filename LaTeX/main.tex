%% ==== General ==============================================================

\newcommand{\authorName}{Jan Bulling}
\newcommand{\authorEmail}{jan.bulling@uni-ulm.de}
\newcommand{\matrikelnummer}{1109395}
\newcommand{\titleOfWork}{The Effects of Casimir Interactions in\\ Experiments on Gravitationally-induced\\Entanglement}
\newcommand{\dateOfWork}{\today}
\newcommand{\subjectOfWork}{Bachelor Thesis}
\newcommand{\publisherOfWork}{Universität Ulm}
\newcommand{\institute}{Institute of Theoretical Physics}
\newcommand{\faculty}{Fakultät für\\Naturwissenschaften}
\newcommand{\supervisor}{Marit O. E. Steiner, Julen S. Pedernales, Martin B. Plenio}
% more: see sections/titlepage and section/titlepage_2


%% ===========================================================================
%% ===========================================================================
%% ===========================================================================
\documentclass[a4paper, 12pt,
bibliography=totoc,   %% references in contents
numbers=noenddot,     %% no point after last number in outline
% abstracton,         %% use abstract
% ngerman,            %% german (new-german)
% twocolumn,          %% two columns
% DIV9,               %% dividing paper into 9 parts
% BCOR7mm,            %% space on left for printing and binding
% twoside,            %% like a book but not using the book class
% chapterprefix,      %% the word "chapter" before each chapter
% headsepline,        %% line after heading (see footsepline)
captions=tableheading,  %% distance between table and text
]{scrreprt} %% srreprt for report (chapters) or scrartcl for article (sections,...), scrbook for book

\parindent 0pt	%% indent of new paragraphs
\parskip 6pt	%% distance between paragraphs

\pagestyle{headings} %% headings (empty, plain)


\interfootnotelinepenalty=10000 %% footnotes are not split between pages

\DeclareEmphSequence{\bfseries\itshape} %% \emph is bold and italic


%% --- fonts ---
\usepackage{lmodern}


%% --- language ---
% \usepackage[T1]{fontenc}
% \usepackage[utf8]{inputenc}
% \usepackage[ngerman]{babel}
\usepackage[english]{babel}


%% --- change font ---
% \usepackage[scaled]{helvet}
% \renewcommand{\familydefault}{\sfdefault} 
% \KOMAoptions{DIV=last}


%% --- graphics and images ---
\usepackage{graphicx}
\usepackage{color}

%% --- sub-figures ---
\usepackage[labelfont=bf]{caption}
\usepackage{subcaption}


%% --- math and physics ---
\usepackage{amsfonts, amssymb, amsmath, amsthm}
\usepackage{bbold} %% for \mathbb{1}
\usepackage{mathrsfs}   %% use scripting (\mathscr{A})
\numberwithin{equation}{chapter}    %% use chapter in equation numbering
\usepackage{physics}
\usepackage[version=4]{mhchem} %% chemical formulas (\ce{H_2O})
\usepackage{mathtools}  %% useful for \xrightarrow[Below]{Above}


%% footnotes keep counting and don't reset for each chapter
\counterwithout{footnote}{chapter}


%% --- Index ---- Usage: \index{word1}word1 - or -  \index{Abschlussarbeit!Bachelor!M. of Sc.}Abschlussarbeit
%% At correct position in the document: \printindex
% \usepackage{index}
% \makeindex


%% --- Acronym ---- Usage: \ac{USA}
% \usepackage[printonlyused,withpage]{acronym} 


%% --- Tikz and libraries ---
% \usepackage{tikz}

%% quantum circuits (https://mirrors.ibiblio.org/CTAN/graphics/pgf/contrib/quantikz/quantikz.pdf)
% \usetikzlibrary{quantikz2}


%% --- references --- Links and references in pdf document
\usepackage[
    % hidelinks,        %% hides colored links
    bookmarksopen=true, %% open bookmarks in e.g. Adobe reader
    bookmarksopenlevel=1,
    bookmarksnumbered=true,
    pdfusetitle,
    % pdfauthor={Jan Bulling},
    % pdfsubject={},
    % pdftitle={},
    % pdfkeywords={},
    % pdfproducer={Universität Ulm},
]{hyperref}

% \usepackage{varioref}   %% \vref for intelligent refs with page, \Vref for starting of sentence
\usepackage{cleveref}   %% \cref for intelligent refs


%% --- Tables ---
\usepackage{multirow}
\usepackage{array}
\usepackage{booktabs}
\newcolumntype{L}[1]{>{\raggedright\let\newline\\\arraybackslash\hspace{0pt}}m{#1}}
\newcolumntype{C}[1]{>{\centering\let\newline\\\arraybackslash\hspace{0pt}}m{#1}}
\newcolumntype{R}[1]{>{\raggedleft\let\newline\\\arraybackslash\hspace{0pt}}m{#1}}


%% --- Biblatex ---
%% (https://de.overleaf.com/learn/latex/Biblatex_bibliography_styles)
%% https://mirror.dogado.de/tex-archive/macros/latex/contrib/biblatex-contrib/biblatex-phys/biblatex-phys.pdf
\usepackage[]{csquotes}
\usepackage[
    backend=biber,
    style=phys,     %% phys, nature, numeric-comp, apa
    sorting=none,   %% none -> appearance, nty -> author, title, year, ...
    natbib=true,
    giveninits=true,
    abbreviate=false,
    block=space,
    %% biblatex-phys options
    url=false, isbn=false,
    doi=true, 
    eprint=true,
    biblabel=brackets
]{biblatex}
\addbibresource{./../Paper/bachelorarbeit.bib}


%% --- Custom commands ---
\newcommand{\q}[1]{``#1''}
\newcommand{\si}[1]{\, \mathrm{#1}}
\newcommand{\Si}[1]{\mathrm{#1}}

\renewcommand{\abs}[1]{\left\lvert#1\right\rvert}
\renewcommand{\norm}[1]{\left\lVert#1\right\rVert}
\newcommand{\mean}[1]{\left\langle#1\right\rangle}
% \newcommand{\mean}[1]{\overline{#1}}
\newcommand{\avg}[1]{\left\langle#1\right\rangle}
\renewcommand{\op}[1]{\hat{#1}}
\renewcommand{\deg}{\mathrm{^\circ}}

%% --- vector notations (leaf commented for default arrow) ----
\renewcommand{\Vec}[1]{\vectorbold{#1}}   %% bold font
% \renewcommand{\Vec}[1]{\bar{#1}}          %% overline
% \renewcommand{\Vec}[1]{\underline{#1}}    %% underline (Audenaert notation)
\renewcommand{\vec}[1]{\Vec{#1}}

%% --- Custom operators ---
\DeclareMathOperator{\diag}{diag}
\DeclareMathOperator*{\argmax}{arg max}
\DeclareMathOperator*{\argmin}{arg min}

%% --- Custom environments ---
%% usage: \begin{theorem}[Pythagorean theorem] ..... \end{theorem}
%% 
%% Definition: an explanation of the mathematical meaning of a word.
%% Theorem: A statement that has been proven to be true.
%% Proposition: A less important but nonetheless interesting true statement.
%% Lemma: A true statement used in proving other true statements (that is, a less important theorem that is helpful in the proof of other results).
%% Corollary: A true statement that is a simple deduction from a theorem or proposition.
\theoremstyle{plain}
\newtheorem{proposition}{Proposition}[chapter]
\newtheorem{theorem}{Theorem}[chapter]
\newtheorem{corollary}{Corollary}[chapter]
\newtheorem{lemma}[theorem]{Lemma}  %% shared counter with theorems

\theoremstyle{definition}
\newtheorem{definition}{Definition}[section]

\theoremstyle{remark}
\newtheorem*{remark}{Remark}    %% unnumbered

% \renewcommand{\qedsymbol}{$blacksquare$} %% redefine proof-end symbol



%% ===========================================================================
%% ===== Document ============================================================
%% ===========================================================================
\begin{document}

%% ---- title page ------------------------------------------------
%% choose title page (either sections/titlepage or sections/titlepage_2)
%\frontmatter
\thispagestyle{empty}

\begin{titlepage}
  \sffamily
  \raggedright

  \hfill
  \includegraphics[height=1.8cm]{logo_uulm_sw.png}\\[2em]

  {\footnotesize
    \hspace*{115mm}
    \parbox[t]{35mm}{
      \bfseries \faculty\\

      \mdseries \institute
    }\\[2cm]
  
    \parbox{140mm}{\bfseries \LARGE \titleOfWork}\\[2.5em]
    {\bfseries \LARGE \subjectOfWork}\\[4em]

    {\footnotesize \bfseries Submitted by:}\\
    {\footnotesize \authorName\\ \authorEmail \\ \matrikelnummer} \\[2em]

    {\footnotesize \bfseries Supervisors:}\\
    {\footnotesize \supervisor}\\[2em]
  }
  
\end{titlepage}
  


% %% ===== Title page ==========================================

% multiple authors: Author 1\thanks{...} \and Author 2\thanks{...} \and ...
\author{\authorName \thanks{\authorEmail}}
\title{\titleOfWork}
\date{\dateOfWork}
\subject{\subjectOfWork}
\publishers{\publisherOfWork}

\titlehead{\hfill\includegraphics[height=1.8cm]{unilogo_sw.jpg}}
%% back of title cover. Only if using a book class
% \uppertitleback{Back of title page}
% \lowertitleback{lower section after title page}

%% dedication on separate page
% \dedication{For my mom}

\maketitle[0]   %% in [] use the offset of the page number (can be negative)


%% ---- abstract ------------------------------------------------
\newpage
\begin{abstract}
  \section*{Abstract}
  The quantum nature of gravity remains an unresolved question in modern physics, despite significant advances in both quantum theory and general relativity.
  The detection of gravitational entanglement between two massive particles is often considered as a crucial step towards establishing the non-classical nature of gravity.
  Recent theoretical proposals suggest measuring such entanglement between delocalized Schrödinger-cat states interacting purely gravitationally in a controlled laboratory setup.
  To mitigate unwanted entanglement caused by short-range Casimir interactions at close particle separation, the use of a conductive Faraday shield between the masses has been proposed.
  This work investigates the impact of Casimir forces arising between the particles and the newly introduced shield.
  It is shown that stochastic variations in the initial placement of the particles over multiple experimental runs can destroy measurable entanglement.
  In addition, the effects of thermal vibrations of the Faraday shield on entanglement generation are analyzed, providing important insights into experimental feasibility and requirements. 
  These results contribute to ongoing efforts to design realistic experiments that probe the quantum nature of gravity.
\end{abstract}

% \begin{abstract}
%   \section*{Abstract (I-Form)}
%   The quantum nature of gravity remains an unresolved question in modern physics, despite significant advances in both quantum theory and general relativity.
%   The detection of gravitational entanglement between two massive particles is often considered as a crucial step towards establishing the non-classical nature of gravity.
%   Recent theoretical proposals suggest measuring such entanglement between delocalized Schrödinger-cat states interacting purely gravitationally in a controlled laboratory setup.
%   To mitigate unwanted entanglement caused by short-range Casimir interactions at close particle separation, the use of a conductive Faraday shield between the masses has been proposed.
%   In my thesis, I investigated the impact of Casimir forces arising between the particles and the newly introduced shield.
%   I showed that stochastic variations in the initial placement of the particles over multiple experimental runs can destroy measurable entanglement.
%   In addition, the effects of thermal vibrations of the Faraday shield on entanglement generation are analyzed.
% \end{abstract}


%% ---- table of contents, list of tables/figures ----------------
\newpage
\tableofcontents
% \listoffigures
% \listoftables

% ---- abbreviation --------------------------------------------
% \input{sections/acronym}


%% ---- content ------------------------------------------------
\newpage
\chapter{The consequences of a thermal shield}\label{cha:the-shield}
Up until now, the dynamics and properties of the shield where neglected. At a non-zero temperature however, thermal vibrations of the shield may influence the generation of entanglement substantially.
In this chapter, firstly the required size of the shield is estimated. Then, the thermal vibrations of a large and small shield as well as the effect on entanglement generation is considered.
In the experiment, the particles are cooled down into the ground state for effective quantum control and the generation of spatial superpositions.
For cooling, often liquid helium at $T\approx 4\si{K}$ is used but cryogenic freezers can cool small setups down to the order of $T\approx 20\si{mK}$.
For all relevant calculations, these temperatures are used as a reference point.


\section{Thickness and size of the shield}\label{sec:5:shield-size}
The required thickness $d$ and the radius $r_s$ of the spherical shield to effectively block electromagnetic interactions between the particles can be estimated by analyzing the properties of a physical conductive material with high electric conductivity $\sigma$.
For an almost perfect shielding, a superconducting shield may be considerable as well.
The transmission $T$ of electromagnetic waves through a physical shield is given by \cite{Vandenbosch_2022}
\begin{equation}\label{eq:5:shield-transmission}
  T = \abs{\frac{\vec{E}_\mathrm{after}}{\vec{E}_\mathrm{before}}} = \frac{2}{Z_0 \sigma d}
\end{equation}
where $Z_0 = 377\si{\Omega}$ is the impedance of free space (assuming the shield is placed in a vacuum or in air).
The electric conductivity $\sigma$ is highly dependent on the temperature \cite[p. 284-286]{Gross_2018}, decreasing approximately with $1/T^5$ at low temperatures \footnote{This behavior is valid for temperatures below the Debye temperature ($\Theta_D = 343\si{K}$ for copper). At the low temperatures used in the experiment, this model accurately describes the conductivity of metals \cite{Berman_1952}.}.
Copper offers a strong electric conductivity of $\sigma = 59.6\times 10^6 \si{S/m}$ at room temperature and measured data showing even $\sigma(T = 10\si{K}) \approx 1.5\times 10^{10}\si{S/m}$ at $10\si{K}$ \cite{Berman_1952}.

To estimate the shield's thickness, the primary criterion used, is that gravitational interactions should dominate the entanglement generation.
Other mutual interactions between the particles, such as Coulomb or Casimir forces, must be sufficiently suppressed by the shield.
The \emph{entanglement rate} $\Gamma$ quantifies the build-up of entanglement over time
\begin{equation}\label{eq:5:entanglement-rate}
  \Gamma = \dv{t} E_N(\rho)\Big|_{t=0} \, ,
\end{equation} 
where $E_N$ is an appropriate entanglement measure - in this case the logarithmic negativity \cite{Plenio_2005} introduced in \cref{sec:2:entanglement-measures}.
For gravitational interactions, the entanglement rate in the parallel orientation is given by using eq. \eqref{eq:2:entanglement-dynamics-parallel} as
\begin{equation}\label{eq:5:entanglement-rate-gravity}
  \Gamma_\mathrm{Gravity} = \frac{G M_A M_B \Delta x_A \Delta x_B}{16 \hbar L^3 \log 2} = \frac{G \pi^2 R^6 \rho_\mathrm{Silica}^2 (\Delta x)^2}{9 \hbar L^3 \log 2} .
\end{equation}
where in the last step $M_A = M_B = 4/3 \pi R^3 \rho_\mathrm{Silica}$ and $\Delta x_A = \Delta x_B \equiv \Delta x$ was used.
The entanglement rate for non-gravitational interactions, such as Coulomb or Casimir forces, must be significantly smaller than the gravitational entanglement rate, ideally by a factor $\chi \gg 1$.
This ensures that any measured entanglement is primarily due to gravitational interactions.
In the following sections, estimations about the thickness and size of the shield are made, to effectively screen Coulomb and Casimir forces.


\subsection{Shielding Coulomb-Interactions}
The primary role of the Faraday shield is to block electromagnetic interactions between particles.
Experimentally, it may be beneficial for the particles to carry a small amount of charge, enabling the use of electrostatic traps with high trapping strength and large controllability \cite{GonzalezBallestero_2021}. 
The Coulomb interaction potential between two charged particles is given by
\begin{equation}
  V = \frac{1}{4\pi\varepsilon_0} \frac{q_A q_B}{2L}
\end{equation}
where $\varepsilon_0 = 8.8542\times 10^{-12}\si{A^2 s^4 m^{-3} kg^{-1}}$ is the permittivity of free space and we assume that each particle caries one electron charge $\abs{q_{A(B)}} = e = 1.6022\times 10^{-19}\si{C}$.
This interaction mimics the form of the gravitational potential and can similarly induce entanglement with a entanglement rate
\begin{equation}\label{eq:5:entanglement-rate-coulomb}
  \Gamma_\mathrm{Coulomb} = \frac{T \abs{q_A q_B} (\Delta x)^2}{64 \pi \varepsilon_0 \hbar L^3 \log 2} .
\end{equation}
The shield suppresses the coupling by a factor of $T$.
Requiring $\Gamma_\mathrm{Gravity} > \chi \Gamma_\mathrm{Coulomb}$, the minimum thickness of the shield can be calculated by using eq. \eqref{eq:5:shield-transmission} as
\begin{align}\label{eq:5:coulomb-gravity-condition}
  T \frac{\abs{q_A q_B}}{64 \pi \varepsilon_0} \chi \, &< \, \frac{G \pi^2 R^6 \rho_\mathrm{Silica}^2}{9} \\
  \Longleftrightarrow \quad\quad\quad\quad d \, &> \, \frac{9}{32}\frac{1}{Z_0 \sigma} \frac{1}{\pi^3 \varepsilon_0 G \rho_\mathrm{Silica}^2} \frac{e^2}{R^6} \chi .
\end{align}
The thickness strongly depends on the particles size $R$, and large or heavy particles will favor gravitational entanglement generation.
Assuming the particles are silica nano-spheres with parameters given in \cref{tab:paramters}, a minimum shield-thickness of $d \approx 10\si{nm}\chi$ at $4\si{K}$ and of $d \approx 2.5\si{\mu m}\chi$ at room temperature is required.
At low temperatures, a realistic shield thickness could therefore be $d=100\si{nm}$, balancing engineering practicality and electromagnetic suppression.
Exact estimations however depend on the realization of the experiment as well as the precision in which the evolved state is measurable.

Electrostatic fields still can propagate around the finite-sized Faraday shield and potentially induce entanglement.
It is possible to estimate the required shield radius $r_s$ to block a specific amount $\eta$ of the electric field (see \cref{apx:blocking-of-the-shield}):
\begin{equation}\label{eq:5:shield-effectiveness}
  \frac{r_s}{L} = \sqrt{\frac{1-(1-\eta)^2}{(1-\eta)^2}}
\end{equation}
The results are visualized in \cref{fig:5:shield-radius}.
\begin{figure}[!ht]
  \centering
  \includegraphics[width=\textwidth]{./../figures/others/shield-radius.pdf}
  \caption{Shield radius as a function of the shielding effectiveness $\eta$ for an ideal shield. Additionally, a real shield with varying thicknesses $d$ is considered at $T=300\si{K}$.
  To achieve shielding of $99.5-99.9\%$ ($\eta = 0.995-0.999$), a radius of $r_s =200-1000L$ is needed.}
  \label{fig:5:shield-radius}
\end{figure}
The shield's transmission $T$ should therefore be modified to $\tilde{T} = T\eta + (1-\eta)$, where the shielding effectiveness $\eta$ depends on $r_s$ as given by eq. \eqref{eq:5:shield-effectiveness}.
Modifying eq. \eqref{eq:5:coulomb-gravity-condition}, a minimum effectiveness $\eta_\mathrm{min}$ for sufficient shielding of
\begin{equation}
  \eta_\mathrm{min} \approx 1 - \frac{64\pi^3 \varepsilon_0 G R^6 \rho^2_\mathrm{Silica}}{9 e^2}
\end{equation}
can be approximated.
Using again the parameters from \cref{tab:paramters}, a minimum shielding of $\eta_\mathrm{min} \gtrsim 0.99997$ and thus a radius of $r_s \gtrsim 28000 L \approx 60\si{cm}$ is required.
Depending on the experimental setup, such a shield might be too large for all practical purposes and it may be beneficial to choose heavier masses ($\tilde{M} \sim 4 M$) to reduce the shield size to the orders of $\sim 1\si{cm}$.
Due to practicality, a shield with $r_s = 1\si{cm}$ is used in the following calculations.
Uncharged particles would eliminate the Coulomb interactions and therefore reducing the shield's size and thickness to only shield mutual Casimir interactions.




\subsection{Shielding Casimir-Interactions}
Similarly to Coulomb interactions, it is possible to to estimate the required thickness for a shield to sufficiently block Casimir interactions.
The Casimir potential between the spheres with radius $R$ separated by $2L$ is given in first order by \cite{Emig_2007}
\begin{equation}
  V = -\frac{23 \hbar c}{4\pi \cdot 128 L^7} \left( \frac{\varepsilon_r - 1}{\varepsilon_r + 2} \right)^2 R^6 .
\end{equation}
The corresponding entanglement rate is calculated similar to before by expanding the potential in small $\Delta x$ and computing the logarithmic negativity:
\begin{equation}
  \Gamma_\mathrm{Casimir} = T^2 \frac{161}{4096} \frac{c R^6 (\Delta x)^2}{\pi L^9 \log 2}\left( \frac{\varepsilon_r - 1}{\varepsilon_r + 2}\right)^2 .
\end{equation}
The dependence on $T^2$ arises because Casimir forces are second order effects in the dipole-dipole interaction \cite{Bordag_2001}.
Requiring gravitational entanglement generation to dominate, $\Gamma_\mathrm{Gravity} > \chi \Gamma_\mathrm{Casimir}$, leads to
\begin{align}
  T^2 \frac{161 c R^6}{4096 \pi L^6} \left( \frac{\varepsilon_r - 1}{\varepsilon_r + 2}\right)^2 \chi \, &< \, \frac{G \pi^2 \rho_\mathrm{Silica}^2 R^6}{9\hbar} \\
  \Longleftrightarrow \quad\quad\quad\  d \, &> \, \sqrt{\frac{1449}{4096} \frac{c \hbar}{G \pi^3}} \frac{2}{Z_0 \sigma \rho L^3} \frac{\varepsilon_r - 1}{\varepsilon_r + 2} \sqrt{\chi} ,
\end{align}
where again $T = 2/Z_0 \sigma d$ from eq. \eqref{eq:5:shield-transmission} was used.
For large separations, the shield thickness can be arbitrarily low, as Casimir forces vanish and at separations of $L\gtrsim 100\si{\mu m}$, the shield might not be required at all (compare \cref{sec:2:experimental-problems}).
Assuming two identical silica nano-spheres with parameters given by \cref{tab:paramters}, the required minimum thickness is between $4\times 10^{-11}\si{m} \sqrt{\chi}$ at $4\si{K}$ and $10 \si{nm} \sqrt{\chi}$ at room temperature.
This is much thinner than what is required for shielding Coulomb interactions.
The factor $\varepsilon_r$ modifies the thickness only by up to a constant factor of $\leq 1$ and is therefore ignored for worst-case estimations.

However, very thin shields lose mechanical rigidity, leading to enhanced vibrational excitations and potential instabilities.
Vibrational frequency and thus the vibrational energy depends linearly on the shield's thickness, making thinner shields prone to large decoherence due to thermal vibrations.
A detailed analysis of these effects is provided in the subsequent section.



\subsection{Gravitational effects of the shield}\label{subsec:5:shield-gravitation}
The gravitational interaction between the masses and the shield is generally neglected because it has no significant impact on the entanglement generation between the particles.
The only potential effect is indirect entanglement mediated by the thermal oscillations of the shield, as both masses couple gravitationally to it. 
However, as shown in \cref{sec:5:thermal-entanglement}, this second-order effect is very weak and does not pose a problem, since it still represents gravitationally mediated entanglement - which is the focus of the experiment anyway.
The gravitational force between a sphere with mass $M$ and an infinitesimal mass segment $\dd m = r d \rho_\mathrm{Cu} \dd r \dd \varphi$ of the shield made of copper with density $\rho_\mathrm{Cu} = 8960\si{kg/m^3}$ at a distance $r$ from the shield's center is given by
\begin{equation}
  \dd \vec{F} = \frac{G M \dd m}{\ell} \boldsymbol{\hat{\ell}} 
  \quad \Rightarrow \quad
  \dd F_z = \frac{G M r \rho_\mathrm{Cu} d}{\ell^2} \dd r \dd \varphi \cos \theta,
\end{equation}
where $\ell^2 = r^2 + L^2$ denotes the distance between the sphere and the mass segment and $\theta = \arccos L/\ell$ is the angle between them.
The total attractive force between the mass and the shield with radius $r_s$ is therefore
\begin{equation}
  F_z = GM \rho_\mathrm{Cu} d L \int\limits_{0}^{r_s} \dd r \int\limits_{0}^{2\pi} \dd \varphi \, \frac{r}{(r^2 + L^2)^{3/2}} = 2\pi G M \rho_\mathrm{Cu} d \left(1 - \frac{L}{\sqrt{L^2 + r_s^2}}\right) .
\end{equation}
For large shields $r_s \rightarrow \infty$ this is independent of the particle-shield separation $L$.
For a shield with thickness $d = 100\si{nm}$ and a silica particle with parameters given in \cref{tab:paramters}, the attraction force is around $F_\mathrm{particle-shield} \approx 4.1\times 10^{-24} \si{N}$ which is comparable with the attraction gravitational attraction force between the two particles themselves at $F_\mathrm{particle-particle} \approx 5.0 \times 10^{-24}\si{N}$ but is much weaker than the Casimir attraction between the particle and the shield with $F_\mathrm{Casimir} \approx 1.4 \times 10^{-17} \si{N}$.
Therefore, the gravitational effect of the shield can be neglected in all practical calculations.

\section{Thermal shield vibrations}\label{sec:4:thermal-vibrations}

\begin{figure}[!htbp]
  \centering
  \includegraphics[width=\textwidth]{./../figures/vibrations/vibrational-modes.pdf}
  \caption{Normalized shape of the vibrational modes $(k,l)$ of a vibrating spherical plate fixed at the edge with $R/d = 1000$.}
  \label{fig:4:vibrational-modes}
\end{figure}



\begin{equation}
  \op{H} = \hbar \omega \left(\op{a}^\dagger\op{a} + \frac{1}{2}\right) + \tilde{g}_A(\op{z}) \ketbra{\psi_A} + \tilde{g}_B(\op{z}) \ketbra{\psi_B}
\end{equation}

Solution:
\begin{equation}
  \rho(t) = \frac{1}{2}\begin{pmatrix}
    1 & e^{-i\varphi(t)}e^{-\gamma(t)}\\
    e^{i\varphi(t)}e^{-\gamma(t)} & 1
  \end{pmatrix}
\end{equation}
with the phase
\begin{equation}
  \varphi(t) = \frac{1}{\hbar^2\omega^2} \left(\sin(\omega t) - \omega t\right) \left(g_A^2 - g_B^2\right)
\end{equation}
and the decoherence term
\begin{equation}
  \gamma(t) = \frac{4(g_A - g_B)^2}{\hbar^2 \omega^2} \sin(\frac{\omega t}{2}) \left[\bar{n} + \frac{1}{2}\right]
\end{equation}

\begin{figure}[!htbp]
  \centering
  \includegraphics[width=\textwidth]{./../figures/vibrations/decoherence-analytical.pdf}
  \caption{Maximum decoherence $\gamma$ at $4\si{K}$ for all modes if the cat-state is placed \textbf{left:} at the point with maximum gradient of the mode $(1,0)$ and \textbf{right:} in the center of the plate. It becomes evident, that only a few low modes play an actual role for the total dephasing.}
  \label{fig:4:}
\end{figure}

\section{Entanglement in front of a thermal shield}\label{sec:5:thermal-entanglement}
The entanglement generation between the two particles depends heavily on the variation of the separation between the shield and the particles, as has been seen in \cref{cha:entanglement-generation}.
The vibrating shield can be interpreted as varying the separation and angle of the cat-state in front of the plate - visualized in \cref{fig:5:vibrating-translation-to-variations}.
\begin{figure}[!htbp]
  \centering
  \def\svgwidth{\textwidth}
  \input{./../figures/plate-vibration.pdf_tex}
  \caption{For a large $r_s \gg R$ and locally flat shield, the thermal vibrations with amplitude $z$ can be interpreted as a static shield where the particle $A$ (shown in the figure) is placed at $L+\Delta L$ at angle $\theta$ and particle $B$ is places at $L-\Delta L$ with angle $-\theta$ where both variations depend on the amplitude. At low vibrational frequencies $1/\omega \approx t_\mathrm{max}$ the amplitude can be assumed to be static during a experimental run and for each measurement thermally distributed around $\mean{z}=0$ with $\Delta z$ given by eq. \eqref{eq:5:amplitude-variance}.}
  \label{fig:5:vibrating-translation-to-variations}
\end{figure}
This is only a good approximation for shields larger than the particles radius $r_s \gg R$ and low vibrating frequencies $1/\omega \approx t_\mathrm{max}$ and can therefore be used well to describe the highly disturbing first few modes on a large shield.
Furthermore, this interpretation is possible because as shown in \cref{sec:3:imperfect-plates}, the Casimir interaction between a sphere and a tilted plane does not differ from the interaction between a flat plane.
Contrary to the problem considered in \cref{cha:entanglement-generation}, here only the thermal amplitude $z_{kl}$ is a independent random variable distributed around $\mean{z_{kl}} = 0$ with standard deviation $\Delta z_{kl}$ given by eq. \eqref{eq:5:amplitude-variance}. 
Both, the variations in the particle-shield separation $\Delta L$ as well as in the angle $\theta$ are correlated to the amplitude $z$.
For a large shield, this can be understood as
\begin{equation}
  \theta = \arctan(z \abs{\nabla u}) \approx z \abs{\nabla u} \quad \text{and} \quad \Delta L = z \abs{u}
\end{equation}
where $\nabla u$ is the gradient of the shape of the vibrational mode.
Performing similar calculations as done before in \cref{cha:entanglement-generation}, the averaged density matrix $\mean{\rho}$ dependent on $\Delta z_{kl}$ can be calculated.
The entanglement quantified by the logarithmic negativity \cite{Plenio_2005} introduced in \cref{sec:2:entanglement-measures} dependent on the temperature $T$ is shown in \cref{fig:5:entanglement-temperature}.
\begin{figure}[!htbp]
  \centering
  \includegraphics[width=\textwidth]{./../figures/vibrations/entanglement-shield-T-L.pdf}
  \caption{Entanglement between the particles (parallel orientation) in front of a thermal shield in the first mode $(1,0)$ at temperature $T$ for different particle-shield separations $L$.}
  \label{fig:5:entanglement-temperature}
\end{figure}
This is not surprising considering that the thermal amplitudes $\Delta z_{1,0} \approx 9 \times 10^{-11}\si{m}$ at $20\si{mK}$ which is comparable with the previously calculated values for $\Delta L_\mathrm{crit}$ in \cref{cha:entanglement-generation}.
Surprisingly this result does not change for different shield radii - at least as long as the condition $r_s \ll R$ is fulfilled and the shield shape can locally be linearized.
This is because the gradient $\abs{\nabla u} \propto 1/r_s$ which perfectly cancels with the dependence on $z \propto r_s$ leaving $\theta$ independent of $r_s$. 
If the cat-state orientation is now chosen parallel to the shield, the dependence on $\Delta L$ is irrelevant leaving the final resulting entanglement independent of $r_s$. 



\subsection{Analytic dynamics}
Surprisingly the influence of the thermal shield on entanglement generation can be calculated analytically.
\begin{align}\label{eq:5:hamiltonian}
\begin{split}
  \op{H} = \sum_{\substack{m\in\{(k,l)\}\\ k \geq 1,\ l \geq 0}} & \hbar \omega_m \left(\op{a}^\dagger_m \op{a}_m + \frac{1}{2}\right) \\
  &+ \left[g^{1,1}_\mathrm{Grav} + \left(g^1_\mathrm{A,m,Cas} +
  g^1_\mathrm{B,m,Cas}\right)(\op{a}_m + \op{a}^\dagger_m)\right] \ketbra{\psi^1_A \psi^1_B} \\
  &+ \left[g^{1,2}_\mathrm{Grav} + \left(g^1_\mathrm{A,m,Cas} + g^2_\mathrm{B,m,Cas}\right)(\op{a}_m + \op{a}^\dagger_m)\right]\ketbra{\psi^1_A \psi^2_B} \\
  &+ \left[g^{2,1}_\mathrm{Grav} + \left(g^2_\mathrm{A,m,Cas} + g^1_\mathrm{B,m,Cas}\right)(\op{a}_m + \op{a}^\dagger_m)\right]\ketbra{\psi^2_A \psi^1_B} \\
  &+ \left[g^{2,2}_\mathrm{Grav} + \left(g^2_\mathrm{A,m,Cas} + g^2_\mathrm{B,m,Cas}\right)(\op{a}_m + \op{a}^\dagger_m)\right]\ketbra{\psi^2_A \psi^2_B}
\end{split}
\end{align}
where $g^{ij}_\mathrm{Grav}$ is the gravitational coupling between the states $\ket{\psi^i_A}$ and $\ket{\psi^j_B}$.
The Casimir interaction between state $\ket{\psi_{A(B)}^i}$ and the shield is denoted by $\tilde{g}^i_\mathrm{A(B),\,m,\,Cas}$. These couplings are dependent on the amplitude $\op{z}_{m} = \sqrt{\hbar/2\tilde{m}\omega_m} (\op{a}_m + \op{a}^\dagger_m)$ and the shape $u_{m}(r_{A(B)})$ of the vibrational mode $m=\{(k,l)\}$ at the position $r_{A(B),i}$ of the cat state:
\begin{equation}
  \tilde{g}^i_\mathrm{A(B),\,m,\,Cas} = \frac{\hbar c \pi^3}{720} \left(\frac{\varepsilon_r - 1}{\varepsilon_r + 1}\right)\varphi(\varepsilon_r) \frac{R}{(\mathscr{L} + \op{z}_m u_m(r))^2} \approx g_\mathrm{PFA} \left(\frac{1}{\mathscr{L}^2} + \frac{2 \op{z}_m u_m(r_{A(B),i})}{\mathscr{L}^3}\right) .
\end{equation}
Ignoring the first term in the expansion, which just produces a global phase in the evolved system, the couplings $g_\mathrm{Cas}$ appearing in eq. \eqref{eq:5:hamiltonian} are finally given by
\begin{equation}
  g^i_\mathrm{A(B),\,m,\,Cas} = g_\mathrm{PFA} \frac{2u_m(r_{A(B),i})}{\mathscr{L}^3} \sqrt{\frac{\hbar}{2 \tilde{m} \omega_m}} .
\end{equation}
It is possible to analytically calculate the time evolution of a system consisting of the initial state $\rho_\mathrm{system}$ (given by eq. \eqref{eq:2:initial-state}) combined with the infinite vibrational modes $\rho_\mathrm{th}$ of the thermal shield
\begin{equation}
  \rho_0 = \bigotimes_{m\in\{(k,l)\}} \left(\rho_\mathrm{th,\,m}\right) \otimes \rho_\mathrm{system} .
\end{equation}
These thermal states can be expanded into coherent states $\ket{\alpha} = \op{D}(\alpha)\ket{0}$ as \cite{Steiner_2024}
\begin{equation}
  \rho_\mathrm{th,m} = \frac{1}{Z} \sum_{n=1}^{\infty} e^{-\beta \hbar \omega_m (n + 1/2)} \ketbra{n} = \int \dd \alpha^2 \frac{1}{\pi \bar{n}} e^{-\frac{\abs{\alpha}^2}{\bar{n}}} \ketbra{\alpha}
\end{equation}
where $\bar{n}$ is the average occupation number. The time evolution of the particle-system $\rho_\mathrm{system}(t) = \tr_{th}\left\{\rho(t)\right\}$ can be calculated by tracing out all states corresponding to the thermal shield.

\begin{figure}[!htbp]
  \centering
  \includegraphics[width=\textwidth]{./../figures/vibrations/entanglement-hamiltonian.pdf}
  \caption{Entanglement dynamics in front of a thermal shield in mode $(1,0)$ at different temperatures. Only at specific times $2\pi k / \omega_{1,0},\ k\in\mathbb{N}$, entanglement is observable. This behavior is expected and aligns with the findings in Ref. \cite{Pedernales_2022}.}
\end{figure}

\begin{figure}[!htbp]
  \centering
  \includegraphics[width=\textwidth]{./../figures/vibrations/entanglement-multiple-modes.pdf}
  \caption{Entanglement dynamics in front of a thermal shield. In orange, the first 50 modes have been used in the numeric calculation. The effect of all remaining modes is around $1.7 \times 10^{-11}\,\%$. In blue, all modes except the first mode $(1,0)$ have been considered at different temperatures ranging vom $1\si{mK}$ up to $20\si{mK}$. The particle-shield separation is fixed at $L = 2R = 20\si{\mu m}$.}
\end{figure}

\begin{figure}[!htbp]
  \centering
  \includegraphics[width=\textwidth]{./../figures/vibrations/all-modes-maximum-entanglement-L-t-max.pdf}
  \caption{All modes, Entanglement at t-max}
\end{figure}


\subsection{Small shields}


\section{Discussion on the optimal setup}\label{sec:4:discussion}

- No global maximum

- dependent on experimental stuff

- Masses and distances and so on must be bounded by 10xECasimir < EGravity

How can one find the best parameters depending on the experimental requirements?
\begin{enumerate}
  \item Specify maximum coherence time (how long the experiment can take)
  \item This fixes $L^3/(M_A M_B \Delta x_A \Delta x_B)$
  \item Look up, what maximum $\Delta \theta$ and $\Delta L$ you can get at this time (Dependent on the amount of entanglement you want to have at the end)
  \item Take the minimum possible $L/R$ in account (dependent on trap stiffness and so on) Ideally, large enough to reduce casimir interactions
\end{enumerate}


-------------------------------------- 

\begin{enumerate}
  \item Given a target measuring time $t_\mathrm{target}$, this fixes the ratio
  \begin{equation}\label{eq:4:fixed-ratio}
    \frac{L^3}{M_A M_B \Delta x_A \Delta x_B} = 5.036\times 10^{22}\si{m/kg^2s} \cdot t_\mathrm{target} = \mathrm{const.}
  \end{equation}
  Most likely, the superposition size $\Delta x_{A,B}$ is also not changeable and one has to be satisfied by what can be achieved experimentally. The delocalized states of mass $M$ has to interact gravitationally with each other in a laboratory setting for at least the duration of the measuring process $t_\mathrm{target}$. These considerations limits the coherence time and usually, low times in the order of milliseconds to seconds are favorable.
  \item In principle, one does not require to measure at the time of maximum entanglement where $E_N = 1$. If a lower quantity of entanglement is enough for a set experimental goal, I recommend, measuring at a lower time than $t_\mathrm{target} = \tau t_\mathrm{max}$ ($\tau < 1$) !!!!FIGURE!!!!. This on the other hand increases the fixed ratio eq. \eqref{eq:4:fixed-ratio} by a factor of $1/\tau$.
  Therefore, it is possible to use smaller superposition sizes $\Delta x$, lighter masses or increase the distance $L$ which decreases Casimir interactions.
  \item Most likely, the minium angular 
\end{enumerate}
\chapter{The consequences of a thermal shield}\label{cha:the-shield}
Up until now, the dynamics and properties of the shield where neglected. At a non-zero temperature however, thermal vibrations of the shield may influence the generation of entanglement substantially.
In this chapter, firstly the required size of the shield is estimated. Then, the thermal vibrations of a large and small shield as well as the effect on entanglement generation is considered.
In the experiment, the particles are cooled down into the ground state for effective quantum control and the generation of spatial superpositions.
For cooling, often liquid helium at $T\approx 4\si{K}$ is used but cryogenic freezers can cool small setups down to the order of $T\approx 20\si{mK}$.
For all relevant calculations, these temperatures are used as a reference point.


\section{Thickness and size of the shield}\label{sec:5:shield-size}
The required thickness $d$ and the radius $r_s$ of the spherical shield to effectively block electromagnetic interactions between the particles can be estimated by analyzing the properties of a physical conductive material with high electric conductivity $\sigma$.
For an almost perfect shielding, a superconducting shield may be considerable as well.
The transmission $T$ of electromagnetic waves through a physical shield is given by \cite{Vandenbosch_2022}
\begin{equation}\label{eq:5:shield-transmission}
  T = \abs{\frac{\vec{E}_\mathrm{after}}{\vec{E}_\mathrm{before}}} = \frac{2}{Z_0 \sigma d}
\end{equation}
where $Z_0 = 377\si{\Omega}$ is the impedance of free space (assuming the shield is placed in a vacuum or in air).
The electric conductivity $\sigma$ is highly dependent on the temperature \cite[p. 284-286]{Gross_2018}, decreasing approximately with $1/T^5$ at low temperatures \footnote{This behavior is valid for temperatures below the Debye temperature ($\Theta_D = 343\si{K}$ for copper). At the low temperatures used in the experiment, this model accurately describes the conductivity of metals \cite{Berman_1952}.}.
Copper offers a strong electric conductivity of $\sigma = 59.6\times 10^6 \si{S/m}$ at room temperature and measured data showing even $\sigma(T = 10\si{K}) \approx 1.5\times 10^{10}\si{S/m}$ at $10\si{K}$ \cite{Berman_1952}.

To estimate the shield's thickness, the primary criterion used, is that gravitational interactions should dominate the entanglement generation.
Other mutual interactions between the particles, such as Coulomb or Casimir forces, must be sufficiently suppressed by the shield.
The \emph{entanglement rate} $\Gamma$ quantifies the build-up of entanglement over time
\begin{equation}\label{eq:5:entanglement-rate}
  \Gamma = \dv{t} E_N(\rho)\Big|_{t=0} \, ,
\end{equation} 
where $E_N$ is an appropriate entanglement measure - in this case the logarithmic negativity \cite{Plenio_2005} introduced in \cref{sec:2:entanglement-measures}.
For gravitational interactions, the entanglement rate in the parallel orientation is given by using eq. \eqref{eq:2:entanglement-dynamics-parallel} as
\begin{equation}\label{eq:5:entanglement-rate-gravity}
  \Gamma_\mathrm{Gravity} = \frac{G M_A M_B \Delta x_A \Delta x_B}{16 \hbar L^3 \log 2} = \frac{G \pi^2 R^6 \rho_\mathrm{Silica}^2 (\Delta x)^2}{9 \hbar L^3 \log 2} .
\end{equation}
where in the last step $M_A = M_B = 4/3 \pi R^3 \rho_\mathrm{Silica}$ and $\Delta x_A = \Delta x_B \equiv \Delta x$ was used.
The entanglement rate for non-gravitational interactions, such as Coulomb or Casimir forces, must be significantly smaller than the gravitational entanglement rate, ideally by a factor $\chi \gg 1$.
This ensures that any measured entanglement is primarily due to gravitational interactions.
In the following sections, estimations about the thickness and size of the shield are made, to effectively screen Coulomb and Casimir forces.


\subsection{Shielding Coulomb-Interactions}
The primary role of the Faraday shield is to block electromagnetic interactions between particles.
Experimentally, it may be beneficial for the particles to carry a small amount of charge, enabling the use of electrostatic traps with high trapping strength and large controllability \cite{GonzalezBallestero_2021}. 
The Coulomb interaction potential between two charged particles is given by
\begin{equation}
  V = \frac{1}{4\pi\varepsilon_0} \frac{q_A q_B}{2L}
\end{equation}
where $\varepsilon_0 = 8.8542\times 10^{-12}\si{A^2 s^4 m^{-3} kg^{-1}}$ is the permittivity of free space and we assume that each particle caries one electron charge $\abs{q_{A(B)}} = e = 1.6022\times 10^{-19}\si{C}$.
This interaction mimics the form of the gravitational potential and can similarly induce entanglement with a entanglement rate
\begin{equation}\label{eq:5:entanglement-rate-coulomb}
  \Gamma_\mathrm{Coulomb} = \frac{T \abs{q_A q_B} (\Delta x)^2}{64 \pi \varepsilon_0 \hbar L^3 \log 2} .
\end{equation}
The shield suppresses the coupling by a factor of $T$.
Requiring $\Gamma_\mathrm{Gravity} > \chi \Gamma_\mathrm{Coulomb}$, the minimum thickness of the shield can be calculated by using eq. \eqref{eq:5:shield-transmission} as
\begin{align}\label{eq:5:coulomb-gravity-condition}
  T \frac{\abs{q_A q_B}}{64 \pi \varepsilon_0} \chi \, &< \, \frac{G \pi^2 R^6 \rho_\mathrm{Silica}^2}{9} \\
  \Longleftrightarrow \quad\quad\quad\quad d \, &> \, \frac{9}{32}\frac{1}{Z_0 \sigma} \frac{1}{\pi^3 \varepsilon_0 G \rho_\mathrm{Silica}^2} \frac{e^2}{R^6} \chi .
\end{align}
The thickness strongly depends on the particles size $R$, and large or heavy particles will favor gravitational entanglement generation.
Assuming the particles are silica nano-spheres with parameters given in \cref{tab:paramters}, a minimum shield-thickness of $d \approx 10\si{nm}\chi$ at $4\si{K}$ and of $d \approx 2.5\si{\mu m}\chi$ at room temperature is required.
At low temperatures, a realistic shield thickness could therefore be $d=100\si{nm}$, balancing engineering practicality and electromagnetic suppression.
Exact estimations however depend on the realization of the experiment as well as the precision in which the evolved state is measurable.

Electrostatic fields still can propagate around the finite-sized Faraday shield and potentially induce entanglement.
It is possible to estimate the required shield radius $r_s$ to block a specific amount $\eta$ of the electric field (see \cref{apx:blocking-of-the-shield}):
\begin{equation}\label{eq:5:shield-effectiveness}
  \frac{r_s}{L} = \sqrt{\frac{1-(1-\eta)^2}{(1-\eta)^2}}
\end{equation}
The results are visualized in \cref{fig:5:shield-radius}.
\begin{figure}[!ht]
  \centering
  \includegraphics[width=\textwidth]{./../figures/others/shield-radius.pdf}
  \caption{Shield radius as a function of the shielding effectiveness $\eta$ for an ideal shield. Additionally, a real shield with varying thicknesses $d$ is considered at $T=300\si{K}$.
  To achieve shielding of $99.5-99.9\%$ ($\eta = 0.995-0.999$), a radius of $r_s =200-1000L$ is needed.}
  \label{fig:5:shield-radius}
\end{figure}
The shield's transmission $T$ should therefore be modified to $\tilde{T} = T\eta + (1-\eta)$, where the shielding effectiveness $\eta$ depends on $r_s$ as given by eq. \eqref{eq:5:shield-effectiveness}.
Modifying eq. \eqref{eq:5:coulomb-gravity-condition}, a minimum effectiveness $\eta_\mathrm{min}$ for sufficient shielding of
\begin{equation}
  \eta_\mathrm{min} \approx 1 - \frac{64\pi^3 \varepsilon_0 G R^6 \rho^2_\mathrm{Silica}}{9 e^2}
\end{equation}
can be approximated.
Using again the parameters from \cref{tab:paramters}, a minimum shielding of $\eta_\mathrm{min} \gtrsim 0.99997$ and thus a radius of $r_s \gtrsim 28000 L \approx 60\si{cm}$ is required.
Depending on the experimental setup, such a shield might be too large for all practical purposes and it may be beneficial to choose heavier masses ($\tilde{M} \sim 4 M$) to reduce the shield size to the orders of $\sim 1\si{cm}$.
Due to practicality, a shield with $r_s = 1\si{cm}$ is used in the following calculations.
Uncharged particles would eliminate the Coulomb interactions and therefore reducing the shield's size and thickness to only shield mutual Casimir interactions.




\subsection{Shielding Casimir-Interactions}
Similarly to Coulomb interactions, it is possible to to estimate the required thickness for a shield to sufficiently block Casimir interactions.
The Casimir potential between the spheres with radius $R$ separated by $2L$ is given in first order by \cite{Emig_2007}
\begin{equation}
  V = -\frac{23 \hbar c}{4\pi \cdot 128 L^7} \left( \frac{\varepsilon_r - 1}{\varepsilon_r + 2} \right)^2 R^6 .
\end{equation}
The corresponding entanglement rate is calculated similar to before by expanding the potential in small $\Delta x$ and computing the logarithmic negativity:
\begin{equation}
  \Gamma_\mathrm{Casimir} = T^2 \frac{161}{4096} \frac{c R^6 (\Delta x)^2}{\pi L^9 \log 2}\left( \frac{\varepsilon_r - 1}{\varepsilon_r + 2}\right)^2 .
\end{equation}
The dependence on $T^2$ arises because Casimir forces are second order effects in the dipole-dipole interaction \cite{Bordag_2001}.
Requiring gravitational entanglement generation to dominate, $\Gamma_\mathrm{Gravity} > \chi \Gamma_\mathrm{Casimir}$, leads to
\begin{align}
  T^2 \frac{161 c R^6}{4096 \pi L^6} \left( \frac{\varepsilon_r - 1}{\varepsilon_r + 2}\right)^2 \chi \, &< \, \frac{G \pi^2 \rho_\mathrm{Silica}^2 R^6}{9\hbar} \\
  \Longleftrightarrow \quad\quad\quad\  d \, &> \, \sqrt{\frac{1449}{4096} \frac{c \hbar}{G \pi^3}} \frac{2}{Z_0 \sigma \rho L^3} \frac{\varepsilon_r - 1}{\varepsilon_r + 2} \sqrt{\chi} ,
\end{align}
where again $T = 2/Z_0 \sigma d$ from eq. \eqref{eq:5:shield-transmission} was used.
For large separations, the shield thickness can be arbitrarily low, as Casimir forces vanish and at separations of $L\gtrsim 100\si{\mu m}$, the shield might not be required at all (compare \cref{sec:2:experimental-problems}).
Assuming two identical silica nano-spheres with parameters given by \cref{tab:paramters}, the required minimum thickness is between $4\times 10^{-11}\si{m} \sqrt{\chi}$ at $4\si{K}$ and $10 \si{nm} \sqrt{\chi}$ at room temperature.
This is much thinner than what is required for shielding Coulomb interactions.
The factor $\varepsilon_r$ modifies the thickness only by up to a constant factor of $\leq 1$ and is therefore ignored for worst-case estimations.

However, very thin shields lose mechanical rigidity, leading to enhanced vibrational excitations and potential instabilities.
Vibrational frequency and thus the vibrational energy depends linearly on the shield's thickness, making thinner shields prone to large decoherence due to thermal vibrations.
A detailed analysis of these effects is provided in the subsequent section.



\subsection{Gravitational effects of the shield}\label{subsec:5:shield-gravitation}
The gravitational interaction between the masses and the shield is generally neglected because it has no significant impact on the entanglement generation between the particles.
The only potential effect is indirect entanglement mediated by the thermal oscillations of the shield, as both masses couple gravitationally to it. 
However, as shown in \cref{sec:5:thermal-entanglement}, this second-order effect is very weak and does not pose a problem, since it still represents gravitationally mediated entanglement - which is the focus of the experiment anyway.
The gravitational force between a sphere with mass $M$ and an infinitesimal mass segment $\dd m = r d \rho_\mathrm{Cu} \dd r \dd \varphi$ of the shield made of copper with density $\rho_\mathrm{Cu} = 8960\si{kg/m^3}$ at a distance $r$ from the shield's center is given by
\begin{equation}
  \dd \vec{F} = \frac{G M \dd m}{\ell} \boldsymbol{\hat{\ell}} 
  \quad \Rightarrow \quad
  \dd F_z = \frac{G M r \rho_\mathrm{Cu} d}{\ell^2} \dd r \dd \varphi \cos \theta,
\end{equation}
where $\ell^2 = r^2 + L^2$ denotes the distance between the sphere and the mass segment and $\theta = \arccos L/\ell$ is the angle between them.
The total attractive force between the mass and the shield with radius $r_s$ is therefore
\begin{equation}
  F_z = GM \rho_\mathrm{Cu} d L \int\limits_{0}^{r_s} \dd r \int\limits_{0}^{2\pi} \dd \varphi \, \frac{r}{(r^2 + L^2)^{3/2}} = 2\pi G M \rho_\mathrm{Cu} d \left(1 - \frac{L}{\sqrt{L^2 + r_s^2}}\right) .
\end{equation}
For large shields $r_s \rightarrow \infty$ this is independent of the particle-shield separation $L$.
For a shield with thickness $d = 100\si{nm}$ and a silica particle with parameters given in \cref{tab:paramters}, the attraction force is around $F_\mathrm{particle-shield} \approx 4.1\times 10^{-24} \si{N}$ which is comparable with the attraction gravitational attraction force between the two particles themselves at $F_\mathrm{particle-particle} \approx 5.0 \times 10^{-24}\si{N}$ but is much weaker than the Casimir attraction between the particle and the shield with $F_\mathrm{Casimir} \approx 1.4 \times 10^{-17} \si{N}$.
Therefore, the gravitational effect of the shield can be neglected in all practical calculations.

\section{Thermal shield vibrations}\label{sec:4:thermal-vibrations}

\begin{figure}[!htbp]
  \centering
  \includegraphics[width=\textwidth]{./../figures/vibrations/vibrational-modes.pdf}
  \caption{Normalized shape of the vibrational modes $(k,l)$ of a vibrating spherical plate fixed at the edge with $R/d = 1000$.}
  \label{fig:4:vibrational-modes}
\end{figure}



\begin{equation}
  \op{H} = \hbar \omega \left(\op{a}^\dagger\op{a} + \frac{1}{2}\right) + \tilde{g}_A(\op{z}) \ketbra{\psi_A} + \tilde{g}_B(\op{z}) \ketbra{\psi_B}
\end{equation}

Solution:
\begin{equation}
  \rho(t) = \frac{1}{2}\begin{pmatrix}
    1 & e^{-i\varphi(t)}e^{-\gamma(t)}\\
    e^{i\varphi(t)}e^{-\gamma(t)} & 1
  \end{pmatrix}
\end{equation}
with the phase
\begin{equation}
  \varphi(t) = \frac{1}{\hbar^2\omega^2} \left(\sin(\omega t) - \omega t\right) \left(g_A^2 - g_B^2\right)
\end{equation}
and the decoherence term
\begin{equation}
  \gamma(t) = \frac{4(g_A - g_B)^2}{\hbar^2 \omega^2} \sin(\frac{\omega t}{2}) \left[\bar{n} + \frac{1}{2}\right]
\end{equation}

\begin{figure}[!htbp]
  \centering
  \includegraphics[width=\textwidth]{./../figures/vibrations/decoherence-analytical.pdf}
  \caption{Maximum decoherence $\gamma$ at $4\si{K}$ for all modes if the cat-state is placed \textbf{left:} at the point with maximum gradient of the mode $(1,0)$ and \textbf{right:} in the center of the plate. It becomes evident, that only a few low modes play an actual role for the total dephasing.}
  \label{fig:4:}
\end{figure}

\section{Entanglement in front of a thermal shield}\label{sec:5:thermal-entanglement}
The entanglement generation between the two particles depends heavily on the variation of the separation between the shield and the particles, as has been seen in \cref{cha:entanglement-generation}.
The vibrating shield can be interpreted as varying the separation and angle of the cat-state in front of the plate - visualized in \cref{fig:5:vibrating-translation-to-variations}.
\begin{figure}[!htbp]
  \centering
  \def\svgwidth{\textwidth}
  \input{./../figures/plate-vibration.pdf_tex}
  \caption{For a large $r_s \gg R$ and locally flat shield, the thermal vibrations with amplitude $z$ can be interpreted as a static shield where the particle $A$ (shown in the figure) is placed at $L+\Delta L$ at angle $\theta$ and particle $B$ is places at $L-\Delta L$ with angle $-\theta$ where both variations depend on the amplitude. At low vibrational frequencies $1/\omega \approx t_\mathrm{max}$ the amplitude can be assumed to be static during a experimental run and for each measurement thermally distributed around $\mean{z}=0$ with $\Delta z$ given by eq. \eqref{eq:5:amplitude-variance}.}
  \label{fig:5:vibrating-translation-to-variations}
\end{figure}
This is only a good approximation for shields larger than the particles radius $r_s \gg R$ and low vibrating frequencies $1/\omega \approx t_\mathrm{max}$ and can therefore be used well to describe the highly disturbing first few modes on a large shield.
Furthermore, this interpretation is possible because as shown in \cref{sec:3:imperfect-plates}, the Casimir interaction between a sphere and a tilted plane does not differ from the interaction between a flat plane.
Contrary to the problem considered in \cref{cha:entanglement-generation}, here only the thermal amplitude $z_{kl}$ is a independent random variable distributed around $\mean{z_{kl}} = 0$ with standard deviation $\Delta z_{kl}$ given by eq. \eqref{eq:5:amplitude-variance}. 
Both, the variations in the particle-shield separation $\Delta L$ as well as in the angle $\theta$ are correlated to the amplitude $z$.
For a large shield, this can be understood as
\begin{equation}
  \theta = \arctan(z \abs{\nabla u}) \approx z \abs{\nabla u} \quad \text{and} \quad \Delta L = z \abs{u}
\end{equation}
where $\nabla u$ is the gradient of the shape of the vibrational mode.
Performing similar calculations as done before in \cref{cha:entanglement-generation}, the averaged density matrix $\mean{\rho}$ dependent on $\Delta z_{kl}$ can be calculated.
The entanglement quantified by the logarithmic negativity \cite{Plenio_2005} introduced in \cref{sec:2:entanglement-measures} dependent on the temperature $T$ is shown in \cref{fig:5:entanglement-temperature}.
\begin{figure}[!htbp]
  \centering
  \includegraphics[width=\textwidth]{./../figures/vibrations/entanglement-shield-T-L.pdf}
  \caption{Entanglement between the particles (parallel orientation) in front of a thermal shield in the first mode $(1,0)$ at temperature $T$ for different particle-shield separations $L$.}
  \label{fig:5:entanglement-temperature}
\end{figure}
This is not surprising considering that the thermal amplitudes $\Delta z_{1,0} \approx 9 \times 10^{-11}\si{m}$ at $20\si{mK}$ which is comparable with the previously calculated values for $\Delta L_\mathrm{crit}$ in \cref{cha:entanglement-generation}.
Surprisingly this result does not change for different shield radii - at least as long as the condition $r_s \ll R$ is fulfilled and the shield shape can locally be linearized.
This is because the gradient $\abs{\nabla u} \propto 1/r_s$ which perfectly cancels with the dependence on $z \propto r_s$ leaving $\theta$ independent of $r_s$. 
If the cat-state orientation is now chosen parallel to the shield, the dependence on $\Delta L$ is irrelevant leaving the final resulting entanglement independent of $r_s$. 



\subsection{Analytic dynamics}
Surprisingly the influence of the thermal shield on entanglement generation can be calculated analytically.
\begin{align}\label{eq:5:hamiltonian}
\begin{split}
  \op{H} = \sum_{\substack{m\in\{(k,l)\}\\ k \geq 1,\ l \geq 0}} & \hbar \omega_m \left(\op{a}^\dagger_m \op{a}_m + \frac{1}{2}\right) \\
  &+ \left[g^{1,1}_\mathrm{Grav} + \left(g^1_\mathrm{A,m,Cas} +
  g^1_\mathrm{B,m,Cas}\right)(\op{a}_m + \op{a}^\dagger_m)\right] \ketbra{\psi^1_A \psi^1_B} \\
  &+ \left[g^{1,2}_\mathrm{Grav} + \left(g^1_\mathrm{A,m,Cas} + g^2_\mathrm{B,m,Cas}\right)(\op{a}_m + \op{a}^\dagger_m)\right]\ketbra{\psi^1_A \psi^2_B} \\
  &+ \left[g^{2,1}_\mathrm{Grav} + \left(g^2_\mathrm{A,m,Cas} + g^1_\mathrm{B,m,Cas}\right)(\op{a}_m + \op{a}^\dagger_m)\right]\ketbra{\psi^2_A \psi^1_B} \\
  &+ \left[g^{2,2}_\mathrm{Grav} + \left(g^2_\mathrm{A,m,Cas} + g^2_\mathrm{B,m,Cas}\right)(\op{a}_m + \op{a}^\dagger_m)\right]\ketbra{\psi^2_A \psi^2_B}
\end{split}
\end{align}
where $g^{ij}_\mathrm{Grav}$ is the gravitational coupling between the states $\ket{\psi^i_A}$ and $\ket{\psi^j_B}$.
The Casimir interaction between state $\ket{\psi_{A(B)}^i}$ and the shield is denoted by $\tilde{g}^i_\mathrm{A(B),\,m,\,Cas}$. These couplings are dependent on the amplitude $\op{z}_{m} = \sqrt{\hbar/2\tilde{m}\omega_m} (\op{a}_m + \op{a}^\dagger_m)$ and the shape $u_{m}(r_{A(B)})$ of the vibrational mode $m=\{(k,l)\}$ at the position $r_{A(B),i}$ of the cat state:
\begin{equation}
  \tilde{g}^i_\mathrm{A(B),\,m,\,Cas} = \frac{\hbar c \pi^3}{720} \left(\frac{\varepsilon_r - 1}{\varepsilon_r + 1}\right)\varphi(\varepsilon_r) \frac{R}{(\mathscr{L} + \op{z}_m u_m(r))^2} \approx g_\mathrm{PFA} \left(\frac{1}{\mathscr{L}^2} + \frac{2 \op{z}_m u_m(r_{A(B),i})}{\mathscr{L}^3}\right) .
\end{equation}
Ignoring the first term in the expansion, which just produces a global phase in the evolved system, the couplings $g_\mathrm{Cas}$ appearing in eq. \eqref{eq:5:hamiltonian} are finally given by
\begin{equation}
  g^i_\mathrm{A(B),\,m,\,Cas} = g_\mathrm{PFA} \frac{2u_m(r_{A(B),i})}{\mathscr{L}^3} \sqrt{\frac{\hbar}{2 \tilde{m} \omega_m}} .
\end{equation}
It is possible to analytically calculate the time evolution of a system consisting of the initial state $\rho_\mathrm{system}$ (given by eq. \eqref{eq:2:initial-state}) combined with the infinite vibrational modes $\rho_\mathrm{th}$ of the thermal shield
\begin{equation}
  \rho_0 = \bigotimes_{m\in\{(k,l)\}} \left(\rho_\mathrm{th,\,m}\right) \otimes \rho_\mathrm{system} .
\end{equation}
These thermal states can be expanded into coherent states $\ket{\alpha} = \op{D}(\alpha)\ket{0}$ as \cite{Steiner_2024}
\begin{equation}
  \rho_\mathrm{th,m} = \frac{1}{Z} \sum_{n=1}^{\infty} e^{-\beta \hbar \omega_m (n + 1/2)} \ketbra{n} = \int \dd \alpha^2 \frac{1}{\pi \bar{n}} e^{-\frac{\abs{\alpha}^2}{\bar{n}}} \ketbra{\alpha}
\end{equation}
where $\bar{n}$ is the average occupation number. The time evolution of the particle-system $\rho_\mathrm{system}(t) = \tr_{th}\left\{\rho(t)\right\}$ can be calculated by tracing out all states corresponding to the thermal shield.

\begin{figure}[!htbp]
  \centering
  \includegraphics[width=\textwidth]{./../figures/vibrations/entanglement-hamiltonian.pdf}
  \caption{Entanglement dynamics in front of a thermal shield in mode $(1,0)$ at different temperatures. Only at specific times $2\pi k / \omega_{1,0},\ k\in\mathbb{N}$, entanglement is observable. This behavior is expected and aligns with the findings in Ref. \cite{Pedernales_2022}.}
\end{figure}

\begin{figure}[!htbp]
  \centering
  \includegraphics[width=\textwidth]{./../figures/vibrations/entanglement-multiple-modes.pdf}
  \caption{Entanglement dynamics in front of a thermal shield. In orange, the first 50 modes have been used in the numeric calculation. The effect of all remaining modes is around $1.7 \times 10^{-11}\,\%$. In blue, all modes except the first mode $(1,0)$ have been considered at different temperatures ranging vom $1\si{mK}$ up to $20\si{mK}$. The particle-shield separation is fixed at $L = 2R = 20\si{\mu m}$.}
\end{figure}

\begin{figure}[!htbp]
  \centering
  \includegraphics[width=\textwidth]{./../figures/vibrations/all-modes-maximum-entanglement-L-t-max.pdf}
  \caption{All modes, Entanglement at t-max}
\end{figure}


\subsection{Small shields}


\section{Discussion on the optimal setup}\label{sec:4:discussion}

- No global maximum

- dependent on experimental stuff

- Masses and distances and so on must be bounded by 10xECasimir < EGravity

How can one find the best parameters depending on the experimental requirements?
\begin{enumerate}
  \item Specify maximum coherence time (how long the experiment can take)
  \item This fixes $L^3/(M_A M_B \Delta x_A \Delta x_B)$
  \item Look up, what maximum $\Delta \theta$ and $\Delta L$ you can get at this time (Dependent on the amount of entanglement you want to have at the end)
  \item Take the minimum possible $L/R$ in account (dependent on trap stiffness and so on) Ideally, large enough to reduce casimir interactions
\end{enumerate}


-------------------------------------- 

\begin{enumerate}
  \item Given a target measuring time $t_\mathrm{target}$, this fixes the ratio
  \begin{equation}\label{eq:4:fixed-ratio}
    \frac{L^3}{M_A M_B \Delta x_A \Delta x_B} = 5.036\times 10^{22}\si{m/kg^2s} \cdot t_\mathrm{target} = \mathrm{const.}
  \end{equation}
  Most likely, the superposition size $\Delta x_{A,B}$ is also not changeable and one has to be satisfied by what can be achieved experimentally. The delocalized states of mass $M$ has to interact gravitationally with each other in a laboratory setting for at least the duration of the measuring process $t_\mathrm{target}$. These considerations limits the coherence time and usually, low times in the order of milliseconds to seconds are favorable.
  \item In principle, one does not require to measure at the time of maximum entanglement where $E_N = 1$. If a lower quantity of entanglement is enough for a set experimental goal, I recommend, measuring at a lower time than $t_\mathrm{target} = \tau t_\mathrm{max}$ ($\tau < 1$) !!!!FIGURE!!!!. This on the other hand increases the fixed ratio eq. \eqref{eq:4:fixed-ratio} by a factor of $1/\tau$.
  Therefore, it is possible to use smaller superposition sizes $\Delta x$, lighter masses or increase the distance $L$ which decreases Casimir interactions.
  \item Most likely, the minium angular 
\end{enumerate}
\chapter{The consequences of a thermal shield}\label{cha:the-shield}
Up until now, the dynamics and properties of the shield where neglected. At a non-zero temperature however, thermal vibrations of the shield may influence the generation of entanglement substantially.
In this chapter, firstly the required size of the shield is estimated. Then, the thermal vibrations of a large and small shield as well as the effect on entanglement generation is considered.
In the experiment, the particles are cooled down into the ground state for effective quantum control and the generation of spatial superpositions.
For cooling, often liquid helium at $T\approx 4\si{K}$ is used but cryogenic freezers can cool small setups down to the order of $T\approx 20\si{mK}$.
For all relevant calculations, these temperatures are used as a reference point.


\section{Thickness and size of the shield}\label{sec:5:shield-size}
The required thickness $d$ and the radius $r_s$ of the spherical shield to effectively block electromagnetic interactions between the particles can be estimated by analyzing the properties of a physical conductive material with high electric conductivity $\sigma$.
For an almost perfect shielding, a superconducting shield may be considerable as well.
The transmission $T$ of electromagnetic waves through a physical shield is given by \cite{Vandenbosch_2022}
\begin{equation}\label{eq:5:shield-transmission}
  T = \abs{\frac{\vec{E}_\mathrm{after}}{\vec{E}_\mathrm{before}}} = \frac{2}{Z_0 \sigma d}
\end{equation}
where $Z_0 = 377\si{\Omega}$ is the impedance of free space (assuming the shield is placed in a vacuum or in air).
The electric conductivity $\sigma$ is highly dependent on the temperature \cite[p. 284-286]{Gross_2018}, decreasing approximately with $1/T^5$ at low temperatures \footnote{This behavior is valid for temperatures below the Debye temperature ($\Theta_D = 343\si{K}$ for copper). At the low temperatures used in the experiment, this model accurately describes the conductivity of metals \cite{Berman_1952}.}.
Copper offers a strong electric conductivity of $\sigma = 59.6\times 10^6 \si{S/m}$ at room temperature and measured data showing even $\sigma(T = 10\si{K}) \approx 1.5\times 10^{10}\si{S/m}$ at $10\si{K}$ \cite{Berman_1952}.

To estimate the shield's thickness, the primary criterion used, is that gravitational interactions should dominate the entanglement generation.
Other mutual interactions between the particles, such as Coulomb or Casimir forces, must be sufficiently suppressed by the shield.
The \emph{entanglement rate} $\Gamma$ quantifies the build-up of entanglement over time
\begin{equation}\label{eq:5:entanglement-rate}
  \Gamma = \dv{t} E_N(\rho)\Big|_{t=0} \, ,
\end{equation} 
where $E_N$ is an appropriate entanglement measure - in this case the logarithmic negativity \cite{Plenio_2005} introduced in \cref{sec:2:entanglement-measures}.
For gravitational interactions, the entanglement rate in the parallel orientation is given by using eq. \eqref{eq:2:entanglement-dynamics-parallel} as
\begin{equation}\label{eq:5:entanglement-rate-gravity}
  \Gamma_\mathrm{Gravity} = \frac{G M_A M_B \Delta x_A \Delta x_B}{16 \hbar L^3 \log 2} = \frac{G \pi^2 R^6 \rho_\mathrm{Silica}^2 (\Delta x)^2}{9 \hbar L^3 \log 2} .
\end{equation}
where in the last step $M_A = M_B = 4/3 \pi R^3 \rho_\mathrm{Silica}$ and $\Delta x_A = \Delta x_B \equiv \Delta x$ was used.
The entanglement rate for non-gravitational interactions, such as Coulomb or Casimir forces, must be significantly smaller than the gravitational entanglement rate, ideally by a factor $\chi \gg 1$.
This ensures that any measured entanglement is primarily due to gravitational interactions.
In the following sections, estimations about the thickness and size of the shield are made, to effectively screen Coulomb and Casimir forces.


\subsection{Shielding Coulomb-Interactions}
The primary role of the Faraday shield is to block electromagnetic interactions between particles.
Experimentally, it may be beneficial for the particles to carry a small amount of charge, enabling the use of electrostatic traps with high trapping strength and large controllability \cite{GonzalezBallestero_2021}. 
The Coulomb interaction potential between two charged particles is given by
\begin{equation}
  V = \frac{1}{4\pi\varepsilon_0} \frac{q_A q_B}{2L}
\end{equation}
where $\varepsilon_0 = 8.8542\times 10^{-12}\si{A^2 s^4 m^{-3} kg^{-1}}$ is the permittivity of free space and we assume that each particle caries one electron charge $\abs{q_{A(B)}} = e = 1.6022\times 10^{-19}\si{C}$.
This interaction mimics the form of the gravitational potential and can similarly induce entanglement with a entanglement rate
\begin{equation}\label{eq:5:entanglement-rate-coulomb}
  \Gamma_\mathrm{Coulomb} = \frac{T \abs{q_A q_B} (\Delta x)^2}{64 \pi \varepsilon_0 \hbar L^3 \log 2} .
\end{equation}
The shield suppresses the coupling by a factor of $T$.
Requiring $\Gamma_\mathrm{Gravity} > \chi \Gamma_\mathrm{Coulomb}$, the minimum thickness of the shield can be calculated by using eq. \eqref{eq:5:shield-transmission} as
\begin{align}\label{eq:5:coulomb-gravity-condition}
  T \frac{\abs{q_A q_B}}{64 \pi \varepsilon_0} \chi \, &< \, \frac{G \pi^2 R^6 \rho_\mathrm{Silica}^2}{9} \\
  \Longleftrightarrow \quad\quad\quad\quad d \, &> \, \frac{9}{32}\frac{1}{Z_0 \sigma} \frac{1}{\pi^3 \varepsilon_0 G \rho_\mathrm{Silica}^2} \frac{e^2}{R^6} \chi .
\end{align}
The thickness strongly depends on the particles size $R$, and large or heavy particles will favor gravitational entanglement generation.
Assuming the particles are silica nano-spheres with parameters given in \cref{tab:paramters}, a minimum shield-thickness of $d \approx 10\si{nm}\chi$ at $4\si{K}$ and of $d \approx 2.5\si{\mu m}\chi$ at room temperature is required.
At low temperatures, a realistic shield thickness could therefore be $d=100\si{nm}$, balancing engineering practicality and electromagnetic suppression.
Exact estimations however depend on the realization of the experiment as well as the precision in which the evolved state is measurable.

Electrostatic fields still can propagate around the finite-sized Faraday shield and potentially induce entanglement.
It is possible to estimate the required shield radius $r_s$ to block a specific amount $\eta$ of the electric field (see \cref{apx:blocking-of-the-shield}):
\begin{equation}\label{eq:5:shield-effectiveness}
  \frac{r_s}{L} = \sqrt{\frac{1-(1-\eta)^2}{(1-\eta)^2}}
\end{equation}
The results are visualized in \cref{fig:5:shield-radius}.
\begin{figure}[!ht]
  \centering
  \includegraphics[width=\textwidth]{./../figures/others/shield-radius.pdf}
  \caption{Shield radius as a function of the shielding effectiveness $\eta$ for an ideal shield. Additionally, a real shield with varying thicknesses $d$ is considered at $T=300\si{K}$.
  To achieve shielding of $99.5-99.9\%$ ($\eta = 0.995-0.999$), a radius of $r_s =200-1000L$ is needed.}
  \label{fig:5:shield-radius}
\end{figure}
The shield's transmission $T$ should therefore be modified to $\tilde{T} = T\eta + (1-\eta)$, where the shielding effectiveness $\eta$ depends on $r_s$ as given by eq. \eqref{eq:5:shield-effectiveness}.
Modifying eq. \eqref{eq:5:coulomb-gravity-condition}, a minimum effectiveness $\eta_\mathrm{min}$ for sufficient shielding of
\begin{equation}
  \eta_\mathrm{min} \approx 1 - \frac{64\pi^3 \varepsilon_0 G R^6 \rho^2_\mathrm{Silica}}{9 e^2}
\end{equation}
can be approximated.
Using again the parameters from \cref{tab:paramters}, a minimum shielding of $\eta_\mathrm{min} \gtrsim 0.99997$ and thus a radius of $r_s \gtrsim 28000 L \approx 60\si{cm}$ is required.
Depending on the experimental setup, such a shield might be too large for all practical purposes and it may be beneficial to choose heavier masses ($\tilde{M} \sim 4 M$) to reduce the shield size to the orders of $\sim 1\si{cm}$.
Due to practicality, a shield with $r_s = 1\si{cm}$ is used in the following calculations.
Uncharged particles would eliminate the Coulomb interactions and therefore reducing the shield's size and thickness to only shield mutual Casimir interactions.




\subsection{Shielding Casimir-Interactions}
Similarly to Coulomb interactions, it is possible to to estimate the required thickness for a shield to sufficiently block Casimir interactions.
The Casimir potential between the spheres with radius $R$ separated by $2L$ is given in first order by \cite{Emig_2007}
\begin{equation}
  V = -\frac{23 \hbar c}{4\pi \cdot 128 L^7} \left( \frac{\varepsilon_r - 1}{\varepsilon_r + 2} \right)^2 R^6 .
\end{equation}
The corresponding entanglement rate is calculated similar to before by expanding the potential in small $\Delta x$ and computing the logarithmic negativity:
\begin{equation}
  \Gamma_\mathrm{Casimir} = T^2 \frac{161}{4096} \frac{c R^6 (\Delta x)^2}{\pi L^9 \log 2}\left( \frac{\varepsilon_r - 1}{\varepsilon_r + 2}\right)^2 .
\end{equation}
The dependence on $T^2$ arises because Casimir forces are second order effects in the dipole-dipole interaction \cite{Bordag_2001}.
Requiring gravitational entanglement generation to dominate, $\Gamma_\mathrm{Gravity} > \chi \Gamma_\mathrm{Casimir}$, leads to
\begin{align}
  T^2 \frac{161 c R^6}{4096 \pi L^6} \left( \frac{\varepsilon_r - 1}{\varepsilon_r + 2}\right)^2 \chi \, &< \, \frac{G \pi^2 \rho_\mathrm{Silica}^2 R^6}{9\hbar} \\
  \Longleftrightarrow \quad\quad\quad\  d \, &> \, \sqrt{\frac{1449}{4096} \frac{c \hbar}{G \pi^3}} \frac{2}{Z_0 \sigma \rho L^3} \frac{\varepsilon_r - 1}{\varepsilon_r + 2} \sqrt{\chi} ,
\end{align}
where again $T = 2/Z_0 \sigma d$ from eq. \eqref{eq:5:shield-transmission} was used.
For large separations, the shield thickness can be arbitrarily low, as Casimir forces vanish and at separations of $L\gtrsim 100\si{\mu m}$, the shield might not be required at all (compare \cref{sec:2:experimental-problems}).
Assuming two identical silica nano-spheres with parameters given by \cref{tab:paramters}, the required minimum thickness is between $4\times 10^{-11}\si{m} \sqrt{\chi}$ at $4\si{K}$ and $10 \si{nm} \sqrt{\chi}$ at room temperature.
This is much thinner than what is required for shielding Coulomb interactions.
The factor $\varepsilon_r$ modifies the thickness only by up to a constant factor of $\leq 1$ and is therefore ignored for worst-case estimations.

However, very thin shields lose mechanical rigidity, leading to enhanced vibrational excitations and potential instabilities.
Vibrational frequency and thus the vibrational energy depends linearly on the shield's thickness, making thinner shields prone to large decoherence due to thermal vibrations.
A detailed analysis of these effects is provided in the subsequent section.



\subsection{Gravitational effects of the shield}\label{subsec:5:shield-gravitation}
The gravitational interaction between the masses and the shield is generally neglected because it has no significant impact on the entanglement generation between the particles.
The only potential effect is indirect entanglement mediated by the thermal oscillations of the shield, as both masses couple gravitationally to it. 
However, as shown in \cref{sec:5:thermal-entanglement}, this second-order effect is very weak and does not pose a problem, since it still represents gravitationally mediated entanglement - which is the focus of the experiment anyway.
The gravitational force between a sphere with mass $M$ and an infinitesimal mass segment $\dd m = r d \rho_\mathrm{Cu} \dd r \dd \varphi$ of the shield made of copper with density $\rho_\mathrm{Cu} = 8960\si{kg/m^3}$ at a distance $r$ from the shield's center is given by
\begin{equation}
  \dd \vec{F} = \frac{G M \dd m}{\ell} \boldsymbol{\hat{\ell}} 
  \quad \Rightarrow \quad
  \dd F_z = \frac{G M r \rho_\mathrm{Cu} d}{\ell^2} \dd r \dd \varphi \cos \theta,
\end{equation}
where $\ell^2 = r^2 + L^2$ denotes the distance between the sphere and the mass segment and $\theta = \arccos L/\ell$ is the angle between them.
The total attractive force between the mass and the shield with radius $r_s$ is therefore
\begin{equation}
  F_z = GM \rho_\mathrm{Cu} d L \int\limits_{0}^{r_s} \dd r \int\limits_{0}^{2\pi} \dd \varphi \, \frac{r}{(r^2 + L^2)^{3/2}} = 2\pi G M \rho_\mathrm{Cu} d \left(1 - \frac{L}{\sqrt{L^2 + r_s^2}}\right) .
\end{equation}
For large shields $r_s \rightarrow \infty$ this is independent of the particle-shield separation $L$.
For a shield with thickness $d = 100\si{nm}$ and a silica particle with parameters given in \cref{tab:paramters}, the attraction force is around $F_\mathrm{particle-shield} \approx 4.1\times 10^{-24} \si{N}$ which is comparable with the attraction gravitational attraction force between the two particles themselves at $F_\mathrm{particle-particle} \approx 5.0 \times 10^{-24}\si{N}$ but is much weaker than the Casimir attraction between the particle and the shield with $F_\mathrm{Casimir} \approx 1.4 \times 10^{-17} \si{N}$.
Therefore, the gravitational effect of the shield can be neglected in all practical calculations.

\section{Thermal shield vibrations}\label{sec:4:thermal-vibrations}

\begin{figure}[!htbp]
  \centering
  \includegraphics[width=\textwidth]{./../figures/vibrations/vibrational-modes.pdf}
  \caption{Normalized shape of the vibrational modes $(k,l)$ of a vibrating spherical plate fixed at the edge with $R/d = 1000$.}
  \label{fig:4:vibrational-modes}
\end{figure}



\begin{equation}
  \op{H} = \hbar \omega \left(\op{a}^\dagger\op{a} + \frac{1}{2}\right) + \tilde{g}_A(\op{z}) \ketbra{\psi_A} + \tilde{g}_B(\op{z}) \ketbra{\psi_B}
\end{equation}

Solution:
\begin{equation}
  \rho(t) = \frac{1}{2}\begin{pmatrix}
    1 & e^{-i\varphi(t)}e^{-\gamma(t)}\\
    e^{i\varphi(t)}e^{-\gamma(t)} & 1
  \end{pmatrix}
\end{equation}
with the phase
\begin{equation}
  \varphi(t) = \frac{1}{\hbar^2\omega^2} \left(\sin(\omega t) - \omega t\right) \left(g_A^2 - g_B^2\right)
\end{equation}
and the decoherence term
\begin{equation}
  \gamma(t) = \frac{4(g_A - g_B)^2}{\hbar^2 \omega^2} \sin(\frac{\omega t}{2}) \left[\bar{n} + \frac{1}{2}\right]
\end{equation}

\begin{figure}[!htbp]
  \centering
  \includegraphics[width=\textwidth]{./../figures/vibrations/decoherence-analytical.pdf}
  \caption{Maximum decoherence $\gamma$ at $4\si{K}$ for all modes if the cat-state is placed \textbf{left:} at the point with maximum gradient of the mode $(1,0)$ and \textbf{right:} in the center of the plate. It becomes evident, that only a few low modes play an actual role for the total dephasing.}
  \label{fig:4:}
\end{figure}

\section{Entanglement in front of a thermal shield}\label{sec:5:thermal-entanglement}
The entanglement generation between the two particles depends heavily on the variation of the separation between the shield and the particles, as has been seen in \cref{cha:entanglement-generation}.
The vibrating shield can be interpreted as varying the separation and angle of the cat-state in front of the plate - visualized in \cref{fig:5:vibrating-translation-to-variations}.
\begin{figure}[!htbp]
  \centering
  \def\svgwidth{\textwidth}
  \input{./../figures/plate-vibration.pdf_tex}
  \caption{For a large $r_s \gg R$ and locally flat shield, the thermal vibrations with amplitude $z$ can be interpreted as a static shield where the particle $A$ (shown in the figure) is placed at $L+\Delta L$ at angle $\theta$ and particle $B$ is places at $L-\Delta L$ with angle $-\theta$ where both variations depend on the amplitude. At low vibrational frequencies $1/\omega \approx t_\mathrm{max}$ the amplitude can be assumed to be static during a experimental run and for each measurement thermally distributed around $\mean{z}=0$ with $\Delta z$ given by eq. \eqref{eq:5:amplitude-variance}.}
  \label{fig:5:vibrating-translation-to-variations}
\end{figure}
This is only a good approximation for shields larger than the particles radius $r_s \gg R$ and low vibrating frequencies $1/\omega \approx t_\mathrm{max}$ and can therefore be used well to describe the highly disturbing first few modes on a large shield.
Furthermore, this interpretation is possible because as shown in \cref{sec:3:imperfect-plates}, the Casimir interaction between a sphere and a tilted plane does not differ from the interaction between a flat plane.
Contrary to the problem considered in \cref{cha:entanglement-generation}, here only the thermal amplitude $z_{kl}$ is a independent random variable distributed around $\mean{z_{kl}} = 0$ with standard deviation $\Delta z_{kl}$ given by eq. \eqref{eq:5:amplitude-variance}. 
Both, the variations in the particle-shield separation $\Delta L$ as well as in the angle $\theta$ are correlated to the amplitude $z$.
For a large shield, this can be understood as
\begin{equation}
  \theta = \arctan(z \abs{\nabla u}) \approx z \abs{\nabla u} \quad \text{and} \quad \Delta L = z \abs{u}
\end{equation}
where $\nabla u$ is the gradient of the shape of the vibrational mode.
Performing similar calculations as done before in \cref{cha:entanglement-generation}, the averaged density matrix $\mean{\rho}$ dependent on $\Delta z_{kl}$ can be calculated.
The entanglement quantified by the logarithmic negativity \cite{Plenio_2005} introduced in \cref{sec:2:entanglement-measures} dependent on the temperature $T$ is shown in \cref{fig:5:entanglement-temperature}.
\begin{figure}[!htbp]
  \centering
  \includegraphics[width=\textwidth]{./../figures/vibrations/entanglement-shield-T-L.pdf}
  \caption{Entanglement between the particles (parallel orientation) in front of a thermal shield in the first mode $(1,0)$ at temperature $T$ for different particle-shield separations $L$.}
  \label{fig:5:entanglement-temperature}
\end{figure}
This is not surprising considering that the thermal amplitudes $\Delta z_{1,0} \approx 9 \times 10^{-11}\si{m}$ at $20\si{mK}$ which is comparable with the previously calculated values for $\Delta L_\mathrm{crit}$ in \cref{cha:entanglement-generation}.
Surprisingly this result does not change for different shield radii - at least as long as the condition $r_s \ll R$ is fulfilled and the shield shape can locally be linearized.
This is because the gradient $\abs{\nabla u} \propto 1/r_s$ which perfectly cancels with the dependence on $z \propto r_s$ leaving $\theta$ independent of $r_s$. 
If the cat-state orientation is now chosen parallel to the shield, the dependence on $\Delta L$ is irrelevant leaving the final resulting entanglement independent of $r_s$. 



\subsection{Analytic dynamics}
Surprisingly the influence of the thermal shield on entanglement generation can be calculated analytically.
\begin{align}\label{eq:5:hamiltonian}
\begin{split}
  \op{H} = \sum_{\substack{m\in\{(k,l)\}\\ k \geq 1,\ l \geq 0}} & \hbar \omega_m \left(\op{a}^\dagger_m \op{a}_m + \frac{1}{2}\right) \\
  &+ \left[g^{1,1}_\mathrm{Grav} + \left(g^1_\mathrm{A,m,Cas} +
  g^1_\mathrm{B,m,Cas}\right)(\op{a}_m + \op{a}^\dagger_m)\right] \ketbra{\psi^1_A \psi^1_B} \\
  &+ \left[g^{1,2}_\mathrm{Grav} + \left(g^1_\mathrm{A,m,Cas} + g^2_\mathrm{B,m,Cas}\right)(\op{a}_m + \op{a}^\dagger_m)\right]\ketbra{\psi^1_A \psi^2_B} \\
  &+ \left[g^{2,1}_\mathrm{Grav} + \left(g^2_\mathrm{A,m,Cas} + g^1_\mathrm{B,m,Cas}\right)(\op{a}_m + \op{a}^\dagger_m)\right]\ketbra{\psi^2_A \psi^1_B} \\
  &+ \left[g^{2,2}_\mathrm{Grav} + \left(g^2_\mathrm{A,m,Cas} + g^2_\mathrm{B,m,Cas}\right)(\op{a}_m + \op{a}^\dagger_m)\right]\ketbra{\psi^2_A \psi^2_B}
\end{split}
\end{align}
where $g^{ij}_\mathrm{Grav}$ is the gravitational coupling between the states $\ket{\psi^i_A}$ and $\ket{\psi^j_B}$.
The Casimir interaction between state $\ket{\psi_{A(B)}^i}$ and the shield is denoted by $\tilde{g}^i_\mathrm{A(B),\,m,\,Cas}$. These couplings are dependent on the amplitude $\op{z}_{m} = \sqrt{\hbar/2\tilde{m}\omega_m} (\op{a}_m + \op{a}^\dagger_m)$ and the shape $u_{m}(r_{A(B)})$ of the vibrational mode $m=\{(k,l)\}$ at the position $r_{A(B),i}$ of the cat state:
\begin{equation}
  \tilde{g}^i_\mathrm{A(B),\,m,\,Cas} = \frac{\hbar c \pi^3}{720} \left(\frac{\varepsilon_r - 1}{\varepsilon_r + 1}\right)\varphi(\varepsilon_r) \frac{R}{(\mathscr{L} + \op{z}_m u_m(r))^2} \approx g_\mathrm{PFA} \left(\frac{1}{\mathscr{L}^2} + \frac{2 \op{z}_m u_m(r_{A(B),i})}{\mathscr{L}^3}\right) .
\end{equation}
Ignoring the first term in the expansion, which just produces a global phase in the evolved system, the couplings $g_\mathrm{Cas}$ appearing in eq. \eqref{eq:5:hamiltonian} are finally given by
\begin{equation}
  g^i_\mathrm{A(B),\,m,\,Cas} = g_\mathrm{PFA} \frac{2u_m(r_{A(B),i})}{\mathscr{L}^3} \sqrt{\frac{\hbar}{2 \tilde{m} \omega_m}} .
\end{equation}
It is possible to analytically calculate the time evolution of a system consisting of the initial state $\rho_\mathrm{system}$ (given by eq. \eqref{eq:2:initial-state}) combined with the infinite vibrational modes $\rho_\mathrm{th}$ of the thermal shield
\begin{equation}
  \rho_0 = \bigotimes_{m\in\{(k,l)\}} \left(\rho_\mathrm{th,\,m}\right) \otimes \rho_\mathrm{system} .
\end{equation}
These thermal states can be expanded into coherent states $\ket{\alpha} = \op{D}(\alpha)\ket{0}$ as \cite{Steiner_2024}
\begin{equation}
  \rho_\mathrm{th,m} = \frac{1}{Z} \sum_{n=1}^{\infty} e^{-\beta \hbar \omega_m (n + 1/2)} \ketbra{n} = \int \dd \alpha^2 \frac{1}{\pi \bar{n}} e^{-\frac{\abs{\alpha}^2}{\bar{n}}} \ketbra{\alpha}
\end{equation}
where $\bar{n}$ is the average occupation number. The time evolution of the particle-system $\rho_\mathrm{system}(t) = \tr_{th}\left\{\rho(t)\right\}$ can be calculated by tracing out all states corresponding to the thermal shield.

\begin{figure}[!htbp]
  \centering
  \includegraphics[width=\textwidth]{./../figures/vibrations/entanglement-hamiltonian.pdf}
  \caption{Entanglement dynamics in front of a thermal shield in mode $(1,0)$ at different temperatures. Only at specific times $2\pi k / \omega_{1,0},\ k\in\mathbb{N}$, entanglement is observable. This behavior is expected and aligns with the findings in Ref. \cite{Pedernales_2022}.}
\end{figure}

\begin{figure}[!htbp]
  \centering
  \includegraphics[width=\textwidth]{./../figures/vibrations/entanglement-multiple-modes.pdf}
  \caption{Entanglement dynamics in front of a thermal shield. In orange, the first 50 modes have been used in the numeric calculation. The effect of all remaining modes is around $1.7 \times 10^{-11}\,\%$. In blue, all modes except the first mode $(1,0)$ have been considered at different temperatures ranging vom $1\si{mK}$ up to $20\si{mK}$. The particle-shield separation is fixed at $L = 2R = 20\si{\mu m}$.}
\end{figure}

\begin{figure}[!htbp]
  \centering
  \includegraphics[width=\textwidth]{./../figures/vibrations/all-modes-maximum-entanglement-L-t-max.pdf}
  \caption{All modes, Entanglement at t-max}
\end{figure}


\subsection{Small shields}


\section{Discussion on the optimal setup}\label{sec:4:discussion}

- No global maximum

- dependent on experimental stuff

- Masses and distances and so on must be bounded by 10xECasimir < EGravity

How can one find the best parameters depending on the experimental requirements?
\begin{enumerate}
  \item Specify maximum coherence time (how long the experiment can take)
  \item This fixes $L^3/(M_A M_B \Delta x_A \Delta x_B)$
  \item Look up, what maximum $\Delta \theta$ and $\Delta L$ you can get at this time (Dependent on the amount of entanglement you want to have at the end)
  \item Take the minimum possible $L/R$ in account (dependent on trap stiffness and so on) Ideally, large enough to reduce casimir interactions
\end{enumerate}


-------------------------------------- 

\begin{enumerate}
  \item Given a target measuring time $t_\mathrm{target}$, this fixes the ratio
  \begin{equation}\label{eq:4:fixed-ratio}
    \frac{L^3}{M_A M_B \Delta x_A \Delta x_B} = 5.036\times 10^{22}\si{m/kg^2s} \cdot t_\mathrm{target} = \mathrm{const.}
  \end{equation}
  Most likely, the superposition size $\Delta x_{A,B}$ is also not changeable and one has to be satisfied by what can be achieved experimentally. The delocalized states of mass $M$ has to interact gravitationally with each other in a laboratory setting for at least the duration of the measuring process $t_\mathrm{target}$. These considerations limits the coherence time and usually, low times in the order of milliseconds to seconds are favorable.
  \item In principle, one does not require to measure at the time of maximum entanglement where $E_N = 1$. If a lower quantity of entanglement is enough for a set experimental goal, I recommend, measuring at a lower time than $t_\mathrm{target} = \tau t_\mathrm{max}$ ($\tau < 1$) !!!!FIGURE!!!!. This on the other hand increases the fixed ratio eq. \eqref{eq:4:fixed-ratio} by a factor of $1/\tau$.
  Therefore, it is possible to use smaller superposition sizes $\Delta x$, lighter masses or increase the distance $L$ which decreases Casimir interactions.
  \item Most likely, the minium angular 
\end{enumerate}
\chapter{The consequences of a thermal shield}\label{cha:the-shield}
Up until now, the dynamics and properties of the shield where neglected. At a non-zero temperature however, thermal vibrations of the shield may influence the generation of entanglement substantially.
In this chapter, firstly the required size of the shield is estimated. Then, the thermal vibrations of a large and small shield as well as the effect on entanglement generation is considered.
In the experiment, the particles are cooled down into the ground state for effective quantum control and the generation of spatial superpositions.
For cooling, often liquid helium at $T\approx 4\si{K}$ is used but cryogenic freezers can cool small setups down to the order of $T\approx 20\si{mK}$.
For all relevant calculations, these temperatures are used as a reference point.


\section{Thickness and size of the shield}\label{sec:5:shield-size}
The required thickness $d$ and the radius $r_s$ of the spherical shield to effectively block electromagnetic interactions between the particles can be estimated by analyzing the properties of a physical conductive material with high electric conductivity $\sigma$.
For an almost perfect shielding, a superconducting shield may be considerable as well.
The transmission $T$ of electromagnetic waves through a physical shield is given by \cite{Vandenbosch_2022}
\begin{equation}\label{eq:5:shield-transmission}
  T = \abs{\frac{\vec{E}_\mathrm{after}}{\vec{E}_\mathrm{before}}} = \frac{2}{Z_0 \sigma d}
\end{equation}
where $Z_0 = 377\si{\Omega}$ is the impedance of free space (assuming the shield is placed in a vacuum or in air).
The electric conductivity $\sigma$ is highly dependent on the temperature \cite[p. 284-286]{Gross_2018}, decreasing approximately with $1/T^5$ at low temperatures \footnote{This behavior is valid for temperatures below the Debye temperature ($\Theta_D = 343\si{K}$ for copper). At the low temperatures used in the experiment, this model accurately describes the conductivity of metals \cite{Berman_1952}.}.
Copper offers a strong electric conductivity of $\sigma = 59.6\times 10^6 \si{S/m}$ at room temperature and measured data showing even $\sigma(T = 10\si{K}) \approx 1.5\times 10^{10}\si{S/m}$ at $10\si{K}$ \cite{Berman_1952}.

To estimate the shield's thickness, the primary criterion used, is that gravitational interactions should dominate the entanglement generation.
Other mutual interactions between the particles, such as Coulomb or Casimir forces, must be sufficiently suppressed by the shield.
The \emph{entanglement rate} $\Gamma$ quantifies the build-up of entanglement over time
\begin{equation}\label{eq:5:entanglement-rate}
  \Gamma = \dv{t} E_N(\rho)\Big|_{t=0} \, ,
\end{equation} 
where $E_N$ is an appropriate entanglement measure - in this case the logarithmic negativity \cite{Plenio_2005} introduced in \cref{sec:2:entanglement-measures}.
For gravitational interactions, the entanglement rate in the parallel orientation is given by using eq. \eqref{eq:2:entanglement-dynamics-parallel} as
\begin{equation}\label{eq:5:entanglement-rate-gravity}
  \Gamma_\mathrm{Gravity} = \frac{G M_A M_B \Delta x_A \Delta x_B}{16 \hbar L^3 \log 2} = \frac{G \pi^2 R^6 \rho_\mathrm{Silica}^2 (\Delta x)^2}{9 \hbar L^3 \log 2} .
\end{equation}
where in the last step $M_A = M_B = 4/3 \pi R^3 \rho_\mathrm{Silica}$ and $\Delta x_A = \Delta x_B \equiv \Delta x$ was used.
The entanglement rate for non-gravitational interactions, such as Coulomb or Casimir forces, must be significantly smaller than the gravitational entanglement rate, ideally by a factor $\chi \gg 1$.
This ensures that any measured entanglement is primarily due to gravitational interactions.
In the following sections, estimations about the thickness and size of the shield are made, to effectively screen Coulomb and Casimir forces.


\subsection{Shielding Coulomb-Interactions}
The primary role of the Faraday shield is to block electromagnetic interactions between particles.
Experimentally, it may be beneficial for the particles to carry a small amount of charge, enabling the use of electrostatic traps with high trapping strength and large controllability \cite{GonzalezBallestero_2021}. 
The Coulomb interaction potential between two charged particles is given by
\begin{equation}
  V = \frac{1}{4\pi\varepsilon_0} \frac{q_A q_B}{2L}
\end{equation}
where $\varepsilon_0 = 8.8542\times 10^{-12}\si{A^2 s^4 m^{-3} kg^{-1}}$ is the permittivity of free space and we assume that each particle caries one electron charge $\abs{q_{A(B)}} = e = 1.6022\times 10^{-19}\si{C}$.
This interaction mimics the form of the gravitational potential and can similarly induce entanglement with a entanglement rate
\begin{equation}\label{eq:5:entanglement-rate-coulomb}
  \Gamma_\mathrm{Coulomb} = \frac{T \abs{q_A q_B} (\Delta x)^2}{64 \pi \varepsilon_0 \hbar L^3 \log 2} .
\end{equation}
The shield suppresses the coupling by a factor of $T$.
Requiring $\Gamma_\mathrm{Gravity} > \chi \Gamma_\mathrm{Coulomb}$, the minimum thickness of the shield can be calculated by using eq. \eqref{eq:5:shield-transmission} as
\begin{align}\label{eq:5:coulomb-gravity-condition}
  T \frac{\abs{q_A q_B}}{64 \pi \varepsilon_0} \chi \, &< \, \frac{G \pi^2 R^6 \rho_\mathrm{Silica}^2}{9} \\
  \Longleftrightarrow \quad\quad\quad\quad d \, &> \, \frac{9}{32}\frac{1}{Z_0 \sigma} \frac{1}{\pi^3 \varepsilon_0 G \rho_\mathrm{Silica}^2} \frac{e^2}{R^6} \chi .
\end{align}
The thickness strongly depends on the particles size $R$, and large or heavy particles will favor gravitational entanglement generation.
Assuming the particles are silica nano-spheres with parameters given in \cref{tab:paramters}, a minimum shield-thickness of $d \approx 10\si{nm}\chi$ at $4\si{K}$ and of $d \approx 2.5\si{\mu m}\chi$ at room temperature is required.
At low temperatures, a realistic shield thickness could therefore be $d=100\si{nm}$, balancing engineering practicality and electromagnetic suppression.
Exact estimations however depend on the realization of the experiment as well as the precision in which the evolved state is measurable.

Electrostatic fields still can propagate around the finite-sized Faraday shield and potentially induce entanglement.
It is possible to estimate the required shield radius $r_s$ to block a specific amount $\eta$ of the electric field (see \cref{apx:blocking-of-the-shield}):
\begin{equation}\label{eq:5:shield-effectiveness}
  \frac{r_s}{L} = \sqrt{\frac{1-(1-\eta)^2}{(1-\eta)^2}}
\end{equation}
The results are visualized in \cref{fig:5:shield-radius}.
\begin{figure}[!ht]
  \centering
  \includegraphics[width=\textwidth]{./../figures/others/shield-radius.pdf}
  \caption{Shield radius as a function of the shielding effectiveness $\eta$ for an ideal shield. Additionally, a real shield with varying thicknesses $d$ is considered at $T=300\si{K}$.
  To achieve shielding of $99.5-99.9\%$ ($\eta = 0.995-0.999$), a radius of $r_s =200-1000L$ is needed.}
  \label{fig:5:shield-radius}
\end{figure}
The shield's transmission $T$ should therefore be modified to $\tilde{T} = T\eta + (1-\eta)$, where the shielding effectiveness $\eta$ depends on $r_s$ as given by eq. \eqref{eq:5:shield-effectiveness}.
Modifying eq. \eqref{eq:5:coulomb-gravity-condition}, a minimum effectiveness $\eta_\mathrm{min}$ for sufficient shielding of
\begin{equation}
  \eta_\mathrm{min} \approx 1 - \frac{64\pi^3 \varepsilon_0 G R^6 \rho^2_\mathrm{Silica}}{9 e^2}
\end{equation}
can be approximated.
Using again the parameters from \cref{tab:paramters}, a minimum shielding of $\eta_\mathrm{min} \gtrsim 0.99997$ and thus a radius of $r_s \gtrsim 28000 L \approx 60\si{cm}$ is required.
Depending on the experimental setup, such a shield might be too large for all practical purposes and it may be beneficial to choose heavier masses ($\tilde{M} \sim 4 M$) to reduce the shield size to the orders of $\sim 1\si{cm}$.
Due to practicality, a shield with $r_s = 1\si{cm}$ is used in the following calculations.
Uncharged particles would eliminate the Coulomb interactions and therefore reducing the shield's size and thickness to only shield mutual Casimir interactions.




\subsection{Shielding Casimir-Interactions}
Similarly to Coulomb interactions, it is possible to to estimate the required thickness for a shield to sufficiently block Casimir interactions.
The Casimir potential between the spheres with radius $R$ separated by $2L$ is given in first order by \cite{Emig_2007}
\begin{equation}
  V = -\frac{23 \hbar c}{4\pi \cdot 128 L^7} \left( \frac{\varepsilon_r - 1}{\varepsilon_r + 2} \right)^2 R^6 .
\end{equation}
The corresponding entanglement rate is calculated similar to before by expanding the potential in small $\Delta x$ and computing the logarithmic negativity:
\begin{equation}
  \Gamma_\mathrm{Casimir} = T^2 \frac{161}{4096} \frac{c R^6 (\Delta x)^2}{\pi L^9 \log 2}\left( \frac{\varepsilon_r - 1}{\varepsilon_r + 2}\right)^2 .
\end{equation}
The dependence on $T^2$ arises because Casimir forces are second order effects in the dipole-dipole interaction \cite{Bordag_2001}.
Requiring gravitational entanglement generation to dominate, $\Gamma_\mathrm{Gravity} > \chi \Gamma_\mathrm{Casimir}$, leads to
\begin{align}
  T^2 \frac{161 c R^6}{4096 \pi L^6} \left( \frac{\varepsilon_r - 1}{\varepsilon_r + 2}\right)^2 \chi \, &< \, \frac{G \pi^2 \rho_\mathrm{Silica}^2 R^6}{9\hbar} \\
  \Longleftrightarrow \quad\quad\quad\  d \, &> \, \sqrt{\frac{1449}{4096} \frac{c \hbar}{G \pi^3}} \frac{2}{Z_0 \sigma \rho L^3} \frac{\varepsilon_r - 1}{\varepsilon_r + 2} \sqrt{\chi} ,
\end{align}
where again $T = 2/Z_0 \sigma d$ from eq. \eqref{eq:5:shield-transmission} was used.
For large separations, the shield thickness can be arbitrarily low, as Casimir forces vanish and at separations of $L\gtrsim 100\si{\mu m}$, the shield might not be required at all (compare \cref{sec:2:experimental-problems}).
Assuming two identical silica nano-spheres with parameters given by \cref{tab:paramters}, the required minimum thickness is between $4\times 10^{-11}\si{m} \sqrt{\chi}$ at $4\si{K}$ and $10 \si{nm} \sqrt{\chi}$ at room temperature.
This is much thinner than what is required for shielding Coulomb interactions.
The factor $\varepsilon_r$ modifies the thickness only by up to a constant factor of $\leq 1$ and is therefore ignored for worst-case estimations.

However, very thin shields lose mechanical rigidity, leading to enhanced vibrational excitations and potential instabilities.
Vibrational frequency and thus the vibrational energy depends linearly on the shield's thickness, making thinner shields prone to large decoherence due to thermal vibrations.
A detailed analysis of these effects is provided in the subsequent section.



\subsection{Gravitational effects of the shield}\label{subsec:5:shield-gravitation}
The gravitational interaction between the masses and the shield is generally neglected because it has no significant impact on the entanglement generation between the particles.
The only potential effect is indirect entanglement mediated by the thermal oscillations of the shield, as both masses couple gravitationally to it. 
However, as shown in \cref{sec:5:thermal-entanglement}, this second-order effect is very weak and does not pose a problem, since it still represents gravitationally mediated entanglement - which is the focus of the experiment anyway.
The gravitational force between a sphere with mass $M$ and an infinitesimal mass segment $\dd m = r d \rho_\mathrm{Cu} \dd r \dd \varphi$ of the shield made of copper with density $\rho_\mathrm{Cu} = 8960\si{kg/m^3}$ at a distance $r$ from the shield's center is given by
\begin{equation}
  \dd \vec{F} = \frac{G M \dd m}{\ell} \boldsymbol{\hat{\ell}} 
  \quad \Rightarrow \quad
  \dd F_z = \frac{G M r \rho_\mathrm{Cu} d}{\ell^2} \dd r \dd \varphi \cos \theta,
\end{equation}
where $\ell^2 = r^2 + L^2$ denotes the distance between the sphere and the mass segment and $\theta = \arccos L/\ell$ is the angle between them.
The total attractive force between the mass and the shield with radius $r_s$ is therefore
\begin{equation}
  F_z = GM \rho_\mathrm{Cu} d L \int\limits_{0}^{r_s} \dd r \int\limits_{0}^{2\pi} \dd \varphi \, \frac{r}{(r^2 + L^2)^{3/2}} = 2\pi G M \rho_\mathrm{Cu} d \left(1 - \frac{L}{\sqrt{L^2 + r_s^2}}\right) .
\end{equation}
For large shields $r_s \rightarrow \infty$ this is independent of the particle-shield separation $L$.
For a shield with thickness $d = 100\si{nm}$ and a silica particle with parameters given in \cref{tab:paramters}, the attraction force is around $F_\mathrm{particle-shield} \approx 4.1\times 10^{-24} \si{N}$ which is comparable with the attraction gravitational attraction force between the two particles themselves at $F_\mathrm{particle-particle} \approx 5.0 \times 10^{-24}\si{N}$ but is much weaker than the Casimir attraction between the particle and the shield with $F_\mathrm{Casimir} \approx 1.4 \times 10^{-17} \si{N}$.
Therefore, the gravitational effect of the shield can be neglected in all practical calculations.

\section{Thermal shield vibrations}\label{sec:4:thermal-vibrations}

\begin{figure}[!htbp]
  \centering
  \includegraphics[width=\textwidth]{./../figures/vibrations/vibrational-modes.pdf}
  \caption{Normalized shape of the vibrational modes $(k,l)$ of a vibrating spherical plate fixed at the edge with $R/d = 1000$.}
  \label{fig:4:vibrational-modes}
\end{figure}



\begin{equation}
  \op{H} = \hbar \omega \left(\op{a}^\dagger\op{a} + \frac{1}{2}\right) + \tilde{g}_A(\op{z}) \ketbra{\psi_A} + \tilde{g}_B(\op{z}) \ketbra{\psi_B}
\end{equation}

Solution:
\begin{equation}
  \rho(t) = \frac{1}{2}\begin{pmatrix}
    1 & e^{-i\varphi(t)}e^{-\gamma(t)}\\
    e^{i\varphi(t)}e^{-\gamma(t)} & 1
  \end{pmatrix}
\end{equation}
with the phase
\begin{equation}
  \varphi(t) = \frac{1}{\hbar^2\omega^2} \left(\sin(\omega t) - \omega t\right) \left(g_A^2 - g_B^2\right)
\end{equation}
and the decoherence term
\begin{equation}
  \gamma(t) = \frac{4(g_A - g_B)^2}{\hbar^2 \omega^2} \sin(\frac{\omega t}{2}) \left[\bar{n} + \frac{1}{2}\right]
\end{equation}

\begin{figure}[!htbp]
  \centering
  \includegraphics[width=\textwidth]{./../figures/vibrations/decoherence-analytical.pdf}
  \caption{Maximum decoherence $\gamma$ at $4\si{K}$ for all modes if the cat-state is placed \textbf{left:} at the point with maximum gradient of the mode $(1,0)$ and \textbf{right:} in the center of the plate. It becomes evident, that only a few low modes play an actual role for the total dephasing.}
  \label{fig:4:}
\end{figure}

\section{Entanglement in front of a thermal shield}\label{sec:5:thermal-entanglement}
The entanglement generation between the two particles depends heavily on the variation of the separation between the shield and the particles, as has been seen in \cref{cha:entanglement-generation}.
The vibrating shield can be interpreted as varying the separation and angle of the cat-state in front of the plate - visualized in \cref{fig:5:vibrating-translation-to-variations}.
\begin{figure}[!htbp]
  \centering
  \def\svgwidth{\textwidth}
  \input{./../figures/plate-vibration.pdf_tex}
  \caption{For a large $r_s \gg R$ and locally flat shield, the thermal vibrations with amplitude $z$ can be interpreted as a static shield where the particle $A$ (shown in the figure) is placed at $L+\Delta L$ at angle $\theta$ and particle $B$ is places at $L-\Delta L$ with angle $-\theta$ where both variations depend on the amplitude. At low vibrational frequencies $1/\omega \approx t_\mathrm{max}$ the amplitude can be assumed to be static during a experimental run and for each measurement thermally distributed around $\mean{z}=0$ with $\Delta z$ given by eq. \eqref{eq:5:amplitude-variance}.}
  \label{fig:5:vibrating-translation-to-variations}
\end{figure}
This is only a good approximation for shields larger than the particles radius $r_s \gg R$ and low vibrating frequencies $1/\omega \approx t_\mathrm{max}$ and can therefore be used well to describe the highly disturbing first few modes on a large shield.
Furthermore, this interpretation is possible because as shown in \cref{sec:3:imperfect-plates}, the Casimir interaction between a sphere and a tilted plane does not differ from the interaction between a flat plane.
Contrary to the problem considered in \cref{cha:entanglement-generation}, here only the thermal amplitude $z_{kl}$ is a independent random variable distributed around $\mean{z_{kl}} = 0$ with standard deviation $\Delta z_{kl}$ given by eq. \eqref{eq:5:amplitude-variance}. 
Both, the variations in the particle-shield separation $\Delta L$ as well as in the angle $\theta$ are correlated to the amplitude $z$.
For a large shield, this can be understood as
\begin{equation}
  \theta = \arctan(z \abs{\nabla u}) \approx z \abs{\nabla u} \quad \text{and} \quad \Delta L = z \abs{u}
\end{equation}
where $\nabla u$ is the gradient of the shape of the vibrational mode.
Performing similar calculations as done before in \cref{cha:entanglement-generation}, the averaged density matrix $\mean{\rho}$ dependent on $\Delta z_{kl}$ can be calculated.
The entanglement quantified by the logarithmic negativity \cite{Plenio_2005} introduced in \cref{sec:2:entanglement-measures} dependent on the temperature $T$ is shown in \cref{fig:5:entanglement-temperature}.
\begin{figure}[!htbp]
  \centering
  \includegraphics[width=\textwidth]{./../figures/vibrations/entanglement-shield-T-L.pdf}
  \caption{Entanglement between the particles (parallel orientation) in front of a thermal shield in the first mode $(1,0)$ at temperature $T$ for different particle-shield separations $L$.}
  \label{fig:5:entanglement-temperature}
\end{figure}
This is not surprising considering that the thermal amplitudes $\Delta z_{1,0} \approx 9 \times 10^{-11}\si{m}$ at $20\si{mK}$ which is comparable with the previously calculated values for $\Delta L_\mathrm{crit}$ in \cref{cha:entanglement-generation}.
Surprisingly this result does not change for different shield radii - at least as long as the condition $r_s \ll R$ is fulfilled and the shield shape can locally be linearized.
This is because the gradient $\abs{\nabla u} \propto 1/r_s$ which perfectly cancels with the dependence on $z \propto r_s$ leaving $\theta$ independent of $r_s$. 
If the cat-state orientation is now chosen parallel to the shield, the dependence on $\Delta L$ is irrelevant leaving the final resulting entanglement independent of $r_s$. 



\subsection{Analytic dynamics}
Surprisingly the influence of the thermal shield on entanglement generation can be calculated analytically.
\begin{align}\label{eq:5:hamiltonian}
\begin{split}
  \op{H} = \sum_{\substack{m\in\{(k,l)\}\\ k \geq 1,\ l \geq 0}} & \hbar \omega_m \left(\op{a}^\dagger_m \op{a}_m + \frac{1}{2}\right) \\
  &+ \left[g^{1,1}_\mathrm{Grav} + \left(g^1_\mathrm{A,m,Cas} +
  g^1_\mathrm{B,m,Cas}\right)(\op{a}_m + \op{a}^\dagger_m)\right] \ketbra{\psi^1_A \psi^1_B} \\
  &+ \left[g^{1,2}_\mathrm{Grav} + \left(g^1_\mathrm{A,m,Cas} + g^2_\mathrm{B,m,Cas}\right)(\op{a}_m + \op{a}^\dagger_m)\right]\ketbra{\psi^1_A \psi^2_B} \\
  &+ \left[g^{2,1}_\mathrm{Grav} + \left(g^2_\mathrm{A,m,Cas} + g^1_\mathrm{B,m,Cas}\right)(\op{a}_m + \op{a}^\dagger_m)\right]\ketbra{\psi^2_A \psi^1_B} \\
  &+ \left[g^{2,2}_\mathrm{Grav} + \left(g^2_\mathrm{A,m,Cas} + g^2_\mathrm{B,m,Cas}\right)(\op{a}_m + \op{a}^\dagger_m)\right]\ketbra{\psi^2_A \psi^2_B}
\end{split}
\end{align}
where $g^{ij}_\mathrm{Grav}$ is the gravitational coupling between the states $\ket{\psi^i_A}$ and $\ket{\psi^j_B}$.
The Casimir interaction between state $\ket{\psi_{A(B)}^i}$ and the shield is denoted by $\tilde{g}^i_\mathrm{A(B),\,m,\,Cas}$. These couplings are dependent on the amplitude $\op{z}_{m} = \sqrt{\hbar/2\tilde{m}\omega_m} (\op{a}_m + \op{a}^\dagger_m)$ and the shape $u_{m}(r_{A(B)})$ of the vibrational mode $m=\{(k,l)\}$ at the position $r_{A(B),i}$ of the cat state:
\begin{equation}
  \tilde{g}^i_\mathrm{A(B),\,m,\,Cas} = \frac{\hbar c \pi^3}{720} \left(\frac{\varepsilon_r - 1}{\varepsilon_r + 1}\right)\varphi(\varepsilon_r) \frac{R}{(\mathscr{L} + \op{z}_m u_m(r))^2} \approx g_\mathrm{PFA} \left(\frac{1}{\mathscr{L}^2} + \frac{2 \op{z}_m u_m(r_{A(B),i})}{\mathscr{L}^3}\right) .
\end{equation}
Ignoring the first term in the expansion, which just produces a global phase in the evolved system, the couplings $g_\mathrm{Cas}$ appearing in eq. \eqref{eq:5:hamiltonian} are finally given by
\begin{equation}
  g^i_\mathrm{A(B),\,m,\,Cas} = g_\mathrm{PFA} \frac{2u_m(r_{A(B),i})}{\mathscr{L}^3} \sqrt{\frac{\hbar}{2 \tilde{m} \omega_m}} .
\end{equation}
It is possible to analytically calculate the time evolution of a system consisting of the initial state $\rho_\mathrm{system}$ (given by eq. \eqref{eq:2:initial-state}) combined with the infinite vibrational modes $\rho_\mathrm{th}$ of the thermal shield
\begin{equation}
  \rho_0 = \bigotimes_{m\in\{(k,l)\}} \left(\rho_\mathrm{th,\,m}\right) \otimes \rho_\mathrm{system} .
\end{equation}
These thermal states can be expanded into coherent states $\ket{\alpha} = \op{D}(\alpha)\ket{0}$ as \cite{Steiner_2024}
\begin{equation}
  \rho_\mathrm{th,m} = \frac{1}{Z} \sum_{n=1}^{\infty} e^{-\beta \hbar \omega_m (n + 1/2)} \ketbra{n} = \int \dd \alpha^2 \frac{1}{\pi \bar{n}} e^{-\frac{\abs{\alpha}^2}{\bar{n}}} \ketbra{\alpha}
\end{equation}
where $\bar{n}$ is the average occupation number. The time evolution of the particle-system $\rho_\mathrm{system}(t) = \tr_{th}\left\{\rho(t)\right\}$ can be calculated by tracing out all states corresponding to the thermal shield.

\begin{figure}[!htbp]
  \centering
  \includegraphics[width=\textwidth]{./../figures/vibrations/entanglement-hamiltonian.pdf}
  \caption{Entanglement dynamics in front of a thermal shield in mode $(1,0)$ at different temperatures. Only at specific times $2\pi k / \omega_{1,0},\ k\in\mathbb{N}$, entanglement is observable. This behavior is expected and aligns with the findings in Ref. \cite{Pedernales_2022}.}
\end{figure}

\begin{figure}[!htbp]
  \centering
  \includegraphics[width=\textwidth]{./../figures/vibrations/entanglement-multiple-modes.pdf}
  \caption{Entanglement dynamics in front of a thermal shield. In orange, the first 50 modes have been used in the numeric calculation. The effect of all remaining modes is around $1.7 \times 10^{-11}\,\%$. In blue, all modes except the first mode $(1,0)$ have been considered at different temperatures ranging vom $1\si{mK}$ up to $20\si{mK}$. The particle-shield separation is fixed at $L = 2R = 20\si{\mu m}$.}
\end{figure}

\begin{figure}[!htbp]
  \centering
  \includegraphics[width=\textwidth]{./../figures/vibrations/all-modes-maximum-entanglement-L-t-max.pdf}
  \caption{All modes, Entanglement at t-max}
\end{figure}


\subsection{Small shields}


\section{Discussion on the optimal setup}\label{sec:4:discussion}

- No global maximum

- dependent on experimental stuff

- Masses and distances and so on must be bounded by 10xECasimir < EGravity

How can one find the best parameters depending on the experimental requirements?
\begin{enumerate}
  \item Specify maximum coherence time (how long the experiment can take)
  \item This fixes $L^3/(M_A M_B \Delta x_A \Delta x_B)$
  \item Look up, what maximum $\Delta \theta$ and $\Delta L$ you can get at this time (Dependent on the amount of entanglement you want to have at the end)
  \item Take the minimum possible $L/R$ in account (dependent on trap stiffness and so on) Ideally, large enough to reduce casimir interactions
\end{enumerate}


-------------------------------------- 

\begin{enumerate}
  \item Given a target measuring time $t_\mathrm{target}$, this fixes the ratio
  \begin{equation}\label{eq:4:fixed-ratio}
    \frac{L^3}{M_A M_B \Delta x_A \Delta x_B} = 5.036\times 10^{22}\si{m/kg^2s} \cdot t_\mathrm{target} = \mathrm{const.}
  \end{equation}
  Most likely, the superposition size $\Delta x_{A,B}$ is also not changeable and one has to be satisfied by what can be achieved experimentally. The delocalized states of mass $M$ has to interact gravitationally with each other in a laboratory setting for at least the duration of the measuring process $t_\mathrm{target}$. These considerations limits the coherence time and usually, low times in the order of milliseconds to seconds are favorable.
  \item In principle, one does not require to measure at the time of maximum entanglement where $E_N = 1$. If a lower quantity of entanglement is enough for a set experimental goal, I recommend, measuring at a lower time than $t_\mathrm{target} = \tau t_\mathrm{max}$ ($\tau < 1$) !!!!FIGURE!!!!. This on the other hand increases the fixed ratio eq. \eqref{eq:4:fixed-ratio} by a factor of $1/\tau$.
  Therefore, it is possible to use smaller superposition sizes $\Delta x$, lighter masses or increase the distance $L$ which decreases Casimir interactions.
  \item Most likely, the minium angular 
\end{enumerate}
\chapter{The consequences of a thermal shield}\label{cha:the-shield}
Up until now, the dynamics and properties of the shield where neglected. At a non-zero temperature however, thermal vibrations of the shield may influence the generation of entanglement substantially.
In this chapter, firstly the required size of the shield is estimated. Then, the thermal vibrations of a large and small shield as well as the effect on entanglement generation is considered.
In the experiment, the particles are cooled down into the ground state for effective quantum control and the generation of spatial superpositions.
For cooling, often liquid helium at $T\approx 4\si{K}$ is used but cryogenic freezers can cool small setups down to the order of $T\approx 20\si{mK}$.
For all relevant calculations, these temperatures are used as a reference point.


\section{Thickness and size of the shield}\label{sec:5:shield-size}
The required thickness $d$ and the radius $r_s$ of the spherical shield to effectively block electromagnetic interactions between the particles can be estimated by analyzing the properties of a physical conductive material with high electric conductivity $\sigma$.
For an almost perfect shielding, a superconducting shield may be considerable as well.
The transmission $T$ of electromagnetic waves through a physical shield is given by \cite{Vandenbosch_2022}
\begin{equation}\label{eq:5:shield-transmission}
  T = \abs{\frac{\vec{E}_\mathrm{after}}{\vec{E}_\mathrm{before}}} = \frac{2}{Z_0 \sigma d}
\end{equation}
where $Z_0 = 377\si{\Omega}$ is the impedance of free space (assuming the shield is placed in a vacuum or in air).
The electric conductivity $\sigma$ is highly dependent on the temperature \cite[p. 284-286]{Gross_2018}, decreasing approximately with $1/T^5$ at low temperatures \footnote{This behavior is valid for temperatures below the Debye temperature ($\Theta_D = 343\si{K}$ for copper). At the low temperatures used in the experiment, this model accurately describes the conductivity of metals \cite{Berman_1952}.}.
Copper offers a strong electric conductivity of $\sigma = 59.6\times 10^6 \si{S/m}$ at room temperature and measured data showing even $\sigma(T = 10\si{K}) \approx 1.5\times 10^{10}\si{S/m}$ at $10\si{K}$ \cite{Berman_1952}.

To estimate the shield's thickness, the primary criterion used, is that gravitational interactions should dominate the entanglement generation.
Other mutual interactions between the particles, such as Coulomb or Casimir forces, must be sufficiently suppressed by the shield.
The \emph{entanglement rate} $\Gamma$ quantifies the build-up of entanglement over time
\begin{equation}\label{eq:5:entanglement-rate}
  \Gamma = \dv{t} E_N(\rho)\Big|_{t=0} \, ,
\end{equation} 
where $E_N$ is an appropriate entanglement measure - in this case the logarithmic negativity \cite{Plenio_2005} introduced in \cref{sec:2:entanglement-measures}.
For gravitational interactions, the entanglement rate in the parallel orientation is given by using eq. \eqref{eq:2:entanglement-dynamics-parallel} as
\begin{equation}\label{eq:5:entanglement-rate-gravity}
  \Gamma_\mathrm{Gravity} = \frac{G M_A M_B \Delta x_A \Delta x_B}{16 \hbar L^3 \log 2} = \frac{G \pi^2 R^6 \rho_\mathrm{Silica}^2 (\Delta x)^2}{9 \hbar L^3 \log 2} .
\end{equation}
where in the last step $M_A = M_B = 4/3 \pi R^3 \rho_\mathrm{Silica}$ and $\Delta x_A = \Delta x_B \equiv \Delta x$ was used.
The entanglement rate for non-gravitational interactions, such as Coulomb or Casimir forces, must be significantly smaller than the gravitational entanglement rate, ideally by a factor $\chi \gg 1$.
This ensures that any measured entanglement is primarily due to gravitational interactions.
In the following sections, estimations about the thickness and size of the shield are made, to effectively screen Coulomb and Casimir forces.


\subsection{Shielding Coulomb-Interactions}
The primary role of the Faraday shield is to block electromagnetic interactions between particles.
Experimentally, it may be beneficial for the particles to carry a small amount of charge, enabling the use of electrostatic traps with high trapping strength and large controllability \cite{GonzalezBallestero_2021}. 
The Coulomb interaction potential between two charged particles is given by
\begin{equation}
  V = \frac{1}{4\pi\varepsilon_0} \frac{q_A q_B}{2L}
\end{equation}
where $\varepsilon_0 = 8.8542\times 10^{-12}\si{A^2 s^4 m^{-3} kg^{-1}}$ is the permittivity of free space and we assume that each particle caries one electron charge $\abs{q_{A(B)}} = e = 1.6022\times 10^{-19}\si{C}$.
This interaction mimics the form of the gravitational potential and can similarly induce entanglement with a entanglement rate
\begin{equation}\label{eq:5:entanglement-rate-coulomb}
  \Gamma_\mathrm{Coulomb} = \frac{T \abs{q_A q_B} (\Delta x)^2}{64 \pi \varepsilon_0 \hbar L^3 \log 2} .
\end{equation}
The shield suppresses the coupling by a factor of $T$.
Requiring $\Gamma_\mathrm{Gravity} > \chi \Gamma_\mathrm{Coulomb}$, the minimum thickness of the shield can be calculated by using eq. \eqref{eq:5:shield-transmission} as
\begin{align}\label{eq:5:coulomb-gravity-condition}
  T \frac{\abs{q_A q_B}}{64 \pi \varepsilon_0} \chi \, &< \, \frac{G \pi^2 R^6 \rho_\mathrm{Silica}^2}{9} \\
  \Longleftrightarrow \quad\quad\quad\quad d \, &> \, \frac{9}{32}\frac{1}{Z_0 \sigma} \frac{1}{\pi^3 \varepsilon_0 G \rho_\mathrm{Silica}^2} \frac{e^2}{R^6} \chi .
\end{align}
The thickness strongly depends on the particles size $R$, and large or heavy particles will favor gravitational entanglement generation.
Assuming the particles are silica nano-spheres with parameters given in \cref{tab:paramters}, a minimum shield-thickness of $d \approx 10\si{nm}\chi$ at $4\si{K}$ and of $d \approx 2.5\si{\mu m}\chi$ at room temperature is required.
At low temperatures, a realistic shield thickness could therefore be $d=100\si{nm}$, balancing engineering practicality and electromagnetic suppression.
Exact estimations however depend on the realization of the experiment as well as the precision in which the evolved state is measurable.

Electrostatic fields still can propagate around the finite-sized Faraday shield and potentially induce entanglement.
It is possible to estimate the required shield radius $r_s$ to block a specific amount $\eta$ of the electric field (see \cref{apx:blocking-of-the-shield}):
\begin{equation}\label{eq:5:shield-effectiveness}
  \frac{r_s}{L} = \sqrt{\frac{1-(1-\eta)^2}{(1-\eta)^2}}
\end{equation}
The results are visualized in \cref{fig:5:shield-radius}.
\begin{figure}[!ht]
  \centering
  \includegraphics[width=\textwidth]{./../figures/others/shield-radius.pdf}
  \caption{Shield radius as a function of the shielding effectiveness $\eta$ for an ideal shield. Additionally, a real shield with varying thicknesses $d$ is considered at $T=300\si{K}$.
  To achieve shielding of $99.5-99.9\%$ ($\eta = 0.995-0.999$), a radius of $r_s =200-1000L$ is needed.}
  \label{fig:5:shield-radius}
\end{figure}
The shield's transmission $T$ should therefore be modified to $\tilde{T} = T\eta + (1-\eta)$, where the shielding effectiveness $\eta$ depends on $r_s$ as given by eq. \eqref{eq:5:shield-effectiveness}.
Modifying eq. \eqref{eq:5:coulomb-gravity-condition}, a minimum effectiveness $\eta_\mathrm{min}$ for sufficient shielding of
\begin{equation}
  \eta_\mathrm{min} \approx 1 - \frac{64\pi^3 \varepsilon_0 G R^6 \rho^2_\mathrm{Silica}}{9 e^2}
\end{equation}
can be approximated.
Using again the parameters from \cref{tab:paramters}, a minimum shielding of $\eta_\mathrm{min} \gtrsim 0.99997$ and thus a radius of $r_s \gtrsim 28000 L \approx 60\si{cm}$ is required.
Depending on the experimental setup, such a shield might be too large for all practical purposes and it may be beneficial to choose heavier masses ($\tilde{M} \sim 4 M$) to reduce the shield size to the orders of $\sim 1\si{cm}$.
Due to practicality, a shield with $r_s = 1\si{cm}$ is used in the following calculations.
Uncharged particles would eliminate the Coulomb interactions and therefore reducing the shield's size and thickness to only shield mutual Casimir interactions.




\subsection{Shielding Casimir-Interactions}
Similarly to Coulomb interactions, it is possible to to estimate the required thickness for a shield to sufficiently block Casimir interactions.
The Casimir potential between the spheres with radius $R$ separated by $2L$ is given in first order by \cite{Emig_2007}
\begin{equation}
  V = -\frac{23 \hbar c}{4\pi \cdot 128 L^7} \left( \frac{\varepsilon_r - 1}{\varepsilon_r + 2} \right)^2 R^6 .
\end{equation}
The corresponding entanglement rate is calculated similar to before by expanding the potential in small $\Delta x$ and computing the logarithmic negativity:
\begin{equation}
  \Gamma_\mathrm{Casimir} = T^2 \frac{161}{4096} \frac{c R^6 (\Delta x)^2}{\pi L^9 \log 2}\left( \frac{\varepsilon_r - 1}{\varepsilon_r + 2}\right)^2 .
\end{equation}
The dependence on $T^2$ arises because Casimir forces are second order effects in the dipole-dipole interaction \cite{Bordag_2001}.
Requiring gravitational entanglement generation to dominate, $\Gamma_\mathrm{Gravity} > \chi \Gamma_\mathrm{Casimir}$, leads to
\begin{align}
  T^2 \frac{161 c R^6}{4096 \pi L^6} \left( \frac{\varepsilon_r - 1}{\varepsilon_r + 2}\right)^2 \chi \, &< \, \frac{G \pi^2 \rho_\mathrm{Silica}^2 R^6}{9\hbar} \\
  \Longleftrightarrow \quad\quad\quad\  d \, &> \, \sqrt{\frac{1449}{4096} \frac{c \hbar}{G \pi^3}} \frac{2}{Z_0 \sigma \rho L^3} \frac{\varepsilon_r - 1}{\varepsilon_r + 2} \sqrt{\chi} ,
\end{align}
where again $T = 2/Z_0 \sigma d$ from eq. \eqref{eq:5:shield-transmission} was used.
For large separations, the shield thickness can be arbitrarily low, as Casimir forces vanish and at separations of $L\gtrsim 100\si{\mu m}$, the shield might not be required at all (compare \cref{sec:2:experimental-problems}).
Assuming two identical silica nano-spheres with parameters given by \cref{tab:paramters}, the required minimum thickness is between $4\times 10^{-11}\si{m} \sqrt{\chi}$ at $4\si{K}$ and $10 \si{nm} \sqrt{\chi}$ at room temperature.
This is much thinner than what is required for shielding Coulomb interactions.
The factor $\varepsilon_r$ modifies the thickness only by up to a constant factor of $\leq 1$ and is therefore ignored for worst-case estimations.

However, very thin shields lose mechanical rigidity, leading to enhanced vibrational excitations and potential instabilities.
Vibrational frequency and thus the vibrational energy depends linearly on the shield's thickness, making thinner shields prone to large decoherence due to thermal vibrations.
A detailed analysis of these effects is provided in the subsequent section.



\subsection{Gravitational effects of the shield}\label{subsec:5:shield-gravitation}
The gravitational interaction between the masses and the shield is generally neglected because it has no significant impact on the entanglement generation between the particles.
The only potential effect is indirect entanglement mediated by the thermal oscillations of the shield, as both masses couple gravitationally to it. 
However, as shown in \cref{sec:5:thermal-entanglement}, this second-order effect is very weak and does not pose a problem, since it still represents gravitationally mediated entanglement - which is the focus of the experiment anyway.
The gravitational force between a sphere with mass $M$ and an infinitesimal mass segment $\dd m = r d \rho_\mathrm{Cu} \dd r \dd \varphi$ of the shield made of copper with density $\rho_\mathrm{Cu} = 8960\si{kg/m^3}$ at a distance $r$ from the shield's center is given by
\begin{equation}
  \dd \vec{F} = \frac{G M \dd m}{\ell} \boldsymbol{\hat{\ell}} 
  \quad \Rightarrow \quad
  \dd F_z = \frac{G M r \rho_\mathrm{Cu} d}{\ell^2} \dd r \dd \varphi \cos \theta,
\end{equation}
where $\ell^2 = r^2 + L^2$ denotes the distance between the sphere and the mass segment and $\theta = \arccos L/\ell$ is the angle between them.
The total attractive force between the mass and the shield with radius $r_s$ is therefore
\begin{equation}
  F_z = GM \rho_\mathrm{Cu} d L \int\limits_{0}^{r_s} \dd r \int\limits_{0}^{2\pi} \dd \varphi \, \frac{r}{(r^2 + L^2)^{3/2}} = 2\pi G M \rho_\mathrm{Cu} d \left(1 - \frac{L}{\sqrt{L^2 + r_s^2}}\right) .
\end{equation}
For large shields $r_s \rightarrow \infty$ this is independent of the particle-shield separation $L$.
For a shield with thickness $d = 100\si{nm}$ and a silica particle with parameters given in \cref{tab:paramters}, the attraction force is around $F_\mathrm{particle-shield} \approx 4.1\times 10^{-24} \si{N}$ which is comparable with the attraction gravitational attraction force between the two particles themselves at $F_\mathrm{particle-particle} \approx 5.0 \times 10^{-24}\si{N}$ but is much weaker than the Casimir attraction between the particle and the shield with $F_\mathrm{Casimir} \approx 1.4 \times 10^{-17} \si{N}$.
Therefore, the gravitational effect of the shield can be neglected in all practical calculations.

\section{Thermal shield vibrations}\label{sec:4:thermal-vibrations}

\begin{figure}[!htbp]
  \centering
  \includegraphics[width=\textwidth]{./../figures/vibrations/vibrational-modes.pdf}
  \caption{Normalized shape of the vibrational modes $(k,l)$ of a vibrating spherical plate fixed at the edge with $R/d = 1000$.}
  \label{fig:4:vibrational-modes}
\end{figure}



\begin{equation}
  \op{H} = \hbar \omega \left(\op{a}^\dagger\op{a} + \frac{1}{2}\right) + \tilde{g}_A(\op{z}) \ketbra{\psi_A} + \tilde{g}_B(\op{z}) \ketbra{\psi_B}
\end{equation}

Solution:
\begin{equation}
  \rho(t) = \frac{1}{2}\begin{pmatrix}
    1 & e^{-i\varphi(t)}e^{-\gamma(t)}\\
    e^{i\varphi(t)}e^{-\gamma(t)} & 1
  \end{pmatrix}
\end{equation}
with the phase
\begin{equation}
  \varphi(t) = \frac{1}{\hbar^2\omega^2} \left(\sin(\omega t) - \omega t\right) \left(g_A^2 - g_B^2\right)
\end{equation}
and the decoherence term
\begin{equation}
  \gamma(t) = \frac{4(g_A - g_B)^2}{\hbar^2 \omega^2} \sin(\frac{\omega t}{2}) \left[\bar{n} + \frac{1}{2}\right]
\end{equation}

\begin{figure}[!htbp]
  \centering
  \includegraphics[width=\textwidth]{./../figures/vibrations/decoherence-analytical.pdf}
  \caption{Maximum decoherence $\gamma$ at $4\si{K}$ for all modes if the cat-state is placed \textbf{left:} at the point with maximum gradient of the mode $(1,0)$ and \textbf{right:} in the center of the plate. It becomes evident, that only a few low modes play an actual role for the total dephasing.}
  \label{fig:4:}
\end{figure}

\section{Entanglement in front of a thermal shield}\label{sec:5:thermal-entanglement}
The entanglement generation between the two particles depends heavily on the variation of the separation between the shield and the particles, as has been seen in \cref{cha:entanglement-generation}.
The vibrating shield can be interpreted as varying the separation and angle of the cat-state in front of the plate - visualized in \cref{fig:5:vibrating-translation-to-variations}.
\begin{figure}[!htbp]
  \centering
  \def\svgwidth{\textwidth}
  \input{./../figures/plate-vibration.pdf_tex}
  \caption{For a large $r_s \gg R$ and locally flat shield, the thermal vibrations with amplitude $z$ can be interpreted as a static shield where the particle $A$ (shown in the figure) is placed at $L+\Delta L$ at angle $\theta$ and particle $B$ is places at $L-\Delta L$ with angle $-\theta$ where both variations depend on the amplitude. At low vibrational frequencies $1/\omega \approx t_\mathrm{max}$ the amplitude can be assumed to be static during a experimental run and for each measurement thermally distributed around $\mean{z}=0$ with $\Delta z$ given by eq. \eqref{eq:5:amplitude-variance}.}
  \label{fig:5:vibrating-translation-to-variations}
\end{figure}
This is only a good approximation for shields larger than the particles radius $r_s \gg R$ and low vibrating frequencies $1/\omega \approx t_\mathrm{max}$ and can therefore be used well to describe the highly disturbing first few modes on a large shield.
Furthermore, this interpretation is possible because as shown in \cref{sec:3:imperfect-plates}, the Casimir interaction between a sphere and a tilted plane does not differ from the interaction between a flat plane.
Contrary to the problem considered in \cref{cha:entanglement-generation}, here only the thermal amplitude $z_{kl}$ is a independent random variable distributed around $\mean{z_{kl}} = 0$ with standard deviation $\Delta z_{kl}$ given by eq. \eqref{eq:5:amplitude-variance}. 
Both, the variations in the particle-shield separation $\Delta L$ as well as in the angle $\theta$ are correlated to the amplitude $z$.
For a large shield, this can be understood as
\begin{equation}
  \theta = \arctan(z \abs{\nabla u}) \approx z \abs{\nabla u} \quad \text{and} \quad \Delta L = z \abs{u}
\end{equation}
where $\nabla u$ is the gradient of the shape of the vibrational mode.
Performing similar calculations as done before in \cref{cha:entanglement-generation}, the averaged density matrix $\mean{\rho}$ dependent on $\Delta z_{kl}$ can be calculated.
The entanglement quantified by the logarithmic negativity \cite{Plenio_2005} introduced in \cref{sec:2:entanglement-measures} dependent on the temperature $T$ is shown in \cref{fig:5:entanglement-temperature}.
\begin{figure}[!htbp]
  \centering
  \includegraphics[width=\textwidth]{./../figures/vibrations/entanglement-shield-T-L.pdf}
  \caption{Entanglement between the particles (parallel orientation) in front of a thermal shield in the first mode $(1,0)$ at temperature $T$ for different particle-shield separations $L$.}
  \label{fig:5:entanglement-temperature}
\end{figure}
This is not surprising considering that the thermal amplitudes $\Delta z_{1,0} \approx 9 \times 10^{-11}\si{m}$ at $20\si{mK}$ which is comparable with the previously calculated values for $\Delta L_\mathrm{crit}$ in \cref{cha:entanglement-generation}.
Surprisingly this result does not change for different shield radii - at least as long as the condition $r_s \ll R$ is fulfilled and the shield shape can locally be linearized.
This is because the gradient $\abs{\nabla u} \propto 1/r_s$ which perfectly cancels with the dependence on $z \propto r_s$ leaving $\theta$ independent of $r_s$. 
If the cat-state orientation is now chosen parallel to the shield, the dependence on $\Delta L$ is irrelevant leaving the final resulting entanglement independent of $r_s$. 



\subsection{Analytic dynamics}
Surprisingly the influence of the thermal shield on entanglement generation can be calculated analytically.
\begin{align}\label{eq:5:hamiltonian}
\begin{split}
  \op{H} = \sum_{\substack{m\in\{(k,l)\}\\ k \geq 1,\ l \geq 0}} & \hbar \omega_m \left(\op{a}^\dagger_m \op{a}_m + \frac{1}{2}\right) \\
  &+ \left[g^{1,1}_\mathrm{Grav} + \left(g^1_\mathrm{A,m,Cas} +
  g^1_\mathrm{B,m,Cas}\right)(\op{a}_m + \op{a}^\dagger_m)\right] \ketbra{\psi^1_A \psi^1_B} \\
  &+ \left[g^{1,2}_\mathrm{Grav} + \left(g^1_\mathrm{A,m,Cas} + g^2_\mathrm{B,m,Cas}\right)(\op{a}_m + \op{a}^\dagger_m)\right]\ketbra{\psi^1_A \psi^2_B} \\
  &+ \left[g^{2,1}_\mathrm{Grav} + \left(g^2_\mathrm{A,m,Cas} + g^1_\mathrm{B,m,Cas}\right)(\op{a}_m + \op{a}^\dagger_m)\right]\ketbra{\psi^2_A \psi^1_B} \\
  &+ \left[g^{2,2}_\mathrm{Grav} + \left(g^2_\mathrm{A,m,Cas} + g^2_\mathrm{B,m,Cas}\right)(\op{a}_m + \op{a}^\dagger_m)\right]\ketbra{\psi^2_A \psi^2_B}
\end{split}
\end{align}
where $g^{ij}_\mathrm{Grav}$ is the gravitational coupling between the states $\ket{\psi^i_A}$ and $\ket{\psi^j_B}$.
The Casimir interaction between state $\ket{\psi_{A(B)}^i}$ and the shield is denoted by $\tilde{g}^i_\mathrm{A(B),\,m,\,Cas}$. These couplings are dependent on the amplitude $\op{z}_{m} = \sqrt{\hbar/2\tilde{m}\omega_m} (\op{a}_m + \op{a}^\dagger_m)$ and the shape $u_{m}(r_{A(B)})$ of the vibrational mode $m=\{(k,l)\}$ at the position $r_{A(B),i}$ of the cat state:
\begin{equation}
  \tilde{g}^i_\mathrm{A(B),\,m,\,Cas} = \frac{\hbar c \pi^3}{720} \left(\frac{\varepsilon_r - 1}{\varepsilon_r + 1}\right)\varphi(\varepsilon_r) \frac{R}{(\mathscr{L} + \op{z}_m u_m(r))^2} \approx g_\mathrm{PFA} \left(\frac{1}{\mathscr{L}^2} + \frac{2 \op{z}_m u_m(r_{A(B),i})}{\mathscr{L}^3}\right) .
\end{equation}
Ignoring the first term in the expansion, which just produces a global phase in the evolved system, the couplings $g_\mathrm{Cas}$ appearing in eq. \eqref{eq:5:hamiltonian} are finally given by
\begin{equation}
  g^i_\mathrm{A(B),\,m,\,Cas} = g_\mathrm{PFA} \frac{2u_m(r_{A(B),i})}{\mathscr{L}^3} \sqrt{\frac{\hbar}{2 \tilde{m} \omega_m}} .
\end{equation}
It is possible to analytically calculate the time evolution of a system consisting of the initial state $\rho_\mathrm{system}$ (given by eq. \eqref{eq:2:initial-state}) combined with the infinite vibrational modes $\rho_\mathrm{th}$ of the thermal shield
\begin{equation}
  \rho_0 = \bigotimes_{m\in\{(k,l)\}} \left(\rho_\mathrm{th,\,m}\right) \otimes \rho_\mathrm{system} .
\end{equation}
These thermal states can be expanded into coherent states $\ket{\alpha} = \op{D}(\alpha)\ket{0}$ as \cite{Steiner_2024}
\begin{equation}
  \rho_\mathrm{th,m} = \frac{1}{Z} \sum_{n=1}^{\infty} e^{-\beta \hbar \omega_m (n + 1/2)} \ketbra{n} = \int \dd \alpha^2 \frac{1}{\pi \bar{n}} e^{-\frac{\abs{\alpha}^2}{\bar{n}}} \ketbra{\alpha}
\end{equation}
where $\bar{n}$ is the average occupation number. The time evolution of the particle-system $\rho_\mathrm{system}(t) = \tr_{th}\left\{\rho(t)\right\}$ can be calculated by tracing out all states corresponding to the thermal shield.

\begin{figure}[!htbp]
  \centering
  \includegraphics[width=\textwidth]{./../figures/vibrations/entanglement-hamiltonian.pdf}
  \caption{Entanglement dynamics in front of a thermal shield in mode $(1,0)$ at different temperatures. Only at specific times $2\pi k / \omega_{1,0},\ k\in\mathbb{N}$, entanglement is observable. This behavior is expected and aligns with the findings in Ref. \cite{Pedernales_2022}.}
\end{figure}

\begin{figure}[!htbp]
  \centering
  \includegraphics[width=\textwidth]{./../figures/vibrations/entanglement-multiple-modes.pdf}
  \caption{Entanglement dynamics in front of a thermal shield. In orange, the first 50 modes have been used in the numeric calculation. The effect of all remaining modes is around $1.7 \times 10^{-11}\,\%$. In blue, all modes except the first mode $(1,0)$ have been considered at different temperatures ranging vom $1\si{mK}$ up to $20\si{mK}$. The particle-shield separation is fixed at $L = 2R = 20\si{\mu m}$.}
\end{figure}

\begin{figure}[!htbp]
  \centering
  \includegraphics[width=\textwidth]{./../figures/vibrations/all-modes-maximum-entanglement-L-t-max.pdf}
  \caption{All modes, Entanglement at t-max}
\end{figure}


\subsection{Small shields}


\section{Discussion on the optimal setup}\label{sec:4:discussion}

- No global maximum

- dependent on experimental stuff

- Masses and distances and so on must be bounded by 10xECasimir < EGravity

How can one find the best parameters depending on the experimental requirements?
\begin{enumerate}
  \item Specify maximum coherence time (how long the experiment can take)
  \item This fixes $L^3/(M_A M_B \Delta x_A \Delta x_B)$
  \item Look up, what maximum $\Delta \theta$ and $\Delta L$ you can get at this time (Dependent on the amount of entanglement you want to have at the end)
  \item Take the minimum possible $L/R$ in account (dependent on trap stiffness and so on) Ideally, large enough to reduce casimir interactions
\end{enumerate}


-------------------------------------- 

\begin{enumerate}
  \item Given a target measuring time $t_\mathrm{target}$, this fixes the ratio
  \begin{equation}\label{eq:4:fixed-ratio}
    \frac{L^3}{M_A M_B \Delta x_A \Delta x_B} = 5.036\times 10^{22}\si{m/kg^2s} \cdot t_\mathrm{target} = \mathrm{const.}
  \end{equation}
  Most likely, the superposition size $\Delta x_{A,B}$ is also not changeable and one has to be satisfied by what can be achieved experimentally. The delocalized states of mass $M$ has to interact gravitationally with each other in a laboratory setting for at least the duration of the measuring process $t_\mathrm{target}$. These considerations limits the coherence time and usually, low times in the order of milliseconds to seconds are favorable.
  \item In principle, one does not require to measure at the time of maximum entanglement where $E_N = 1$. If a lower quantity of entanglement is enough for a set experimental goal, I recommend, measuring at a lower time than $t_\mathrm{target} = \tau t_\mathrm{max}$ ($\tau < 1$) !!!!FIGURE!!!!. This on the other hand increases the fixed ratio eq. \eqref{eq:4:fixed-ratio} by a factor of $1/\tau$.
  Therefore, it is possible to use smaller superposition sizes $\Delta x$, lighter masses or increase the distance $L$ which decreases Casimir interactions.
  \item Most likely, the minium angular 
\end{enumerate}

%% ---- bibliography ------------------------------------------------
\newpage
\nocite{*}

\printbibliography

%% ---- appendix ------------------------------------------------
\appendix
\newpage

% \chapter{Proofs and other stuff}



%%%% ---------- Variant 2 ---------------
% \begin{proposition}\label{theorem:negativity}
% The \emph{negativity} $\mathscr{N}(\rho)$ of a state $\rho$ is given as the absolute sum of all negative eigenvalues of $\rho$: 
% \begin{equation}
%   \mathscr{N}(\rho) \equiv \frac{\norm{\rho^{\Gamma_A}}_1 - 1}{2} = \abs{\sum_{\lambda_i < 0} \lambda_i}
% \end{equation}
% \end{proposition}

% \begin{proof}
%   The proof is given in two steps, which in parts is given by Vidal \cite{Vidal_2001}:
% \begin{enumerate}
%   \item It is known, that the density matrix is hermitian: $\rho^\dagger = \rho$. For a general hermitian matrix $A$, the trace norm $\norm{A}_1 \equiv \tr \sqrt{A^\dagger A}$ is equal to the sum of the absolute eigenvalues of A.
%   This can be immediately seen by the spectral theorem:
%   \begin{equation*}
%     \tr \sqrt{A^\dagger A} = \tr \sqrt{A^2} = \tr{U\sqrt{\diag(\lambda_1, \dots)^2}U^\dagger} = \sum_i \sqrt{\lambda_i^2} .
%   \end{equation*}
%   \item The trace norm of the density matrix is given as $\norm{\rho}_1=\sum \lambda_i = \tr \rho = 1$. The partial transpose $\rho^{\Gamma_A}$ obviously also satisfies $\tr \rho^{\Gamma_A} = 1$ but might have negative eigenvalues. Since $\rho^{\Gamma_A}$ is still hermitian, the trace norm is given by
%   \begin{equation*}
%     \norm{\rho^{\Gamma_A}}_1 = \sum_i\abs{\lambda_i} = \sum_{\lambda_i \ge 0} \lambda_i + \sum_{\lambda_i < 0} \abs{\lambda_i} = \sum_i \lambda_i + 2\sum_{\lambda_i < 0} \abs{\lambda_i} = 1 + 2\sum_{\lambda_i < 0} \abs{\lambda_i} ,
%   \end{equation*}
%   where in the last step $\sum \lambda_i = \tr \rho^{\Gamma_A} = 1$ was used. The negativity can be defined as $\mathscr{N}(\rho) = \abs{\sum_{\lambda_i < 0} \lambda_i}$ and the statement is shown.
% \end{enumerate}
% \end{proof}
% \begin{remark}
%   The \emph{logarithmic negativity} \cite{Plenio_2005} relates to the negativity as follows
%   \begin{equation}
%     E_N(\rho) = \log_2 \norm{\rho^{\Gamma_A}}_1 = \log_2 \left( 2\mathcal{N}(\rho) + 1 \right)
%   \end{equation}
%   and can therefore be easily calculated by using the above \cref{theorem:negativity}. In comparison to the negativity, logarithmic negativity has additive properties \cite{Plenio_2005a}:
%   \begin{equation*}
%     E_N\left(\rho \otimes \sigma \right) = E_N(\rho) + E_N(\sigma) 
%   \end{equation*}
% \end{remark}

\chapter{TITLE TO BE DONE}
\section{Evolution under a gravitational Hamiltonian}
In this section the time evolution of a system under Hamiltonian eq. \eqref{eq:2:general-hamiltonian} is calculated a) using the gravitational interaction $\op{H}_G$ as a perturbation b) using an exact time evolution of coherent states.

\subsection{Using time dependent perturbation theory}
\label{apx:general-state-perturbation-theory}
A general biparty Fock state $\ket{\psi_0} = \ket{kl}$ with $k, l \in \mathbb{N}_0$ can be evolved in time under a Hamiltonian eq. \eqref{eq:2:general-hamiltonian} treating the gravitational interaction $H_G = -\hbar g (\op{a}_1\op{a}_2^\dagger + \op{a}_1^\dagger\op{a}_2)$ as a perturbation. 
The resulting state $\ket{\psi(t)}$ after some time $t$ is in the most general form given as
\begin{equation}
  \ket{\psi(t)} = \sum_{i, j \geq 0} c_{i,j}(t) \ket{i,j}
\end{equation}
where the coefficients $c_{i,j}(t)$ are given by first order perturbation theory as
\begin{equation}
  c_{i,j}(t) = c_{i,j}(t = 0) - \frac{i}{\hbar} \int_0^t \dd t' \bra{ij}\op{H}_G\ket{kl} e^{-i(E_{kl}-E_{ij})t'/\hbar} .
\end{equation}
The exponent is given by the energy of the appropriate Fock states $E_{kl}-E_{ij} = \hbar \omega (k+l - (i+j))$ and the matrix element in the integrand can be calculated to
\begin{equation}
  \bra{ij}\op{H}_G\ket{kl} =
  \begin{cases}
    -\hbar g & \text{if } i = k \pm 1 \text{ and } j = l \mp 1 \\
    0 & \text{otherwise}
  \end{cases}.
\end{equation}
The coefficients for $t=0$ are trivially given from the initial state as
\begin{equation}
  c_{i,j}(t=0) = \begin{cases}
    1 & \text{for } i,j = k,l \\
    0 & \text{otherwise}
  \end{cases}.
\end{equation}
For the non-zero states the energies in the exponent equate to zero and the evolved state is given by (up to a normalization)
\begin{equation}\label{eq:apx:perturbation-result}
  \ket{\psi(t)} = \ket{kl} - i g t \ket{k-1,l+1} - i g t \ket{k+1,l-1} + \mathcal{O}(g^2).
\end{equation}
The result eq. \eqref{eq:2:general-evolved-state} is represented by eq. \eqref{eq:apx:perturbation-result} for the case of $k=1$ and $l=0$.


\subsection{Using an exact time evolution}
\label{apx:general-coherent-state-evolution}
The Hamiltonian eq. \eqref{eq:2:general-hamiltonian} can be rewritten using symmetric and antisymmetric normal modes
\begin{equation}
  \op{a}_\pm = \frac{1}{\sqrt{2}} \left( \op{a}_1 \pm \op{a}_2 \right)
\end{equation}
in the form of 
\begin{equation}
  \op{H} = \hbar \omega_+ \op{a}^\dagger_+\op{a}_+ + \hbar \omega_- \op{a}^\dagger_-\op{a}_- ,  \quad  \omega_\pm = \omega \pm (-g)
\end{equation}
The time evolution of a state
\begin{equation}\label{eq:apx:coherent-initial-state}
  \ket{\psi(t)} = \ket{\alpha}_1\ket{\beta}_2 = \ket{\frac{1}{\sqrt{2}} (\alpha + \beta)}_+\ket{\frac{1}{\sqrt{2}} (\alpha - \beta)}_-
\end{equation}
A general coherent state $\ket{\gamma}$ evolves in time under an Hamiltonian $\op{H} = \hbar \omega \op{a}^\dagger \op{a}$ like $\ket{\gamma(t)} = \ket{e^{-i\omega t} \gamma}$ which can be used to evolve the state in eq. \eqref{eq:apx:coherent-initial-state}:
\begin{align}
  \ket{\psi(t)} 
  &= \ket{ \frac{1}{\sqrt{2}} e^{-i\omega_+t} (\alpha + \beta) }_+\ket{ \frac{1}{\sqrt{2}} e^{-i\omega_-t} (\alpha - \beta) }_- \\
  &= \ket{ e^{-i\omega t} \left( \alpha \cos gt - \beta \sin gt \right) }_1 \ket{ e^{-i\omega t} \left( - \alpha \sin gt + \beta \cos gt \right) }_2, 
\end{align}
where in the last line the back-transformation from the $\pm$-modes \eqref{eq:apx:coherent-initial-state} was used.
\chapter{Casimir interactions}
\section{Polarizability of a dielectric sphere}
\label{apx:polarizability-sphere}

The polarizability $\alpha$ is defined via
\begin{equation}
  \vec{E_\infty} \alpha = \vec{p},
\end{equation}
where $\vec{p}$ is the induced dipole moment and $\vec{E_\infty}$ is the external electric field that induces the dipole moment. For a linear and uniform dielectric, it is given as $\vec{p} = \mathcal{V} \varepsilon_0 (\varepsilon_r - 1) \vec{E_\mathrm{in}}$ \cite[p. 220-226]{Griffiths_2018}. Here, $\mathcal{V}$ is the volume of the object and $\vec{E_\mathrm{in}}$ is the electric field inside the dielectric.
The electrostatic boundary conditions for the problem are given by
\begin{equation}
  V_\mathrm{in} \big|_{r=R} = V_\mathrm{out} \big|_{r=R} 
  \quad \text{and} \quad 
  \varepsilon_r\varepsilon_0\pdv{V_\mathrm{in}}{r}\bigg|_{r=R} = \varepsilon_0\pdv{V_\mathrm{out}}{r} \bigg|_{r=R}
\end{equation}
and the electric potential outside of the sphere at $r\rightarrow\infty$ should be equal to the external dipole-inducing field $V_\mathrm{out} |_{r\rightarrow\infty} = -\vec{E_\infty} \cdot \vec{r} = -E_\infty r\cos\theta$.
The electric potential inside and outside the sphere can be calculated using the spherical decomposition of the general electric potential $V \propto 1/\abs{\vec{r} - \vec{r'}}$ into Legendre Polynomials $P_l$ \cite[p. 188-190]{Griffiths_2018}:
\begin{align}
  V_\mathrm{in}(r, \theta) &= -E_\infty r\cos\theta + \sum_{l=0}^{\infty} A_l r^l P_l(\cos\theta), \\
  V_\mathrm{out}(r, \theta) &= -E_\infty r \cos\theta + \sum_{l=0}^{\infty} \frac{B_l}{r^{l+1}} P_l(\cos\theta).
\end{align}
Applying both boundary conditions, it follows that \cite[p. 249-251]{Griffiths_2018}
\begin{equation}
  \begin{cases}
    A_l = B_l = 0 & \text{for } l \neq 1, \\
  A_1 = -\frac{3}{\varepsilon_r + 2}E_\infty, \quad B_1 = \frac{\varepsilon_r-1}{\varepsilon_r + 2}R^3E_\infty
  \end{cases}
\end{equation}
and the resulting  homogenous electric field $\vec{E_\mathrm{in}} = -\vec{\nabla} V_\mathrm{in}$ inside the sphere is given as
\begin{equation}
  \vec{E_\mathrm{in}} = \frac{3}{\varepsilon_r + 2} \vec{E_\infty} .
\end{equation}
The field is shown on the right in \cref{fig:dielectric-sphere-field}.The polarizability $\alpha$ of the sphere can be now be determined to
\begin{equation}
  \alpha_\mathrm{sphere} = 4\pi \varepsilon_0 R^3 \left(\frac{\varepsilon_r - 1}{\varepsilon_r + 2}\right).
\end{equation}
Depending on the definition, sometimes the factor $4\pi\varepsilon_r$ is dropped.
% \begin{figure}[!htbp]
%   \centering
%   \includegraphics[width=0.8\textwidth]{./../figures/field-dielectric-sphere.pdf}
%   \label{fig:field-dielectric-sphere}
%   \caption{Electric field lines through an dielectric sphere}
% \end{figure}

\begin{figure}[!htbp]
  \centering
  \begin{subfigure}[b]{0.48\textwidth}
      \centering
      \includegraphics[width=\textwidth]{./../figures/potential-dielectric-sphere-small.pdf}
  \end{subfigure}
  \hfill
  \begin{subfigure}[b]{0.48\textwidth}
      \centering
      \includegraphics[width=\textwidth]{./../figures/others/field-dielectric-sphere-small.pdf}
  \end{subfigure}
  \caption{\textbf{left:} Electric potential $V$ of a dielectric sphere in a external electric field $\vec{E_\infty} \parallel \vec{e_z}$. \textbf{right:} The corresponding electric field lines inside and outside the dielectric sphere.}
  \label{fig:dielectric-sphere-field}
\end{figure}
\chapter{The shield and its consequences}

\section{Blocking of the shield}\label{apx:blocking-of-the-shield}
Assume two spheres $A$ and $B$ with charge $q_A$ and $q_B$ separated by a distance $2L$ on the $x$-axis. A circular shield is placed perfectly in the center of the spheres orthogonal to the direct connection between them. The magnitude of the field at a distance $z$ in the direction $\vec{e}_x$ from this connection line is given by 
\begin{equation}
  E_x(z) = \frac{L(q_A - q_B)}{4\pi\varepsilon_0 (L^2 + z^2)^{3/2}}
\end{equation}
The total flux the circular shield with radius $r_s$ is given by
\begin{equation}
  \Phi = \int_{0}^{r_s}\dd z \int_{0}^{2\pi} z \dd \varphi \, E_x(z) = \frac{(q_A - q_B)}{2 \varepsilon_0} \left[ 1 - \frac{L}{\sqrt{L^2 + r_s^2}} \right] .
\end{equation}
Comparing the total flux for $r_s \rightarrow \infty$ with the flux through the shield, one can arrive at the charge-independent \emph{effectiveness} $\eta$ of the shield as
\begin{equation}
  \eta = \frac{\Phi}{\Phi_\infty} = 1 - \frac{L}{\sqrt{L^2 + r_s^2}}
\end{equation}
and thus a shield with radius
\begin{equation}
  r_s = L \sqrt{\frac{1 - (1-\eta)^2}{(1-\eta)^2}}
\end{equation}
will block a fraction $\eta$ of the total field.

\section{Thermal harmonic oscillator}\label{apx:thermal-harmonic-oscillator}
The amplitude $z$ of a single vibrational shield-mode $(k,l)$ with frequency $\omega_{kl} \equiv \omega$ behaves like a quantum harmonic oscillator. 
The average amplitude $\avg{z}_n = 0$. 
The variance $(\Delta z)^2 = \avg{z^2} - \avg{z}^2$ however is given by
\begin{equation}
  (\Delta z)^2_n = \avg{z^2}_n = \frac{\hbar}{2m\omega} (1+2n) .
\end{equation}
At a temperature $T$, the occupation of the modes is described by the boltzmann distribution:
\begin{equation}\label{eq:apx:thermal-oscillator-variance}
  \avg{z^2}_T = \sum_{n=0}^{\infty} \frac{1}{Z} e^{-\beta E_n} \avg{z^2}_n,
\end{equation}
where $\beta = 1/k_B T$, $E_n = \hbar \omega (n+1/2)$ is the energy of mode $n$ and
\begin{equation}
  Z = \sum_{n=0}^{\infty} e^{-\beta E_n} = \frac{e^{-\beta \frac{\hbar \omega}{2}}}{1-e^{-\beta \hbar \omega}}
\end{equation}
is the partition function. Using known series, the expression eq. \eqref{eq:apx:thermal-oscillator-variance} can be evaluated to
\begin{align}
  (\Delta z)^2_T = \avg{z^2}_T &= \frac{\hbar}{2m\omega} \sum_{n=0}^{\infty} \frac{1}{Z}\left[e^{-\beta E_n} + 2ne^{-\beta E_n}\right] \\
  &= \frac{\hbar}{2m\omega} \left[1 + \frac{2}{Z}\sum_{n=0}^{\infty} n e^{-\beta E_n}\right] \\
  &= \frac{\hbar}{2m\omega} \left[1 + \frac{2e^{-\beta\hbar\omega}}{1-e^{-\beta\hbar\omega}} \right] = \frac{\hbar}{2m\omega} \coth(\frac{\hbar \omega}{2 k_B T})
\end{align}

%% ---- index --------------------------------------------------
% \printindex

\end{document}
