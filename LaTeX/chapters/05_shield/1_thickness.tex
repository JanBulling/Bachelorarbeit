\section{Thickness and size of the shield}\label{sec:5:shield-size}
The thickness and radius $r_s$ of the shield can be estimated by considering a real shield made of a conducting material with large conductivity $\sigma$. 
Even a superconducting shield could be considered for a almost perfect shielding against electrostatic field.
For a real shield made of e.g. copper, the transmission $T$ of electromagnetic waves is given by \cite{Vandenbosch_2022}
\begin{equation}
  T = \abs{\frac{\vec{E}_\mathrm{after}}{\vec{E}_\mathrm{before}}} = \frac{2}{Z_0 \sigma d}
\end{equation}
where $Z_0 = 377\si{\Omega}$ the impedance of free space (provided the shield is placed in a vacuum or air) and $d$ the thickness of the shield.
The electric conductivity has a strong dependence on temperature and decreases with $1/T^5$ \footnote{This behavior is only true for temperatures below the Debye temperature. For copper, this limit is around $\Theta_D = 343\si{K}$. In the experiment, the shield is cooled down, so the low-temperature limit of the electric conductivity for metals is very valid \cite{Berman_1952}.} with increasing temperatures \cite[p. 284-286]{Gross_2018}. The electric conductivity for copper at room temperature ($\sigma = 59.6\times 10^6 \si{S/m}$) should therefore be a very valid worst-case approximation. 
Measured data suggest a conductivity of $\sigma(T = 10) \approx 1.5\times 10^{10}\si{S/m}$ \cite{Berman_1952}.

The general goal is to place the particles as close as possible to increase the gravitational interaction between them, requiring to choose the thickness of the shield as thin as possible.
To estimate the thickness of the shield, the condition that entanglement between the masses should be built up faster mainly due to gravity is used.
All other possible interactions like Coulomb or Casimir forces, should be suppressed sufficiently by the shield.
To quantify the amount of entanglement built-up over time, I obt for a measure I call \emph{entanglement rate}
\begin{equation}\label{eq:5:entanglement-rate}
  \Gamma = \dv{t} E_N(\rho)\Big|_{t=0}
\end{equation} 
where $E_N$ is an appropriate entanglement measure - in this case the logarithmic negativity \cite{Plenio_2005} introduced in \cref{sec:2:entanglement-measures}.
For the gravitational entanglement, the entanglement rate can be calculated in the parallel orientation using eq. \eqref{eq:2:entanglement-dynamics-parallel}:
\begin{equation}\label{eq:5:entanglement-rate-gravity}
  \Gamma_\mathrm{Gravity} = \frac{G M_A M_B \Delta x_A \Delta x_B}{16 \hbar L^3 \log 2} .
\end{equation}
The entanglement rate for other interactions should be smaller (and ideally a lot smaller) than the entanglement rate due to gravity, otherwise the eventually measured entanglement may not be solely due to gravitational interactions.
In the following, the required size of the shield is estimated for undesired Coulomb- and Casimir interactions.

\subsection{Shielding Coulomb-Interactions}
The main focus of the shield is to block electromagnetic interactions between the particles.
Experimentally, it might be beneficial for the masses to carry a small amount of charge to help with the trapping and allow the usage of electric traps \cite{GonzalezBallestero_2021}.
The potential energy due to Coulomb interactions
\begin{equation}
  V = \frac{1}{4\pi\varepsilon_0} \frac{q_A q_B}{2L}
\end{equation}
between the particles has the same structure as the gravitational potential and can therefore induce entanglement by the same logic.
Using a shield, this coupling is suppressed by a factor of $T$.
The entanglement rate $\Gamma_\mathrm{Coulomb}$ is therefore easily calculable in a similar form as eq. \eqref{eq:5:entanglement-rate-gravity} to
\begin{equation}\label{eq:5:entanglement-rate-coulomb}
  \Gamma_\mathrm{Coulomb} = \frac{T \abs{q_A q_B} (\Delta x)^2}{64 \pi \varepsilon_0 \hbar L^3 \log 2}
\end{equation}
Requiring that $\Gamma_\mathrm{Gravity} > \Gamma_\mathrm{Coulomb}$, this inequality yields a maximum transmission and thus a minimum thickness of
\begin{align}\label{eq:5:coulomb-gravity-condition}
  T \frac{\abs{q_A q_B}}{4 \pi \varepsilon_0} \, &< \, G M_A M_B \\
  \Longleftrightarrow \quad\quad\quad\quad d \, &> \, \frac{1}{2 \pi \varepsilon_0 Z_0 \sigma G} \frac{\abs{q_A q_B}}{M_A M_B} = \frac{9}{32}\frac{1}{Z_0 \sigma} \frac{1}{\pi^3 \varepsilon_0 G \rho^2} \frac{e^2}{R^6}
\end{align}
where in the last step $M_A = M_B$ and $q_A = q_B = e$ has been assumed.
The shield thickness heavily depends on the radius of the particle where small and light masses make the shield unworkably thick.
A silica sphere with a radius of $R=10^{-5}\si{m}$ would require a shield around the thickness of $10\si{nm}$ at $4\si{K}$ up to $2.5\si{\mu m}$ at room temperature.
A realistic thickness for the shield at low temperatures could therefore be $d=100\si{nm}$.

Electric fields however can still propagate around a Faraday shield of finite size and induce entanglement. It is however possible, to estimate the required radius $r_s$ of the shield to block a specific amount $\eta$ of the electric field.
The resulting radius corresponding to $\eta$ is derived in appendix \ref{apx:blocking-of-the-shield} and the result is shown in \cref{fig:5:shield-radius}.
\begin{figure}[!ht]
  \centering
  \includegraphics[width=\textwidth]{./../figures/others/shield-radius.pdf}
  \caption{Radius of the shield depending on the shield effectiveness $\eta$. Additionally a real shield with different thicknesses $d$ at room temperature is considered. For a shielding between $99.5-99.9\%$ ($\eta = 0.995-0.999$), a radius of at least $r_s =200-1000L$ should be used. For a sphere-plate distance of $L=2 \times 10^{-5}\si{m}$, the shield radius should be in the order of millimeters and centimeters.}
  \label{fig:5:shield-radius}
\end{figure}
The transmission $T$ should be replaced by a modified transmission $\tilde{T} = T\eta + (1-\eta)$ where the shield effectiveness $\eta$ depends on $r_s$. 
Now, the condition eq. \eqref{eq:5:coulomb-gravity-condition} introduces a limit for the minimum effectiveness and thus a limit on the minimum radius of
\begin{equation}
  \eta_\mathrm{min} = 1 - \frac{4\pi \varepsilon_0 G M_A M_B}{\abs{q_1 q_2}} .
\end{equation}
Thus using the setup with the parameters from earlier, a minimum effectiveness of $\eta_\mathrm{min} \gtrsim 0.99997$ and thus a radius of $\gtrsim 66\si{cm}$ is required.
This shield is too large for all practical purposes and it might be beneficial to choose slightly heavier masses to reduce the shield size to the orders of $\sim 1\si{cm}$. This would require both spheres to have approximately double the radius than before.
Using neutral masses without any charge would is also beneficial and the shield size could be reduced to only the size of the spheres themselves, however it might be an engineering challenge, to trap and levitate uncharged massive particles.





\subsection{Shielding Casimir-Interactions}
Similarly to Coulomb interactions, it is possible to to estimate the required thickness of the shield necessary to sufficiently suppress Casimir interactions. Between two spheres with radius $R$ and separation $2L$, the Casimir potential reads \cite{Emig_2007}
\begin{equation}
  V = -\frac{23 \hbar c}{4\pi \cdot 128 L^7} \left( \frac{\varepsilon_r - 1}{\varepsilon_r + 2} \right)^2 R^6 .
\end{equation}
The entanglement rate can be obtained similarly as before by expanding the casimir potential between the spheres in small $\Delta x$ and computing the logarithmic negativity as before:
\begin{equation}
  \Gamma_\mathrm{Casimir} = T^2 \frac{161}{4096} \frac{c R^6 (\Delta x)^2}{\pi L^9 \log 2}\left( \frac{\varepsilon_r - 1}{\varepsilon_r + 2}\right)^2 .
\end{equation}
The dependence on $T^2$ is only a systematic guess but should be sufficient for a basic estimation. Casimir and van der Waals forces are second order effects in the dipole-dipole interaction \cite{Bordag_2001}.
Demanding again, that the entanglement due to gravity should be mediated faster than due to Casimir interactions $\Gamma_\mathrm{Gravity} > \Gamma_\mathrm{Casimir}$, one arrives at the expression
\begin{align}
  T^2 \frac{161 c R^6}{256 \pi L^6} \left( \frac{\varepsilon_r - 1}{\varepsilon_r + 2}\right)^2 \, &< \, \frac{G M_A M_B}{\hbar} \\
  \Longleftrightarrow \quad\quad\quad d^2 \, &> \, \frac{4 \cdot 161 c \hbar R^6}{256 Z_0^2 \sigma^2 \pi L^6 G M_A M_B} \left( \frac{\varepsilon_r - 1}{\varepsilon_r + 2}\right)^2 \\
  \Longleftrightarrow \quad\quad\quad\  d \, &> \, \sqrt{\frac{1449}{4096} \frac{c \hbar}{G \pi^3}} \frac{2}{Z_0 \sigma \rho L^3} \frac{\varepsilon_r - 1}{\varepsilon_r + 2}
\end{align}
where again in the last step I assume $M_A = M_B$. For large separations, the shield thickness can go arbitrary low because of the weakness of the casimir interactions at larger distances. Between two silica spheres separated in the order of magnitude as the radius ($L = 2\times 10^{-5}\si{m}$), the required minimum thickness is between $4\times 10^{-11}\si{m}$ at $4\si{K}$ and $10 \si{nm}$ at room temperature.
Either way, it is much thinner than a Faraday shield required to shield electrostatic Coulomb interactions. The effect of $\varepsilon_r$ was neglected in these numbers. The minimum thickness only changes by a factor between $0$ and $1$ due to dielectric materials.
In fact, these thicknesses are much thinner than recommended, because the shield loses rigidity.
It turns out that the vibrational frequency and thus the energy of a thermal shield depends linearly on the thickness and a thinner shield would excite more and larger vibrations. A more detailed discussion of this matter is given in the next section.


\subsection{Gravitational effects of the shield}
The gravitational interaction between the masses and the shield is generally neglected, as it has no influence on the generation of entanglement between the particles.
At most, indirect entanglement between the particles, mediated by the thermal oscillations of the shield, is possible, since both masses couple to the shield gravitationally.
However, as calculated in \cref{sec:5:thermal-entanglement}, this second-order effect is very weak and not problematic at all, since it is still a gravitationally mediated entanglement - which is exactly what the experiment wants to measure anyway.
The gravitational force between a sphere with mass $M$ and a infinitesimal mass segment $\dd m = r \rho_\mathrm{Cu} d \dd r \dd \varphi$ of the shield made of copper with density $\rho_\mathrm{Cu} = 8960\si{kg/m^3}$ at a distance $r$ from the center is given by
\begin{equation}
  \dd \vec{F} = \frac{G M \dd m}{\ell} \boldsymbol{\hat{\ell}} 
  \quad \Rightarrow \quad
  \dd F_z = \frac{G M r \rho_\mathrm{Cu} d}{\ell^2} \dd r \dd \varphi \cos \theta,
\end{equation}
where $\ell^2 = r^2 + L^2$ denotes the distance between the sphere and the mass segment and $\theta = \arccos L/\ell$ is the angle between them.
The total attractive force between the mass and the shield with radius $r_s$ is therefore
\begin{equation}
  F_z = GM \rho_\mathrm{Cu} d L \int_{0}^{r_s} \dd r \int_{0}^{2\pi} \dd \varphi \, \frac{r}{(r^2 + L^2)^{3/2}} = 2\pi G M \rho_\mathrm{Cu} d \left(1 - \frac{L}{\sqrt{L^2 + r_s^2}}\right) .
\end{equation}
For large shields $r_s \gg L$ this is independent of the particle-shield separation $L$.
For a shield with thickness $d = 100\si{nm}$ and the usual silica particle, the attraction force is around $F_\mathrm{particle-shield} \approx 4.1\times 10^{-24} \si{N}$ which is comparable with the attraction gravitational attraction force between the two masses at $F_\mathrm{particle-particle} \approx 5.0 \times 10^{-24}\si{N}$ but is much weaker than the Casimir attraction between the particle and the shield with $F_\mathrm{Casimir} \approx 1.4 \times 10^{-17} \si{N}$.
Therefore, the gravitational effect of the shield can really be neglected in all practical calculations.