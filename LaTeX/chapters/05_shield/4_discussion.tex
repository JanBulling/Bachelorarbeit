\section{Discussions of the effects of the thermal shield} \label{sec:5:discussion}

In this chapter, two different approaches to quantify the impact of a thermal shield at temperature $T$ on the entanglement generation have been employed, revealing different nuanced insights into the system's dynamic behavior.

The naive approach described in the beginning of \cref{sec:5:thermal-entanglement} assumes a locally static and flat thermal shield, treating vibrational amplitudes $z_{kl}$ as statically and normally distributed over multiple measurements.
This method systematically overestimates the dynamics by presuming simultaneous maximal contributions from all vibrational modes.
In reality, different modes might cancel themselves partially out, reducing the overall deformation.
Critically, this approximation is constrained by two fundamental conditions of requiring a sufficiently large and thus linearizbale shield ($r_s \gg R$) as well as low-frequency vibrational mode satisfying $2\pi/\omega_{kl} \approx t_\mathrm{max}$.

For the second method, the entanglement generation between the particles was calculated analytically by solving the Hamiltonian dynamics for small amplitudes $\Delta z \ll L$. This approach provides a more accurate description of the time-dependent system and reveals that entanglement can be partially recovered at specific times (notably at $2\pi/\omega_{1,0}$), even for small separations $L$ and large shields.

In general, the effects of the thermal shield can be mitigated by reducing the decoherence effects via the following methods:
\begin{description}
  \item[Lowering the shield's temperature] Reducing the temperature of the shield decreases vibrational amplitudes and the associated decoherence. Temperatures or around $4\si{K}$ are desirable as they are experimentally accessible using liquid helium cooling. Temperatures as low as $20\si{mK}$ are theoretically reachable by $^3\mathrm{He}/^4\mathrm{He}$ dilution refrigerators \cite{Das_1965, Zu_2022}. All cooling mechanisms however induce additional vibrational noise due to their mechanical components and additional studies about the effect of such vibrations has to be considered.

  \item[Increasing the particle-shield separation] Larger separations reduce the relative effect of the vibrations as $\Delta z / L \rightarrow 0$. In the naive approach, separations of at least $L \gtrsim 10 R$ are required, making the shield almost unnecessary as for similar separations, the Casimir interactions between both particles are smaller than the gravitational interaction (see discussion in \cref{sec:2:experimental-problems}). The presence of the shield could even potentially worsen the entanglement generation. In the more detailed and analytical method in \cref{subsec:5:entanglement-analytical}, separations of around $5R$ are possible, but require measurements at very precise points in time.
  
  \item[Reducing the shield's radius] Decreasing $r_s$ does increase the vibrational frequencies quadratically ($\omega_{kl} \propto 1/r_s^2$) and simultaneously decreases the amplitudes $\Delta z_{kl} \sim r_s$. The analytical approach shows a strong dependence on the shield's radius $r_s$, where halving the radius nearly restores entanglement even for small separations (see \cref{fig:apx:entanglement-thermal-shield-rs-5mm} and \cref{fig:apx:entanglement-thermal-shield-rs}). The increase in the vibration frequency suggests that in the limit $2\pi/\omega \ll t_\mathrm{max}$, vibrations average out over time. A reduction in $r_s$ is however only possible for uncharged, neutral particles that do not interact via a direct Coulomb interaction, necessitating magnetic or optical trapping methods.
\end{description}
By combining these approaches, the thermal shield's impact can be reduced, creating better conditions for entanglement generation. Measurements however might only be possible at specific and precise points in time, particularly at $2\pi/\omega_{1,0}$ with required accuracies of approximately $\Delta t \sim \sqrt{\omega_{1,0} / T}$.

Improvements on the rigidity of the shield can also be considered:
Reinforcing the shield, for instance with a cross structure of thicker material, could increase vibrational frequencies and simultaneously decrease amplitudes of the vibrations.
Alternative shield designs (e.g. a star shape) might also be beneficial by potentially offering a more uniform vibrational mode distribution and potentially higher frequencies. 
For rectangular plates in instance, frequency increases are marginal, scaling by $\omega_{kl} = (k^2 + l^2)/(2r_s)^2 \sqrt{D \pi^4 / \rho d}$ \cite[p. 471-474]{Rao_2019} and thus improving entanglement generation only at most up to a constant.