\section{Discussions of the effects of the thermal shield} \label{sec:5:discussion}

In this chapter, two different strategies to calculate the effect of a thermal shield at temperature $T$ on the entanglement generation have been considered:

The naive approach described in \cref{sec:5:thermal-entanglement} assumes the shield to be locally flat and static over the measurement time. Vibrational amplitudes $z_{kl}$ are treated statically and normally distributed over multiple measurements with each mode inducing random phase shifts, similar to the placement variations considered in \cref{cha:entanglement-generation}.
This approach greatly overestimates the dynamics by assuming maximal effects from all vibrational that modes occur simultaneously. In reality however, different modes might cancel themselves partially out reducing the overall total deformation.
Furthermore this approximation is only valid for large and thus linearizbale shields ($r_s \gg R$) and low vibrating frequencies $2\pi/\omega_{kl} \approx t_\mathrm{max}$.

In the second method, the entanglement generation between the particles was calculates analytically by solving the Hamiltonian dynamics for small amplitude $\Delta z \ll L$. This method provides a more accurate depiction of the time-dependent system and reveals, that entanglement can be partially recovered at specific times (notably at $2\pi/\omega_{1,0}$), even for small separations $L$ and large shields.

In general, the effects of the thermal shield can be mitigated by reducing the decoherence effects via the following methods:
\begin{description}
  \item[Lowering the shield's temperature] Reducing the temperature of the setup and the shield decreases vibrational amplitudes and the associated decoherence.
  Temperatures around $4\si{K}$ are desirable as they are experimentally accessible using liquid helium cooling. Temperatures as low as $20\si{mK}$ are theoretically reachable by $^3\mathrm{He}/^4\mathrm{He}$ dilution refrigerators \cite{Das_1965, Zu_2022}. All cooling mechanisms however induce additional vibrational noise due to their mechanical components and additional studies about the effect of such vibrations has to be considered.

  \item[Increasing the particle-shield separation] Larger separations reduce the relative effect of the vibrations as $\Delta z / L \rightarrow 0$. In the naive approach in \cref{sec:5:thermal-entanglement}, separations of at least $L \gtrsim 10 R$ are required making the shield almost unnecessary, as for similar separations, the Casimir interactions between both particles are smaller than the gravitational interaction (see discussion in \cref{sec:2:experimental-problems}). The presence of the shield could even potentially worsen the entanglement generation. In the more detailed and analytical method in \cref{subsec:5:entanglement-analytical}, separations of around $5R$ are possible, but require measurement at very precise points in time.
  
  \item[Reducing the shield's radius] Decreasing $r_s$ does increase the vibrational frequencies quadratically ($\omega_{kl} \propto 1/r_s^2$) and simultaneously decreases the amplitudes $\Delta z_{kl} \sim r_s$. The results in the naive approach are independent of the shields radius. However, for small shields with large frequencies, this method is applicable. The analytical approach on the other hand shows a strong dependence on the shield's radius $r_s$, where halving the radius nearly restores entanglement even for small separations (see \cref{fig:apx:entanglement-thermal-shield-rs-5mm}). A reduction in $r_s$ is however only possible for uncharged, neutral particles that do not interact via a direct Coulomb interaction, necessitating magnetic or optical trapping methods.
\end{description}
By combining these approaches, the thermal shield's impact can be reduced, creating better conditions for entanglement generation. Measurements however might only be possible at specific and precise points in time, particularly at $2\pi/\omega_{kl}$ with fluctuations limited to approximately $\Delta t \sim 1/\sqrt{\bar{n}} \sim \sqrt{\omega_{1,0} / T}$.
Measuring some entanglement at other arbitrary points in times would require either lower temperatures or larger particle-shield separations.

For very small shields, which are only considerable for uncharged particles, the Casimir forces cannot be simplified to a sphere interacting with a perfectly flat plate.
Instead, mode shapes must be taken into account, slightly modifying the Casimir potential between the uneven plate and the particle.
Although challenging to estimate, the results from \cref{sec:3:imperfect-plates} suggest that the effects can be upper bounded by considering a flat plate moved closer by a distance $\Delta z \sim r_s$.
Additionally, the rapid frequency increase to $2\pi/\omega \ll t_\mathrm{max}$ during the measurement period suggests that shield vibrations would average out over time.
Thus, uncharged particles and therefore small shields are greatly preferable.

Improvements on the rigidity of the shield can also be considered. 
Reinforcing the shield, for instance with a cross structure of thicker material, could reduce vibrational frequencies of the shield by effectively reducing the size and increasing the overall thickness.
Alternative shield designs, such as a star shape, might also be beneficial by potentially offering more uniformly distributed and higher-frequency vibrations. For rectangular plates, frequency increases are marginal, scaling by $\omega_{kl} = (k^2 + l^2)/(2r_s)^2 \sqrt{D \pi^4 / \rho d}$ \cite[p. 471-474]{Rao_2019} and thus improving entanglement generation only at most up to a constant.