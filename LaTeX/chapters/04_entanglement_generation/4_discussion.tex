\section{Discussion on the optimal setup}\label{sec:4:discussion}

- No global maximum

- dependent on experimental stuff

How can one find the best parameters depending on the experimental requirements?
\begin{enumerate}
  \item Specify maximum coherence time (how long the experiment can take)
  \item This fixes $L^3/(M_A M_B \Delta x_A \Delta x_B)$
  \item Look up, what maximum $\Delta \theta$ and $\Delta L$ you can get at this time (Dependent on the amount of entanglement you want to have at the end)
  \item Take the minimum possible $L/R$ in account (dependent on trap stiffness and so on) Ideally, large enough to reduce casimir interactions
\end{enumerate}


-------------------------------------- 

\begin{enumerate}
  \item Given a target measuring time $t_\mathrm{target}$, this fixes the ratio
  \begin{equation}\label{eq:4:fixed-ratio}
    \frac{L^3}{M_A M_B \Delta x_A \Delta x_B} = 5.036\times 10^{22}\si{m/kg^2s} \cdot t_\mathrm{target} = \mathrm{const.}
  \end{equation}
  Most likely, the superposition size $\Delta x_{A,B}$ is also not changeable and one has to be satisfied by what can be achieved experimentally. The delocalized states of mass $M$ has to interact gravitationally with each other in a laboratory setting for at least the duration of the measuring process $t_\mathrm{target}$. These considerations limits the coherence time and usually, low times in the order of milliseconds to seconds are favorable.
  \item In principle, one does not require to measure at the time of maximum entanglement where $E_N = 1$. If a lower quantity of entanglement is enough for a set experimental goal, I recommend, measuring at a lower time than $t_\mathrm{target} = \tau t_\mathrm{max}$ ($\tau < 1$) !!!!FIGURE!!!!. This on the other hand increases the fixed ratio eq. \eqref{eq:4:fixed-ratio} by a factor of $1/\tau$.
\end{enumerate}