\chapter{Entanglement generation in front of a static shield}\label{cha:entanglement-generation}

The generalized form of the system described in \cref{cha:first-look} with the addition of a conducting Faraday shield is shown in \cref{fig:4:complete-setup}. As before, the masses are delocalized with superposition separations $\Delta x_A$ for mass $A$ and $\Delta x_B$ for mass $B$ respectively.
\begin{figure}[!htbp]
  \centering
  \def\svgwidth{\textwidth}
  \input{./../figures/problem.pdf_tex}
  \caption{Schematic depiction of a experimental setup for the detection of gravitationally induced entanglement between two masses $A$ and $B$. They are separated by a distance of $2L + \Delta L_A + \Delta L_B$ in arbitrary orientations given by the angles $\alpha$ and $\beta$ with small variations $\Delta \theta_{A(B)}$. All variations are assumed to be gaussian distributed. The masses are delocalized in a cat state with a separation $\Delta x_{A(B)}$ between the states $\psi_{A(B)}^1$ and $\psi_{A(B)}^2$. A conducting Faraday shield with thickness $d$ is placed in the center between the masses.}
  \label{fig:4:complete-setup}
\end{figure}
The superpositions are extended in arbitrary orientations $\alpha$ and $\beta$ ($\alpha, \beta \in [0, \pi)$ because of symmetry). Most notably, the orientation of $\alpha = \beta = 0$ represents the same \q{parallel configuration} as discussed earlier and I will refer to $\alpha = \beta = \pi/2$ as the \q{orthogonal configuration}.
As demonstrated earlier, if gravity is assumed to be able to mediate entanglement, the above system can generate entanglement between states $A$ and $B$ due to their gravitational interaction.
Placing a shield in the center to shield potential interactions between the masses because of various electromagnetic interactions such as Coulomb- or Casimir-forces should in theory not influence the entanglement generation. 
However, Casimir interactions between the shield and the spheres are still present at small variations as discussed in \cref{cha:casimir-effect}.
It is straight forward to convince yourself, that these interactions can only induce local phases to each of the delocalized states depending only on the shield-mass distance of set state. 
These interactions can therefore - assuming a static shield e.g. at zero temperature - not induce any entanglement at all between different masses.

For the complete picture however, one has to consider, how the entanglement is going to be measured in a real experiment.
Most likely, the density matrix of the evolved system after some time $t$ is measured by \emph{full state tomography} !!!CITATIONS!!!. Other proposals aim to measure the entanglement using a suitable witness \cite{Bose_2017,Chevalier_2020}, but the creation of such a witness requires knowledge of the specifics on the experimental realization of massive delocalized states.
In this thesis, I will focus on the most general and universally applicable case of measuring the complete density matrix of the system and checking for entanglement.
The density matrix of a 2 qubit system however consists of 16 different entries (only 9 independent \footnote{Using the known characteristics of the density matrix like hermiticity $\rho^\dagger = \rho$ and $\tr \rho = 1$, it is possible to reconstruct $\rho$ by measuring only 9 specific entries.}) and thus at least 9 measurements of the system are required for a full state tomography. However because of the statistical nature of quantum mechanics and the fact, that each measurement destroys the previously evolved state, in practice a lot more measurements are required.
One cannot be sure that the masses are placed in the exact same position each run or if the superposition is extended in precisely the same orientation over and over again. Even if it was somehow possible, to place the particles at the \textit{exact} same position each measurement, thermal vibrations of the shield, the masses or the setup still induce small deviations in the placement.
Thus, each run the entanglement of a slightly different system is measured creating effectively a result which looks like a mixed state
\begin{equation}
  \rho = \int_{-\infty}^{\infty} \dd \theta \frac{1}{\sqrt{2\pi}\Delta \theta} e^{-\theta^2/2(\Delta \theta)^2} \ketbra{\psi_\theta} .
\end{equation}
Here, $\ket{\psi_\theta}$ is the pure state of a single measurement dependent on the angular variations $\theta$. These variations are assumed to be gaussian distributed around mean $0$ and standard deviation $\Delta \theta_{A,B}$.
The same argument can be made for variations in the distance, which are distributed with $\Delta L_{A,B}$ (compare with \cref{fig:4:complete-setup}).
In some cases, like for example if the plate position moves slightly, the variations are correlated and $\Delta L_A = \Delta L_B$. But in the most general case, they are assumed to be independent and follow their own probability distribution.

\begin{figure}[!htbp]
  \centering
  \includegraphics[width=\textwidth]{./../figures/theta-variance/EN-delta-theta.pdf}
  \caption{Entanglement quantified by the logarithmic negativity (eq. \eqref{eq:2:logarithmic-negativity}) dependent on the angular variation $\Delta\theta$. The entanglement is shown at different times, where $t_\mathrm{max}$ is the time of maximal entanglement calculable by eq. \eqref{eq:4:t-max}. Additionally, a few selected exact numerical results are shown to align precisely with the approximated version.}
  \label{fig:4:EN-delta-theta}
\end{figure}


% \section{Stability in different orientations}\label{sec:4:orientation}
As could be already seen in \cref{fig:2:entanglement-dynamics} in \cref{cha:first-look}, the entanglement dynamics and especially the time $t_\mathrm{max}$ of maximum entanglement highly depend on the orientation.
There, it was shown already, that the for the parallel configuration the maximum entanglement is reached twice as slow as in the orthogonal orientation.
This result can be more generalized for an arbitrary orientation quantified by $\alpha$ and $\beta$ from \cref{fig:4:complete-setup}.


\begin{equation}
  E_N = \log_2\left(1 + \abs{\sin\Delta \phi}\right)
\end{equation}
where $\Delta \phi$ is now dependent on the orientation and is given by (for $\Delta x \ll L$)
\begin{equation}
  \Delta \phi = \frac{G M_A M_B t}{\hbar}\frac{\Delta x_A \Delta x_B}{8 L^3} \left[\sin\alpha\sin\beta - \frac{1}{2}\cos\alpha\cos\beta\right] .
\end{equation}
The maximum entanglement is again reached after $\Delta\phi = \pi/2$ and thus is also dependent on the orientation
\begin{equation}\label{eq:4:t-max}
  t_\mathrm{max} = \frac{4\pi L^3\hbar}{GM_AM_B\Delta x_A \Delta x_B} \abs{\sin\alpha\sin\beta - \frac{1}{2}\cos\alpha\cos\beta}^{-1}.
\end{equation}
The resulting times are shown in \cref{fig:4:t-max-orientation} and align with earlier results for the special cases of the orthogonal and parallel configuration.
\begin{figure}[!htbp]
  \centering
  \includegraphics[width=\textwidth]{./../figures/ideal-entanglement/EN-orientation.pdf}
  \caption{Time of maximum entanglement $t_\mathrm{max}$ relative to $t_\mathrm{max,orthogonal}$ of the orthogonal configuration for different orientations. Some orientations like $\alpha=\pi/2$ and $\beta=0$ cannot induce any entanglement at all because of symmetry considerations.}
  \label{fig:4:t-max-orientation}
\end{figure}
The global minima of $t_\mathrm{max}$ is given in the orthogonal configuration meaning that of all possible configurations, this one induces entanglement the fastest. Considering the discussions in the end of \cref{cha:first-look}, this is not very surprising.
Much more interesting are the apparent singularities which arise for 
\begin{equation}
  \sin\alpha\sin\beta = \frac{1}{2}\cos\alpha\cos\beta .
\end{equation}
For $\beta=0$, the singularity in $t_\mathrm{max}$ at $\alpha=\pi/2$ is not surprising. At this configuration, the distances $\ket{\psi_A^1} \leftrightarrow \ket{\psi_B^{1,2}}$ are identical as well as $\ket{\psi_A^2} \leftrightarrow \ket{\psi_B^{1,2}}$. 
For the case $\alpha = \beta$, the two singularities are precisely given at
\begin{equation}
  \alpha = \beta = 2\arctan(\sqrt{3}\pm\sqrt{2}) \approx 90\deg \pm 54.74\deg.
\end{equation}
There does not exist a clear geometric explanation why no entanglement is generated in this configuration, however, all distances between the superpositions for a harmonic mean as visualized in \cref{fig:4:harmonic-mean}.
\begin{figure}[!htbp]
  \centering
  \def\svgwidth{\textwidth}
  \input{./../figures/harmonic-mean.pdf_tex}
  \caption{\textbf{left:} Arrangement in the orientation $\alpha=\beta=2\arctan(\sqrt{3}-\sqrt{2})$. All distances exactly lie in the \textit{harmonic mean} (for $\Delta x \ll L$). \textbf{right:} Geometric visualization of the harmonic mean.}
  \label{fig:4:harmonic-mean}
  % COLORFUL:: $\textcolor[HTML]{aa0000}{\blacksquare} = \displaystyle\frac{2}{\frac{1}{\textcolor[HTML]{0044aa}{\blacksquare}}+\frac{1}{\textcolor[HTML]{447821}{\blacksquare}}}$
\end{figure}





\begin{figure}[!htbp]
  \centering
  \includegraphics[width=0.95\textwidth]{./../figures/theta-variance/theta-max-orientation-complete.pdf}
  \caption{Maximum possible allowed angular variation $\Delta\theta_\mathrm{max}$ for different orientations. All data-points where calculated at the time of maximum entanglement shown in \cref{fig:4:t-max-orientation}. The \emph{orthogonal configuration} is very stable against angular disturbances. At $\alpha=\beta=\pi/2$, only exact numerical results show a finite value. The singularities on the left figure arise from the fact, that these configurations need infinite time to entangle as already seen in \cref{fig:4:t-max-orientation}.}
  \label{fig:4:theta-max-orientation}
\end{figure}

\begin{figure}[!htbp]
  \centering
  \includegraphics[width=0.95\textwidth]{./../figures/L-variance/L-max-orientation-complete.pdf}
  \caption{Maximum possible allowed distance variation $\Delta L_\mathrm{max}$ for different orientations. This figure is similar to \cref{fig:4:theta-max-orientation}, only for distance variations. On the contrary to angular variations, the \emph{parallel configuration} is infinitely stable against changes in the distance between the shield and the cat-state.}
  \label{fig:4:L-max-orientation}
\end{figure}



\begin{figure}[!htbp]
  \centering
  \includegraphics[width=\textwidth]{./../figures/optimize/optimized-orientation.pdf}
  \caption{Optimal orientation for arbitrary variations in the angle $\Delta\theta$ and the distance $\Delta L$. The optimum was calculated for different models of the Casimir-interaction (PFA eq. \eqref{eq:3:PFA-sphere-plate} and LSL eq. \eqref{eq:3:casimir-sphere-plate}).If angular variations dominate, the orthogonal configuration is best, whereas for large distance variations, a parallel orientation is advised.}
  \label{fig:4:optimal-orientation}
\end{figure}

\section{Discussion on the optimal setup}\label{sec:4:discussion}

- No global maximum

- dependent on experimental stuff

- Masses and distances and so on must be bounded by 10xECasimir < EGravity

How can one find the best parameters depending on the experimental requirements?
\begin{enumerate}
  \item Specify maximum coherence time (how long the experiment can take)
  \item This fixes $L^3/(M_A M_B \Delta x_A \Delta x_B)$
  \item Look up, what maximum $\Delta \theta$ and $\Delta L$ you can get at this time (Dependent on the amount of entanglement you want to have at the end)
  \item Take the minimum possible $L/R$ in account (dependent on trap stiffness and so on) Ideally, large enough to reduce casimir interactions
\end{enumerate}


-------------------------------------- 

\begin{enumerate}
  \item Given a target measuring time $t_\mathrm{target}$, this fixes the ratio
  \begin{equation}\label{eq:4:fixed-ratio}
    \frac{L^3}{M_A M_B \Delta x_A \Delta x_B} = 5.036\times 10^{22}\si{m/kg^2s} \cdot t_\mathrm{target} = \mathrm{const.}
  \end{equation}
  Most likely, the superposition size $\Delta x_{A,B}$ is also not changeable and one has to be satisfied by what can be achieved experimentally. The delocalized states of mass $M$ has to interact gravitationally with each other in a laboratory setting for at least the duration of the measuring process $t_\mathrm{target}$. These considerations limits the coherence time and usually, low times in the order of milliseconds to seconds are favorable.
  \item In principle, one does not require to measure at the time of maximum entanglement where $E_N = 1$. If a lower quantity of entanglement is enough for a set experimental goal, I recommend, measuring at a lower time than $t_\mathrm{target} = \tau t_\mathrm{max}$ ($\tau < 1$) !!!!FIGURE!!!!. This on the other hand increases the fixed ratio eq. \eqref{eq:4:fixed-ratio} by a factor of $1/\tau$.
  Therefore, it is possible to use smaller superposition sizes $\Delta x$, lighter masses or increase the distance $L$ which decreases Casimir interactions.
  \item Most likely, the minium angular 
\end{enumerate}