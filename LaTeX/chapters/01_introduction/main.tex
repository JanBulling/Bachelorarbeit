\chapter{Introduction}\label{cha:introduction}

Understanding gravity, nature's weakest fundamental force, has been a persistent challenge ever since Newton formulated the laws of gravitational force in 1687 \cite{Newton_1687}.
Newton's equations were groundbreaking, providing unprecedented accuracy in predicting planetary motion and establishing mathematics as the universal language of physical description.
In 1774, Maskelyne conducted the first measurement of gravitational attraction of an object other than planets, solidifying the concept of the interaction of masses across multiple magnitudes \cite{Maskelyne_1775,Davies_1985}. 
Cavendish measured the gravitational constant $G$ using a torsion pendulum in 1798, providing the first quantitative determination of the weak gravitational coupling.
Einstein revolutionized our understanding of gravity with his theory of general relativity, offering a geometric interpretation of spacetime. Yet, the fundamental nature of gravity remains elusive. Whether gravity is purely classical or fundamentally quantum is one of the most pressing and debated questions in modern physics and many different theories for unifying general relativity and quantum mechanics immerge \cite{Becker_2007, Ashtekar_1986, Oppenheim_2023}.
The extraordinary success of quantum field theory for electromagnetic, weak, and strong interactions suggests that gravity may also have a quantum foundation.

Recent proposals \cite{Bose_2017,Marletto_2017} aim to test the non-classicality of gravity by measuring gravitationally induced entanglement between two masses, showing the necessity of a new gravitational model.
In \cref{cha:first-look} of this thesis, the arguments and assumptions behind the proposals are discussed and the general experimental setup is explained. A useful framework of later employed methods is outlined.
The necessity of a Faraday-Shield in the experiments is explained as well as potential occurring issues due to Casimir interactions between the newly introduced shield and the particles.
\Cref{cha:casimir-effect} introduces the Casimir force and establishes approximation models for short- and long-range interactions. The effect of rough and uneven surfaces is studied.

The decoherence effect and the consequent loss of entanglement caused by stochastic variations in the experimental initialization are analyzed in-depth in \cref{cha:entanglement-generation}. Potential modifications in the setup and the proposed parameter-space as well as the ability to trap and levitate the particles close to the shield is discussed. 
\Cref{cha:the-shield} estimates the required size and thickness of the Faraday shield for block unwanted interactions between the masses sufficiently. Additionally, effects of decoherence and on entanglement generation due to thermal vibrations of the shield are calculated and analyzed.