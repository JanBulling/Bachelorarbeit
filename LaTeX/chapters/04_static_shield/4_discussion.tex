\section{Discussions}\label{sec:4:discussion}
Looking at the preceding results, it is clear that the planned experiment represents a significant engineering challenge. The decoherence due to the Casimir interactions between the particles and the Faraday shield requires a accuracy in the placement of the particles in the order of $\Delta L = 10^{-10}\si{m}$ and $\Delta \theta = 10^{-9}\si{rad}$.
Achieving these accuracies appears to be very challenging and it will be necessary to adjust the originally proposed parameters in \cref{fig:4:complete-setup}. The separation distance $L$ as well as the orientation are particularly easy to change.
As discussed earlier, the parallel orientation might be the only viable option, as this position is almost infinitely stable against variations in the distance. The orthogonal orientation would require placement accuracies in the order of single atoms (compare to \cref{fig:4:L-crit-orientation}).
The separation $L$ can be freely chosen and a larger separation reduces the effect of placement variations as seen in \cref{fig:4:theta-crit-L} but substantially increases the required coherence time $t_\mathrm{max} \propto L^3$.
It could also be argued that at a distance of $L \geq 200\si{\mu m} = 20 R$ (compare to \cref{sec:2:experimental-problems}), the Faraday shield would no longer be required because the Casimir forces between the particles are 10 times weaker than the gravitational interactions due to their rapid decrease at large distances $\sim 10$.
However, the loss of entanglement due to angular and distance variations in placement is not solely due to Casimir forces between the particle and the shield, so that a complete removal of the shield does not fully eliminate the requiem placement accuracy.
The slightly varying gravitational interaction alone can induce enough decoherence on its own to destroy entanglement. 
The critical variations for a gravitational interaction alone in the parallel configuration after $t_\mathrm{max}$ are given by $\Delta \theta_\mathrm{crit,\,ideal} = 1.1 \times 10^{-3}\si{rad}$ and $\Delta L_\mathrm{crit,\,ideal} = 7\times 10^{-4}\si{m}$, which should not pose an engineering problem.

Changing the other parameters such as the particles size or superposition size might not be possible. Substantially changing both would increase the difficulty in groundstate cooling and quantum control unforeseeable.
For all these considerations, the trapping does not play a role as this should be possible with a suitable magnetic or optical trap for almost any possible configuration of the setup parameters (compare to the results from \cref{sec:4:trapping}).

One of the objectives of this thesis is, to determine whether it is possible to bring the particles closer together through the presence of the Faraday shield in order to increase gravitational entanglement and reduce the required coherence times.
To achieve this, the previous results from this chapter can be used find the optimal parameters of the experimental setup.
The goal of the optimization process can be expressed as the following:
One wants to get \textit{as much entanglement as possible} in the \textit{shortest time possible} with the \textit{largest possible variations} in the placement while still considering the limitations in the mass as well as in the superposition size.

In full generality, it is not possible to find a local optimum for choosing the parameters. This is because (if the mass $M$ and the superposition size $\Delta x$ is fixed) the coherence time - which should be minimized - scales with $t \propto L^3$ by eq. \eqref{eq:4:t-max} and the critical angular variation - which should be as large as possible - scale with $\Delta \theta_\mathrm{crit} \propto (L-R)^3/L^3$ for small separations (eq. \eqref{eq:4:delta-theta-scaling-L}) or $\Delta \theta_\mathrm{crit} \propto L^2$ for $L \gg R$. Both of these optimization criteria cannot be fulfilled simultaneously as long as no constraints are given.
Given however a coherence time $t_\mathrm{target}$ and/or the minimum possible placement accuracy, it is possible to determine the required sphere-plate separation $L$ as well as the amount of entanglement, one can maximally expect using the following steps:
\begin{enumerate}
  \item Lets assume that the size of the particle $R$ and consequently the mass $M=4/3 \pi R^3 \rho_\mathrm{Silica}$ as well as the superposition size $\Delta x$ are fixed. An increase in either of them would have a positive effect of the optimization goal stated above, as the time $t_\mathrm{max}$ decreases and the stability against placement variations increases simultaneously.
  \item The following ratio given by eq. \eqref{eq:4:t-max} and by Ref. \cite{Aspelmeyer_2024} to
  \begin{equation}
    \frac{M^2 (\Delta x)^2}{L^3}t_\mathrm{max} = \frac{8 \pi \hbar}{G} = \mathrm{const.} 
  \end{equation} 
  in the parallel orientation is fixed. For orthogonal configurations, this constant would reduce by a factor of $1/2$. 
  \item In general it is possible to measure at a earlier time $t_\mathrm{target} = \tau t_\mathrm{max}$ (i.e. the coherence time) with $\tau \leq 1$, where less total entanglement has been build up but in general a grater stability against placement variations can be achieved (see \cref{fig:4:time-delta-theta}). Putting all assumptions together, the ratio
  \begin{equation}\label{eq:4:fixed-ratio}
    \frac{t_\mathrm{target}}{\tau L^3} = \frac{8\pi \hbar}{G} \frac{1}{M^2 (\Delta x)^2}
  \end{equation}
  is constant.
  \item In the parallel orientation, the distance variations don't matter as the system is infinitely stable against variations in the particle-shield separation. The critical angular variation however scales like $\Delta \theta_\mathrm{crit} \sim (L-R)^3/L^3$ for small distances and like $\Delta \theta_\mathrm{crit} \sim L^2$ shown in \cref{fig:4:theta-crit-L}. it is therefore possible to determine the minimum separation $L_\mathrm{min} > R$ for a given placement accuracy.
  \item Using the required separation, one can calculate $\tau \in (0, 1]$ using eq. \eqref{eq:4:fixed-ratio} and look up the maximal possible entanglement in the top right of \cref{fig:4:time-delta-theta} after an evolution time $\tau t_\mathrm{max}$.
\end{enumerate}
As an example, the radius is fixed as $R=10\si{\mu m}$ and the superposition size is $\Delta x = 100\si{nm}$. Such a particle can be placed with an accuracy of $\Delta \theta = 10^{-7} \si{rad}$ and coherence times of $1\si{s}$ are reachable. 
Using the steps outlined above, the resulting particle-shield separation is around $L\approx 15R$ and the maximal amount of measurable entanglement is given by $E_N \approx 9.2\times 10^{-3}$.
For more entanglement, either a heavier particle, a larger superposition size, a higher placement accuracy or larger coherence times are required. 
It is therefore actually possible, to bring the particles closer together than without the Faraday shield and still measure entanglement. One is only limited by the placement accuracy and repeatability.