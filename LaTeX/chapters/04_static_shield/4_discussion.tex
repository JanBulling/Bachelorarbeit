\section{Discussions}\label{sec:4:discussion}
Looking at the preceding results, it is clear that the planed experiment is challenging. The decoherence due to the presence of Casimir interactions between the particles and the Faraday shield requires a exactness in variations for the placement of the particles in the order of $10^{-10}\si{m}$ and $10^{-9}\si{rad}$.
These accuracies seem very challenging and one therefore has to opt for slightly larger masses or larger separation distances, increasing either the difficulty of cooling and quantum control, or increasing the required coherence time $t_\mathrm{max} \propto L^3$.
Additionally, one could argue that at a separation of ..... (!!!!) (compare \cref{sec:2:experimental-problems}) the casimir interaction between the particles is small enough to not be problematic and the Casimir shield might not be required after all. 

The loss of entanglement due to angular and distance variations in the placement are not solely due to the Casimir interaction between particle and shield. 
The slightly varying gravitational interaction alone can induce enough decoherence on its own to destroy entanglement. 
The critical variations for a solely gravitational interaction in the parallel configuration after $t_\mathrm{max}$ are given by $\Delta \theta_\mathrm{crit,\,ideal} = 1.1 \times 10^{-3}\si{rad}$ and $\Delta L_\mathrm{crit,\,ideal} = 7\times 10^{-4}\si{m}$, which should be far better than the actual experimental placement accuracy.

- Without casimir (so only gravity with variations)

- No global maximum

- dependent on experimental stuff

- Masses and distances and so on must be bounded by 10xECasimir < EGravity

How can one find the best parameters depending on the experimental requirements?
\begin{enumerate}
  \item Specify maximum coherence time (how long the experiment can take)
  \item This fixes $L^3/(M_A M_B \Delta x_A \Delta x_B)$
  \item Look up, what maximum $\Delta \theta$ and $\Delta L$ you can get at this time (Dependent on the amount of entanglement you want to have at the end)
  \item Take the minimum possible $L/R$ in account (dependent on trap stiffness and so on) Ideally, large enough to reduce casimir interactions
\end{enumerate}


-------------------------------------- 

\begin{enumerate}
  \item Given a target measuring time $t_\mathrm{target}$, this fixes the ratio
  \begin{equation}\label{eq:4:fixed-ratio}
    \frac{L^3}{M_A M_B \Delta x_A \Delta x_B} = 5.036\times 10^{22}\si{m/kg^2s} \cdot t_\mathrm{target} = \mathrm{const.}
  \end{equation}
  Most likely, the superposition size $\Delta x_{A,B}$ is also not changeable and one has to be satisfied by what can be achieved experimentally. The delocalized states of mass $M$ has to interact gravitationally with each other in a laboratory setting for at least the duration of the measuring process $t_\mathrm{target}$. These considerations limits the coherence time and usually, low times in the order of milliseconds to seconds are favorable.
  \item In principle, one does not require to measure at the time of maximum entanglement where $E_N = 1$. If a lower quantity of entanglement is enough for a set experimental goal, I recommend, measuring at a lower time than $t_\mathrm{target} = \tau t_\mathrm{max}$ ($\tau < 1$) !!!!FIGURE!!!!. This on the other hand increases the fixed ratio eq. \eqref{eq:4:fixed-ratio} by a factor of $1/\tau$.
  Therefore, it is possible to use smaller superposition sizes $\Delta x$, lighter masses or increase the distance $L$ which decreases Casimir interactions.
  \item Most likely, the minium angular 
\end{enumerate}