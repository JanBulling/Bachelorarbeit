\chapter{The particle in front of a static shield}\label{cha:entanglement-generation}

The generalized setup of the system described in \cref{cha:first-look} with the addition of a conducting Faraday shield is shown in \cref{fig:4:complete-setup}. As before, the particles $A$ and $B$ are delocalized in spatial cat-states with superposition sizes $\Delta x_A$ and $\Delta x_B$ respectively.
\begin{figure}[!htbp]
  \centering
  \def\svgwidth{\textwidth}
  \input{./../figures/problem.pdf_tex}
  \caption{Schematic experimental setup for detecting gravitationally induced entanglement between particles $A$ and $B$ with radius $R$, separated by a distance $2L + L_A + L_B$. Their orientations are defined by angles $\alpha$ and $\beta$, with small variations $\theta_{A(B)}$. All variations are assumed to be normally distributed around zero with standard deviation $\Delta L_{A(B)}$ and $\Delta \theta_{A(B)}$. The particles are delocalized in cat-states with size $\Delta x_{A(B)}$ between states $\ket{\psi_{A(B)}^1}$ and $\ket{\psi_{A(B)}^2}$. A conducting Faraday shield of thickness $d$ is centrally placed.}
  \label{fig:4:complete-setup}
\end{figure}
They are positioned at a distance $L$ from the Faraday-shield with the cat-state superpositions oriented at angles $\alpha,\beta\in[0,\pi)$ relative to it.
We define the configuration with $\alpha=\beta=0$ as the \q{parallel orientation} and $\alpha = \beta = \pi/2$ as the \q{orthogonal orientation}.

If gravity is assumed to be able to mediate entanglement, particles $A$ and $B$ can get entangled.
Placing a Faraday shield in the center between the masses should not substantially influence the gravitational entanglement generation.
However, Casimir interactions between the newly placed shield and particles must be accounted for, especially at small separations.
These interactions contribute only phase shifts to individual cat-states, but do not couple the particles and thus cannot generate additional entanglement - assuming a static shield e.g. at zero temperature.

For a complete picture, experimental challenges and limitations have to be considered. 
The question remains open on how to verify gravitational entanglement between the two particles.
One possibility is to find a suitable entanglement witness \cite{Bose_2017,Chevalier_2020}, which requires knowledge about the specific implementation of the experiment.
In this thesis, it will be assumed that the full density
matrix of the state is accessible and the entanglement will be quantified with the logarithmic negativity introduced in \cref{sec:2:entanglement-measures}.
Furthermore, the simplifying assumption is made, that each cat-state is confined in a truncated two-dimensional Hilbert space, as only the dynamic relative phase differences between two such states are of interest.
Consequently, the density matrix has 16 entries, of which only 9 are independent, making reconstruction possible using known matrix properties (e.g., hermiticity $\rho^\dagger = \rho$ and $\tr \rho = 1$).
Accurate tomography requires numerous measurements due to quantum probabilistic behavior. 
From run to run, the placement angle $\alpha + \theta_A$ or $\beta + \theta_B$ as well as the separation distances $L + L_A$ or $L + L_B$ in the setup will vary, resulting in a mixed initial state $\rho_0$.
Other experimental noise sources, such as variations in the measurement timing, have been considered previously in Ref. \cite{Nguyen_2020}.
Even if it was possible to place the particles at the exact same position for each measurement, thermal vibrations of the shield induce noisy variations in the particle-shield separation, analyzed in \cref{cha:the-shield}.
If the random variable $X \in \{\theta_{A}, \theta_{B}, L_{A}, L_{B}\}$ is subject to statistical variations, the resulting mixed state can be described by
\begin{equation}
  \rho = \int\limits_{-\infty}^{\infty} \dd X \frac{1}{\sqrt{2\pi}\Delta X} e^{-X^2/2(\Delta X)^2} \ketbra{\psi_X} ,
\end{equation}
where $\ket{\psi_X}$ is the evolved pure state of a single measurement dependent on the random variable $X$.
These variations are assumed to be normally distributed with mean $\mean{X} = 0$ and standard deviation $\Delta X$ on the basis of the central limit theorem \cite[p. 1195]{Riley_2018}.
While correlations between $\theta_{A(B)}$ and $L_{A(B)}$ are possible, for now they are assumed to be uncorrelated as a worst-case estimation.
In \cref{cha:the-shield} during the analysis of thermal vibrations of the Faraday shield, cases involving correlated variables due to thermal shield vibrations are analyzed.



\subsection*{Convergence for a finite number of measurements}
Experimentally, it would be very interesting to know how fast the averaged density matrix $\bar{\rho}$ after a finite number of $N$ measurements converges to the idealized asymptotic mean $\mean{\rho}$ given by eq. \eqref{eq:4:average-density}.
After $N$ measurements, the sample average is given by
\begin{equation}
  \bar{\rho} = \frac{1}{N} \sum_{k=1}^{N} \rho(X_k)
\end{equation}
where $\rho(X)$ depends on the random variable $X \in \{\theta_{A(B)}, L_{A(B)}\}$ and $X_k$ is the $k$-th sample drawn from the normal distribution $\mathcal{N}(0, (\Delta X)^2)$ \footnote{Here it is not strictly required that $X_k$ are normally distributed. As long as they are i.i.d. random variables, any distribution is sufficient for the following argumentation \cite[p. 1195]{Riley_2018}.}.
As $N \rightarrow \infty$, the law of large numbers and in particular the central limit theorem (CLT) ensures that $\bar{\rho} \rightarrow \mean{\rho}$ \cite[p. 1195]{Riley_2018}.
According to the CLT, the sample average $\bar{\rho}(X)$ fluctuates around $\mean{\rho}$ with a standard deviation given by the Berry-Esseen theorem for independent and identically distributed random variables $X_k$ by $\sigma \sim N^{-1/2}$ \cite{Berry_1941}.
Thus, if the placements of the particles in each measurement are completely independent from each other, the rate of convergence to the ideal mean $\mean{\rho}$ is governed by $\mathcal{O}(1/\sqrt{N})$.

It is however very likely that measurements are mostly performed consecutively in the same trap so that the particle placements in successive measurements are correlated.
The correlations $\mathrm{Cov}[\rho(X_i), \rho(X_j)] = c_{\abs{i-j}}$ between the $i$-th and $j$-th measurement should therefore decrease with increasing $\abs{i-j}$.
The variance of $\bar{\rho}$ is now dependent of these correlations in the form \cite[p. 1227]{Riley_2018}
\begin{equation}\label{eq:4:correlation-variance}
  \mathrm{Var}[\bar{\rho}] = \frac{1}{N^2} \sum_{i,j=1}^{N} \mathrm{Cov}[\rho(X_i), \rho(X_j)] = \frac{1}{N}\mathrm{Var}[\rho] + \frac{2}{N^2}\sum_{n=1}^{N - 1}(N - n) c_n
\end{equation}
where $\mathrm{Cov}[\rho, \rho] = \mathrm{Var}[\rho]$ was used for the variance of the mean density matrix $\mean{\rho}$.
For correlations $c_n \sim n^{-\alpha}$ ($\alpha < 1$) the sum in eq. \eqref{eq:4:correlation-variance} can by asymptotically calculated by the Euler-Maclaurin formula and scales like
\begin{equation}
  \sum_{n=1}^{N - 1}(N - n)n^{-\alpha} \xlongrightarrow{N\rightarrow\infty} \int_1^{N}\dd n \, (N-n)n^{-\alpha} \sim N^{2-\alpha}
\end{equation}
which results in $\mathrm{Var}[\bar{\rho}] \sim N^{-\alpha}$. In the asymptotic limit the standard deviation of the sample average $\sqrt{\mathrm{Var}[\bar{\rho}]}$ and thus the convergence rate to the mean $\mean{\rho}$ scales with $\mathcal{O}(1/\sqrt{N^\alpha})$.
This convergence is arbitrary slow for small $\alpha$ (if the setup does not change a lot between individual measurements) and thus the calculations in the next sections are just a worst-case estimation of the actual experimental results.
If a weaker correlation in the form of $c_n \sim e^{-\alpha n}$ is assumed, the convergence rate is again asymptotically governed by $\mathcal{O}(1/\sqrt{N})$.



\section{Entanglement generation}\label{sec:4:entanglement-generation}
If the full state $\rho$ is measured repeatedly but for each measurement the initial setup was slightly different, this effectively is an averaging process over all the variations in the setup.
As mentioned earlier, this averaging mixes the state and for large variations $\Delta \theta$ and $\Delta L$, this process can destroy entanglement.
To see this, we calculate the averaged state $\mean{\rho}$ as
\begin{equation}\label{eq:4:average-density}
  \mean{\rho} = \int_{\infty}^{\infty} \dd \theta_A p(\theta_A) \int_{\infty}^{\infty} \dd \theta_B p(\theta_B) \int_{\infty}^{\infty} \dd L_A p(L_A) \int_{\infty}^{\infty} \dd L_B p(L_B) \ \rho(\theta_A, \theta_B, L_A, L_B)
\end{equation} 
where $p(\,\cdot\,)$ is the gaussian probability distribution. For both, $\theta$ and $L$, this is distributed around mean $0$ and with standard deviation $\Delta \theta$ and $\Delta L$. $\rho(\theta_A, \theta_B, L_A, L_B)$ is the state of a single measurement, dependent on the parameters $\theta_i$ and $L_i$ of the setup.
This state is very similar as before in \cref{cha:first-look} but with the additional effect of the casimir interactions taken into account.
The initial state $\ket{\psi_0}$ at $t=0$ is given by eq. \eqref{eq:2:initial-state}.
During the time evolution, not only the mutual gravitational interactions between the masses but also the Casimir interactions between the shield and the states has to be taken into account. A state $\ket{\psi^i_{A(B)}}$ ($i = 1, 2$) accumulated the phase $\phi^i_{A(B),\,\mathrm{Cas}}(t)$ during time evolution.
This phase is given by
\begin{equation}
  \phi^i_{A(B),\,\mathrm{Cas}}(t) = \frac{t}{\hbar}
  \begin{cases}
     \frac{3 \hbar c}{8 \pi} \left(\frac{\varepsilon_r - 1}{\varepsilon_r + 2}\right) \frac{R^3}{(L^i_{A(B)})^4} & \text{for large separations (LSL)} \\
    \frac{\hbar c \pi^3}{720} \varphi(\varepsilon_r) \left(\frac{\varepsilon_r - 1}{\varepsilon_r + 1}\right) \frac{R}{(\mathscr{L}^i_{A(B)})^2} & \text{for small separations (PFA)}
  \end{cases}
\end{equation}
where a distinction between the different Casimir-models discussed in \cref{cha:casimir-effect} has been made.
The plate-sphere separations $L^i_{A(B)}$ and $\mathscr{L}^i_{A(B)} = L^i_{A(B)}-R$ are of course dependent on the placement of the particles.
In full generality, they are given by
\begin{equation}
  L^i_{A(B)} = L + L_{A(B)} - \frac{d}{2} \pm \frac{\Delta x_{A(B)}}{2} \sin(\xi + \theta_{A(B)})
\end{equation}
where $\pm$ depends on the considered state and $\xi = \alpha, \beta$ was used as an abbreviation.
The mutual gravitational interaction for states $\ket{\psi^i_A}\otimes\ket{\psi^j_B}$ is given similar to before by the accumulated phase
\begin{equation}
  \phi^{ij}_\mathrm{Grav}(t) = \frac{t}{\hbar} \frac{G M_A M_B}{L^{ij}} .
\end{equation}
The separation between the spheres $L^{ij}$ ($i,j = 1,2$) in full generality is given by
\begin{multline}
  L^{ij} = \sqrt{\left(2L + L_A + L_B \pm \frac{\Delta x_A}{2}\sin(\alpha + \theta_A) \mp \frac{\Delta x_B}{2}\sin(\beta + \theta_B)\right)^2 +} \\ \overline{\left(\frac{\Delta x_A}{2}\cos(\alpha + \theta_A) \pm \frac{\Delta x_B}{2}\cos(\beta + \theta_B)\right)^2}
\end{multline}
Expanding both phases in first order of $\Delta x_{A(B)}$ as already done before and in $\theta_{A(B)} \ll 1$, $L_{A(B)} \ll 1$ (which is feasible as both are very small as seen later), the averaging in eq. \eqref{eq:4:average-density} can be performed analytically (see \cref{apx:average-density}).
It turns out, that with $\Delta \theta_A = \Delta \theta_B \equiv \Delta\theta$ and $\Delta L_A = \Delta L_B \equiv \Delta L$ all off-diagonal elements of the averaged state $\mean{\rho}$ are given in the form
\begin{equation}\label{eq:4:average-density-element}
  \mean{\rho_{kl}} = \frac{1}{4} e^{i \Delta \phi_{kl}(t)} \exp{-\frac{(\Delta\theta)^2}{2} (\Delta\phi_{kl,\,\theta})^2 t^2} \exp{-\frac{(\Delta L)^2}{2} (\Delta\phi_{kl,\,L})^2 t^2}
\end{equation}
where all $\Delta \phi$-terms are replacements for rather lengthy linearized phase expressions which depend on the separation $L$, the orientation $\alpha, \beta$, the masses $M_{A(B)}$ and the delocalization size $\Delta x_{A(B)}$.
It becomes evident, that for large times or for large variations in the placement, these off-diagonal elements tend to zero. 
For $t\rightarrow \infty$ or for $\Delta \theta, \Delta L \rightarrow \infty$, this state represents the maximally mixed state with $\tr\rho^2 = 1/4$, which obviously is not entangled.
For large variations in the placement, one therefore expects a loss of entanglement.

The resulting logarithmic negativity of the averaged state $E_N(\mean{\rho})$ was computed numerically for different values of $\Delta \theta$ and $\Delta L$ and is shown in \cref{fig:4:EN-delta-theta}.
\begin{figure}[!htb]
  \centering
  \includegraphics[width=\textwidth]{./../figures/theta-variance/EN-deltaTheta-deltaL.pdf}
  \caption{Entanglement quantified by the logarithmic negativity (eq. \eqref{eq:2:logarithmic-negativity}) dependent on the angular variation $\Delta\theta$ and the distance variation $\Delta L$ in the parallel configuration. The entanglement is shown at different times, where $t_\mathrm{max} \approx 258\si{ms}$ is the time of maximal entanglement from eq. \eqref{eq:2:t-max-parallel}. At a critical point $\Delta \theta_\mathrm{crit}$ and $\Delta L_\mathrm{crit}$ all entanglement is lost.}
  \label{fig:4:EN-delta-theta}
\end{figure}
For this figure, the parallel configuration with $\alpha = \beta = 0$ was used with $t_\mathrm{max}$ given by eq. \eqref{eq:2:t-max-parallel} as well as a radius $R=1\times 10^{-5}\si{m}$, a corresponding mass of $M_A = M_B = 4/3\, \pi R^3 \rho_\mathrm{Silica} \approx 1.1\times 10^{-11}\si{kg}$, a separation of $L=2R$ and a superposition size of $\Delta x_A = \Delta x_B = 100\si{nm}$.
In the rest of the thesis, if not otherwise specified, these parameters are used as a default.
They are chosen in these orders of magnitude, because they result in a feasible low experiment-time $t_\mathrm{max}\approx 258\si{ms}$ and are in the region of what is soon\footnote{\q{Soon} in this context means still a long time, but it could be reachable within the next century.} possible \cite{Aspelmeyer_2024}.
It is important however to stress out, that all these parameters are orders of magnitude away of from what is experimentally reachable today.
The largest mass that was studied in matter-wave interferometry is in the order of $4\times 10^{-23}\si{kg}$ \cite{Fein_2019} with an superposition size of $\Delta x > 500\si{nm}$.
For solid state mechanical systems quantum control up to masses of order $10^{-13}\si{kg}$ \cite{OConnell_2010}, $10^{-11}\si{kg}$ \cite{Lee_2011} and $10^{-11}\si{kg}$ \cite{Bild_2023} but with very short coherence times $\lesssim 1\si{\mu s}$ have been demonstrated.
On the contrary, the smallest mass with measured gravity is around $92 \si{mg}$ \cite{Westphal_2021}.
Levitated particles combine the best of both world with quantum control of large and heavy trapped solid objects as well as long coherence times up to the order of seconds \cite{Aspelmeyer_2024}.

The entanglement of the system shown in \cref{fig:4:EN-delta-theta} behaves as expected. At some critical point $\Delta \theta_\mathrm{crit}$ and $\Delta L_\mathrm{crit}$, the entanglement is completely lost.
For the used parameters and in the parallel orientation, this threshold is around $\Delta \theta_\mathrm{crit} \approx 6 \times 10^{-10} \si{rad}$ and $\Delta L_\mathrm{crit} \approx 1.4 \times 10^{-10} \si{m}$, which seems quite challenging experimentally.
However, it seems like that for smaller times $t<t_\mathrm{max}$ larger variations can be tolerated for the cost of having less total entanglement.
This again is expected. For smaller times, the gravitational force did not have enough time to fully entangle the two particles, but also the decoherences (dependent on $\propto t^2$ in eq. \eqref{eq:4:average-density-element}) did not have enough time to built up. 
It is therefore logical, that if one does not require to measure a fully entangled state and less entanglement $E_N < 1$ is also sufficient, it may be beneficial to measure at a time $t < t_\mathrm{max}$. 
This does not only reduce the time of a single experimental run, but also increases the stability against displacement variations. This optimal time of measuring for a certain required amount of entanglement is shown in \cref{fig:4:time-delta-theta}.
\begin{figure}[!htb]
  \centering
  \includegraphics[width=\textwidth]{./../figures/theta-variance/time-delta-theta-crit-EN.pdf}
  \caption{Maximal angular variation for a given time and a given required amount of entanglement. The outer most black line corresponds to the time dependence of $\Delta \theta_\mathrm{crit}$. The top left as well as the red curve in the main figure shows the optimal measuring time for a sufficient amount of entanglement.
  At times $2k t_\mathrm{max}$ no entanglement can be measured.}
  \label{fig:4:time-delta-theta}
\end{figure}
The chart lets one read out the optimal measuring time $t$ for a sufficient amount of entanglement as well as the corresponding maximal angular variation for this to be possible.
On the other hand, if one is experimentally limited by a certain maximal angular variation, the corresponding best measuring time as well as the amount of entanglement one maximally gets can be read out.
It becomes evident, that at times $2k t_\mathrm{max}$, $k\in\mathbb{N}$ there is no entanglement present.
This aligns with the findings form the ideal scenario in \cref{cha:first-look}. 

\section{The optimal setup}\label{sec:4:optimal-setup}
With the general framework in hand, the next logical question to ask is, if the stability against placement-variations can be improved.
The general rule of thumb for these optimizations is the following:
Increase the gravitational interaction by either heavier and larger particles or by reducing the separation distance $L$ without substantial sacrifices of experimental realization.
As an example, the stability increases intuitively by increasing the separation distance $L$. However, this does also increase the time $t_\mathrm{max}$ until the maximum amount of entanglement can be measured which would increase the total time  $\sim \# t_\mathrm{max}$ of the experiment with $\#$ individual measurements.
It is not immediately obvious, how the stability and the maximum possible variations $\Delta \theta_\mathrm{crit}$ and $\Delta L_\mathrm{crit}$ behave for the change in parameters.
In the following section, precisely the changing of this stability is discussed for changing the orientation $\alpha, \beta$, the separation $L$, the mass $M_A = M_B \equiv M$ and the superposition size $\Delta x_A = \Delta x_B \equiv \Delta x$.


\subsection{Orientation}
The arguably easiest parameter to change experimentally is the orientation of the superpositions relative quantified by $\alpha$ and $\beta$ in \cref{fig:4:complete-setup}.
As could be already seen in \cref{fig:2:entanglement-dynamics} in \cref{cha:first-look}, the entanglement dynamics are dependent on the orientation. 
The particles in the parallel orientation have to evolve twice as long compared to the orthogonal orientation to get maximally entangled.
In general, it is beneficial to aim for the highest entanglement rate and thus for the smallest $t_\mathrm{max}(\alpha, \beta)$ as this requires a smaller coherence time and the total experimental time can be reduced.
The earlier results from \cref{cha:first-look} can be more generalized for an arbitrary orientation $\alpha, \beta$. The logarithmic negativity is given by
\begin{equation}
  E_N = \log_2\left(1+\abs{\sin\Delta\phi}\right)
\end{equation}
where $\Delta\phi$ now depends on the orientation and is given by (for $\Delta x \ll L$)
\begin{equation}
  \Delta \phi = \frac{G M_A M_B t \Delta x_A \Delta x_B}{8\hbar L^3} \left(\sin\alpha\sin\beta-\frac{1}{2}\cos\alpha\cos\beta\right) .
\end{equation}
The maximum entanglement $E_N=1$ is reached for $\Delta\phi = \pm \pi/2$ and thus after a time
\begin{equation}\label{eq:4:t-max}
  t_\mathrm{max}(\alpha,\beta) = \frac{4\pi \hbar L^3}{G M_A M_B \Delta x_A \Delta x_B}\abs{\sin\alpha\sin\beta - \frac{1}{2}\cos\alpha\cos\beta}^{-1}.
\end{equation}
The resulting times are shown in \cref{fig:4:t-max-orientation}.
\begin{figure}[!htbp]
  \centering
  \includegraphics[width=\textwidth]{./../figures/ideal-entanglement/t-max-orientation.pdf}
  \caption{Time $t_\mathrm{max}$ after which maximum entanglement ($E_N = 1$) is reached for different orientations. Only the most interesting and highly symmetric cases $\alpha=\pm\beta$ and $\beta=0$ are shown.}
  \label{fig:4:t-max-orientation}
\end{figure}
The global minima of $t_\mathrm{max}(\alpha,\beta)$ is reached in the orthogonal orientation. This is not surprising considering that this orientation maximizes the difference in separation distance between all superposition states.
Much more interesting are the apparent singularities in \cref{fig:4:t-max-orientation} which arise for 
\begin{equation}
  \sin\alpha\sin\beta=\frac{1}{2}\cos\alpha\cos\beta .
\end{equation}
For $\beta=0$ the singularity for $\alpha=\pi/2$ is not surprising. In this configuration, the distances $\ket{\psi^1_A}\leftrightarrow\ket{\psi^{1,2}_B}$ and $\ket{\psi^2_A}\leftrightarrow\ket{\psi^{1,2}_B}$ are the same and thus these states accumulated the same phases, which results in a factorable global phase.
In the case of $\alpha=\beta$, the two singularities are precisely given in the orientation
\begin{equation}
  \alpha=\beta=2 \arctan(\sqrt{3}\pm\sqrt{2})\approx 90\deg \pm 54.74\deg .
\end{equation}
There does not exist a straight-forward geometric interpretation why no entanglement is generated in this configuration, however all 4 separation distances between the states precisely form the \q{harmonic mean} visualized in \cref{fig:4:harmonic-mean}.
\begin{figure}[!htbp]
  \centering
  \def\svgwidth{\textwidth}
  \input{./../figures/harmonic-mean.pdf_tex}
  \caption{\textbf{left:} The system in the orientation $\alpha=\beta=2\arctan(\sqrt{3}-\sqrt{2})$. For $\Delta x \ll L$, ll separation distances exactly lie in the \textit{harmonic mean}. \textbf{right:} Geometric visualization of the harmonic mean.}
  \label{fig:4:harmonic-mean}
  % COLORFUL:: $\textcolor[HTML]{aa0000}{\blacksquare} = \displaystyle\frac{2}{\frac{1}{\textcolor[HTML]{0044aa}{\blacksquare}}+\frac{1}{\textcolor[HTML]{447821}{\blacksquare}}}$
\end{figure}
To avoid all these singularities, it is advisable to always take $\alpha=-\beta$, where all orientations result in roughly similar entanglement times, at most only differing by a factor of $2$.

It should come by no surprise that the different orientations exhibit different stabilities. Logically, one would expects that the orthogonal configuration is much more sensitive against angular variations compared to the parallel one.
On the contrary, the parallel configuration should be much more stable against variations in the distance, as no phase difference (\q{dephasing}) between the two superposition states $\ket{\psi^1_{A(B)}}$ and $\ket{\psi^2_{A(B)}}$ of the same mass is induced.

The effect of different orientations on the stability against angular variations and how the critical angular variation $\Delta \theta_\mathrm{crit}$ behaves is shown in \cref{fig:4:theta-crit-orientation}.
\begin{figure}[!htbp]
  \centering
  \includegraphics[width=0.95\textwidth]{./../figures/theta-variance/theta-crit-orientation-complete.pdf}
  \caption{Maximum possible allowed angular variation $\Delta\theta_\mathrm{crit}$ for different orientations. All data-points where calculated at the time of maximum entanglement shown in \cref{fig:4:t-max-orientation}. The \emph{orthogonal configuration} is very stable against angular disturbances. At $\alpha=\beta=\pi/2$, only exact numerical results show a finite value. The singularities on the left figure arise from the fact, that these configurations need infinite time to entangle as already seen in \cref{fig:4:t-max-orientation}.}
  \label{fig:4:theta-crit-orientation}
\end{figure}
As expected, the orthogonal configuration is the most stable against these kind of variations. This is, because the dephasing ultimately depends on the distance between the state and the shield $L \pm \Delta x/2 \cos\theta \approx L \pm \Delta x/2 (1 - \theta^2/2)$, which is a only second order effect of the angular variations $\theta$.
This explains the apparent \q{infinitely} good stability in the figure, as the analytical solution only uses first order approximations in $\theta$.
Exact numerical results however cap the stability at $\Delta \theta_\mathrm{crit,\,orthogonal} \approx 7.3\times 10^{-5}\si{rad}$.

Respectively, the stability against distance variations $\Delta L_\mathrm{crit}$ for different orientations is shown in \cref{fig:4:L-crit-orientation}.
\begin{figure}[!htbp]
  \centering
  \includegraphics[width=0.95\textwidth]{./../figures/L-variance/L-max-orientation-complete.pdf}
  \caption{Maximum possible allowed distance variation $\Delta L_\mathrm{crit}$ for different orientations. This figure is similar to \cref{fig:4:theta-crit-orientation}, only for distance variations. On the contrary to angular variations, the \emph{parallel configuration} is infinitely stable against changes in the distance between the shield and the cat-state.}
  \label{fig:4:L-crit-orientation}
\end{figure}
Again aligning with expectations, the parallel configuration is (in theory) exhibits an infinite stability.
One however could argue, that a for this to hold, the uncertainties in the angular placement have to be zero. As could be seen in \cref{fig:4:theta-crit-orientation}, these variations are at most around $\sim 5 \times 10^{-5}\si{rad}$ and thus a realistic upper bound for the minimum required distance variations is given by $\Delta L_\mathrm{crit,parallel} \simeq 4\times 10^{-11}\si{m}$.
It is important to keep in mind, that these stability values can be improved substantially by changing e.g. the separation distance $L$ or the particle size $R$.

Looking at these results, the parallel orientation seems like the only realistic experimental option even though it requires lager coherence times $t_\mathrm{max}$.
Keeping separation variations below $0.01\si{nm}$ - roughly the size of a single atom - is virtually impossible considering thermal vibrations of the shield and the sphere cannot be fully prevented and are in the same order of magnitude, as seen later in \cref{cha:the-shield}.
With this data on hand, it is possible to construct a stability diagram \cref{fig:4:optimal-orientation}, showing the ideal orientation in which the most entanglement can be measured.
\begin{figure}[!htbp]
  \centering
  \includegraphics[width=\textwidth]{./../figures/optimize/optimized-orientation-advanced.pdf}
  \caption{Optimal orientation for the experimental setup dependent on the variations in angle $\Delta\theta$ and distance $\Delta L$ for a initial separation distance of $L=2R=2\times 10^{-5}\si{m}$ at time $t_\mathrm{max}$. The different predictions for the proximity-force-approximation (PFA) and the large-separation-limit (LSL) are shown. At distance $L=20\si{\mu m}$ the actual casimir interaction is somewhere in the middle between both approximation methods. In the region where entanglement is given regardless of the orientation (the bottom left), the orientation with \textit{more} entanglement is still colored. The dotted line corresponds to the realistic upper bound discussed in the text.}
  \label{fig:4:optimal-orientation}
\end{figure}
For most of the regions in the diagram, entanglement is only given in one certain orientation.


\subsection{Separation, mass and superposition size}

\begin{figure}[!htbp]
  \centering
  \includegraphics[width=\textwidth]{./../figures/theta-variance/theta-crit-L.pdf}
  \caption{R is fixed as $R=10^{-5}\si{m}$}
  \label{fig:4:theta-crit-L}
\end{figure}

\begin{figure}[!htbp]
  \centering
  \includegraphics[width=\textwidth]{./../figures/theta-variance/theta-max-mass.pdf}
  \caption{...}
  \label{fig:4:theta-crit-mass}
\end{figure}

\begin{figure}[!htbp]
  \centering
  \includegraphics[width=\textwidth]{./../figures/theta-variance/theta-max-superpos-extension.pdf}
  \caption{...}
  \label{fig:4:theta-crit-superposition-size}
\end{figure}



% % \section{Stability in different orientations}\label{sec:4:orientation}
As could be already seen in \cref{fig:2:entanglement-dynamics} in \cref{cha:first-look}, the entanglement dynamics and especially the time $t_\mathrm{max}$ of maximum entanglement highly depend on the orientation.
There, it was shown already, that the for the parallel configuration the maximum entanglement is reached twice as slow as in the orthogonal orientation.
This result can be more generalized for an arbitrary orientation quantified by $\alpha$ and $\beta$ from \cref{fig:4:complete-setup}.


\begin{equation}
  E_N = \log_2\left(1 + \abs{\sin\Delta \phi}\right)
\end{equation}
where $\Delta \phi$ is now dependent on the orientation and is given by (for $\Delta x \ll L$)
\begin{equation}
  \Delta \phi = \frac{G M_A M_B t}{\hbar}\frac{\Delta x_A \Delta x_B}{8 L^3} \left[\sin\alpha\sin\beta - \frac{1}{2}\cos\alpha\cos\beta\right] .
\end{equation}
The maximum entanglement is again reached after $\Delta\phi = \pi/2$ and thus is also dependent on the orientation
\begin{equation}\label{eq:4:t-max}
  t_\mathrm{max} = \frac{4\pi L^3\hbar}{GM_AM_B\Delta x_A \Delta x_B} \abs{\sin\alpha\sin\beta - \frac{1}{2}\cos\alpha\cos\beta}^{-1}.
\end{equation}
The resulting times are shown in \cref{fig:4:t-max-orientation} and align with earlier results for the special cases of the orthogonal and parallel configuration.
\begin{figure}[!htbp]
  \centering
  \includegraphics[width=\textwidth]{./../figures/ideal-entanglement/EN-orientation.pdf}
  \caption{Time of maximum entanglement $t_\mathrm{max}$ relative to $t_\mathrm{max,orthogonal}$ of the orthogonal configuration for different orientations. Some orientations like $\alpha=\pi/2$ and $\beta=0$ cannot induce any entanglement at all because of symmetry considerations.}
  \label{fig:4:t-max-orientation}
\end{figure}
The global minima of $t_\mathrm{max}$ is given in the orthogonal configuration meaning that of all possible configurations, this one induces entanglement the fastest. Considering the discussions in the end of \cref{cha:first-look}, this is not very surprising.
Much more interesting are the apparent singularities which arise for 
\begin{equation}
  \sin\alpha\sin\beta = \frac{1}{2}\cos\alpha\cos\beta .
\end{equation}
For $\beta=0$, the singularity in $t_\mathrm{max}$ at $\alpha=\pi/2$ is not surprising. At this configuration, the distances $\ket{\psi_A^1} \leftrightarrow \ket{\psi_B^{1,2}}$ are identical as well as $\ket{\psi_A^2} \leftrightarrow \ket{\psi_B^{1,2}}$. 
For the case $\alpha = \beta$, the two singularities are precisely given at
\begin{equation}
  \alpha = \beta = 2\arctan(\sqrt{3}\pm\sqrt{2}) \approx 90\deg \pm 54.74\deg.
\end{equation}
There does not exist a clear geometric explanation why no entanglement is generated in this configuration, however, all distances between the superpositions for a harmonic mean as visualized in \cref{fig:4:harmonic-mean}.
\begin{figure}[!htbp]
  \centering
  \def\svgwidth{\textwidth}
  \input{./../figures/harmonic-mean.pdf_tex}
  \caption{\textbf{left:} Arrangement in the orientation $\alpha=\beta=2\arctan(\sqrt{3}-\sqrt{2})$. All distances exactly lie in the \textit{harmonic mean} (for $\Delta x \ll L$). \textbf{right:} Geometric visualization of the harmonic mean.}
  \label{fig:4:harmonic-mean}
  % COLORFUL:: $\textcolor[HTML]{aa0000}{\blacksquare} = \displaystyle\frac{2}{\frac{1}{\textcolor[HTML]{0044aa}{\blacksquare}}+\frac{1}{\textcolor[HTML]{447821}{\blacksquare}}}$
\end{figure}





\begin{figure}[!htbp]
  \centering
  \includegraphics[width=0.95\textwidth]{./../figures/theta-variance/theta-max-orientation-complete.pdf}
  \caption{Maximum possible allowed angular variation $\Delta\theta_\mathrm{max}$ for different orientations. All data-points where calculated at the time of maximum entanglement shown in \cref{fig:4:t-max-orientation}. The \emph{orthogonal configuration} is very stable against angular disturbances. At $\alpha=\beta=\pi/2$, only exact numerical results show a finite value. The singularities on the left figure arise from the fact, that these configurations need infinite time to entangle as already seen in \cref{fig:4:t-max-orientation}.}
  \label{fig:4:theta-max-orientation}
\end{figure}

\begin{figure}[!htbp]
  \centering
  \includegraphics[width=0.95\textwidth]{./../figures/L-variance/L-max-orientation-complete.pdf}
  \caption{Maximum possible allowed distance variation $\Delta L_\mathrm{max}$ for different orientations. This figure is similar to \cref{fig:4:theta-max-orientation}, only for distance variations. On the contrary to angular variations, the \emph{parallel configuration} is infinitely stable against changes in the distance between the shield and the cat-state.}
  \label{fig:4:L-max-orientation}
\end{figure}



\begin{figure}[!htbp]
  \centering
  \includegraphics[width=\textwidth]{./../figures/optimize/optimized-orientation.pdf}
  \caption{Optimal orientation for arbitrary variations in the angle $\Delta\theta$ and the distance $\Delta L$. The optimum was calculated for different models of the Casimir-interaction (PFA eq. \eqref{eq:3:PFA-sphere-plate} and LSL eq. \eqref{eq:3:casimir-sphere-plate}).If angular variations dominate, the orthogonal configuration is best, whereas for large distance variations, a parallel orientation is advised.}
  \label{fig:4:optimal-orientation}
\end{figure}


\section{Trapping the particle}\label{sec:4:trapping}
Another consequence of shielding that requires consideration is the trapping. Levitated particles are trapped and cooled in an ultra-high vacuum by either a magnetic, optical or electrical radiofrequency Paul-trap \cite{GonzalezBallestero_2021}.
These traps differ in the trapping mechanism, but if the particle is cooled close to the ground state, all trapping potentials can be considered \q{harmonic} with trapping frequency $\omega_\mathrm{trap} = 2\pi \times f$.
The strength of the trapping potential $V \propto f^2$ differs for the different trapping types. Typical values range from $1\si{Hz}-1\si{kHz}$ for magnetic traps \cite{Slezak_2018,GonzalezBallestero_2021} up to $10\si{kHz}-300\si{kHz}$ for optical traps \cite{GonzalezBallestero_2021}. 
The different types of traps also offer different advantages and disadvantages: Optical traps are relatively noisy due to the constant interaction between the particle and the light. Magnetic traps for large particles are less noisy, but only low trapping frequencies are possible \cite{GonzalezBallestero_2021}. 
For electric traps, the particle must be charged, which causes a lot of different problems, as seen in \cref{sec:5:shield-size}.

If the particle in the harmonic trapping potential is placed close to the shield, the Casimir interaction $\sim \mathscr{L}^{-2}$ can disturb the trapping and eventually even suck the particle onto the shield.
The total potential $V_\mathrm{tot}=V_\mathrm{trap} + V_\mathrm{Casimir}$ is shown in \cref{fig:4:trap-eigenstates} for a stable and unstable configuration.
\begin{figure}[!htbp]
  \centering
  \includegraphics[width=\textwidth]{./../figures/others/trapping-potential-eigenstates.pdf}
  \caption{Visualization of the potential as an overlay of the harmonic trapping potential $V_\mathrm{trap} = m (2\pi f)^2 L^2 / 2$ and the casimir potential $V_\mathrm{Casimir}$. $f$ is the trapping frequency and $L_0$ the position of the trap. In red, eigenstates of the potential are visualized offset by the eigen-energies.
  \textbf{(a)} Almost harmonic bounded potential which can hold the particle, if its energy is less than $E_0$.
  \textbf{(b)} Potential with no bounded states. Here, trapping is not possible.}
  \label{fig:4:trap-eigenstates}
\end{figure}
Due to the influence of the attractive Casimir force, the equilibrium position of the trap shifts slightly closer to the shield by $\Delta \xi$. This shift
\begin{equation}
  \Delta \xi = \frac{-\nabla V_\mathrm{Casimir}}{m (2\pi f)^2} = \frac{2 \hbar c \pi^3}{720} \left(\frac{\varepsilon_r - 1}{\varepsilon_r + 1}\right) \varphi(\varepsilon_r) \frac{R}{\mathscr{L}^3} \frac{1}{m (2\pi f)^2}
\end{equation}
is negligibly small as it is in the order of $\Delta \xi \approx 10^{-13}\si{m}$ for $f=1\si{kHz}$ and $L = 2R = 20\si{\mu m}$.

To determine the stability of a trapped particle with mass $M \propto R^3$ in a trap with frequency $f$ placed at a distance $L_0 > R$ in front of the shield, the number of bound energy-eigenstates in the potential $V_\mathrm{tot}$ is considered.
From \cref{fig:4:trap-eigenstates} it becomes clear, that as long as the particles thermal energy is well below $E_0$, the trap is stable and the particle is bound.
Here, $E_0$ is defined as the local maximum of the potential 
\begin{equation}
  E_0 = \max_{L\in(R,L_0)} \left( V_\mathrm{trap} + V_\mathrm{Casimir} \right)
\end{equation}
where $L_0$ is the position where the particle is trapped.
If there does not exist a local maximum, i.e.
\begin{equation}
  \pdv{L} \left( V_\mathrm{trap} + V_\mathrm{Casimir} \right) > 0
\end{equation}
for all $L \in (R, L_0)$, the resulting trap cannot be stable. These regions of no stability are shown as a white area in the stability diagram \cref{fig:4:trap-stability}.
In the general case, the stability can be measured by computing the number of bound eigenstates $n(E_0)$ with energies less than $E_0$ and comparing them with the number of thermally excited states $\bar{n}$.
At a temperature $T$ on average 
\begin{equation}
  \bar{n} = \frac{1}{e^{\beta \hbar \omega} - 1}
\end{equation}
states are occupied, where $\beta = 1/k_B T$ and $\omega = 2\pi f$. This is true, as long as the potential is assumed to be harmonic, which is, as seen shortly, a very good approximation.
To find the number of possible bound energy-eigenstates in the potential, I am using the \emph{WKB-approximation} \cite{Schleich_2001}.
In this approximation, the energy of the $n$-th eigenstate of a smooth and appropriately slow varying potential $V(x)$ can be calculated using \cite[p. 163]{Schleich_2001}
\begin{equation}
  \int_{x_1}^{x_2} \dd x \, \sqrt{2m(E-V(x))} = \left(n + \frac{1}{2}\right)\pi\hbar ,
\end{equation}
where $V(x_1) = V(x_2) = E$ are two turning points corresponding to energy $E$.
Conversely, it is possible to use this approximation to numerically estimate the total number of bound states in the potential $V = V_\mathrm{trap} + V_\mathrm{Casimir}$ using
\begin{equation}
  n(E_0) \approx \frac{1}{\hbar \pi} \int_{x_1}^{x_2} \dd x \, \sqrt{2m(E_0 - V(x))},
\end{equation}
which is closely given (highest deviation around $40\%$; averaged relative error $\sim 0.9\%$) by the harmonic approximation $n(E_0) \sim E_0 / \hbar \omega$.
The resulting number of bound states is shown in \cref{fig:4:trap-stability} as well as the stability boundaries at specific temperatures where $\bar{n} = n(E_0)$.
\begin{figure}[!htbp]
  \centering
  \includegraphics[width=\textwidth]{./../figures/others/trap-stability-with-R.pdf}
  \caption{Stability diagram for different trapping frequencies $f = \omega/2\pi$ and particle-shield separations $L_0$. The number of bound energy-eigenstates for each combination of $f$ and $L_0$ are calculated using the WKB-approximation. The number of thermally occupied states $\bar{n}$ at different Temperatures is overlaid. As an example, for $f=1\si{Hz}$, $\bar{n}(T=300\si{K})\approx 10^{13}$ states are thermally occupied. All regions below these boundaries are unstable. A increase in the radius $R$ and thus the mass $M$ improves the regions of stability massively.}
  \label{fig:4:trap-stability}
\end{figure}
It turns out, that regardless the type of the trap, a successful trapping even at room temperature should be possible as long as the particle is placed appropriately far away from the trap.
The ability to trap and levitate the masses is therefore not significantly impaired by the presence of the Faraday shield.

\section{Discussion on the optimal setup}\label{sec:4:discussion}

- No global maximum

- dependent on experimental stuff

- Masses and distances and so on must be bounded by 10xECasimir < EGravity

How can one find the best parameters depending on the experimental requirements?
\begin{enumerate}
  \item Specify maximum coherence time (how long the experiment can take)
  \item This fixes $L^3/(M_A M_B \Delta x_A \Delta x_B)$
  \item Look up, what maximum $\Delta \theta$ and $\Delta L$ you can get at this time (Dependent on the amount of entanglement you want to have at the end)
  \item Take the minimum possible $L/R$ in account (dependent on trap stiffness and so on) Ideally, large enough to reduce casimir interactions
\end{enumerate}


-------------------------------------- 

\begin{enumerate}
  \item Given a target measuring time $t_\mathrm{target}$, this fixes the ratio
  \begin{equation}\label{eq:4:fixed-ratio}
    \frac{L^3}{M_A M_B \Delta x_A \Delta x_B} = 5.036\times 10^{22}\si{m/kg^2s} \cdot t_\mathrm{target} = \mathrm{const.}
  \end{equation}
  Most likely, the superposition size $\Delta x_{A,B}$ is also not changeable and one has to be satisfied by what can be achieved experimentally. The delocalized states of mass $M$ has to interact gravitationally with each other in a laboratory setting for at least the duration of the measuring process $t_\mathrm{target}$. These considerations limits the coherence time and usually, low times in the order of milliseconds to seconds are favorable.
  \item In principle, one does not require to measure at the time of maximum entanglement where $E_N = 1$. If a lower quantity of entanglement is enough for a set experimental goal, I recommend, measuring at a lower time than $t_\mathrm{target} = \tau t_\mathrm{max}$ ($\tau < 1$) !!!!FIGURE!!!!. This on the other hand increases the fixed ratio eq. \eqref{eq:4:fixed-ratio} by a factor of $1/\tau$.
  Therefore, it is possible to use smaller superposition sizes $\Delta x$, lighter masses or increase the distance $L$ which decreases Casimir interactions.
  \item Most likely, the minium angular 
\end{enumerate}