\chapter{Primary calculations}

\section{Entanglement in different orientations}\label{apx:entanglement-orientation}
Expanding the separations $1/L^{(ij)}$ between the states $\ket{\psi^{(i)}_{A}}$ and $\ket{\psi^{(j)}_{B}}$ in second order in $\Delta x$ and calculating the dynamic phases, the following represents the time-evolved state:
\begin{equation}
  \rho_\mathrm{Phase} = \frac{1}{4} \begin{pmatrix}
    1 & e^{i(\phi_B + \phi_{AB})} & e^{-i(\phi_A - \phi_{AB})} & e^{-i(\phi_A - \phi_{B})} \\
    & 1 & e^{-i(\phi_A + \phi_{B})} & e^{-i(\phi_A + \phi_{AB})} \\
    \multicolumn{2}{c}{\multirow{2}{*}{c.c}} & 1 & e^{i(\phi_B - \phi_{AB})} \\
    & & & 1
  \end{pmatrix}
\end{equation}
with the substitutions
\begin{align}\label{eq:apx:phi-definition}
  &\phi_A = g_\mathrm{Grav} \frac{\Delta x_A \sin\alpha}{4L^2},\quad
  \phi_B = g_\mathrm{Grav} \frac{\Delta x_B \sin\beta}{4L^2},\\ \label{eq:apx:phi-AB}
  &\phi_{AB} = g_\mathrm{Grav} \frac{\Delta x_A \Delta x_B}{8 L^3} \left(\frac{1}{2}\cos\alpha\cos\beta - \sin\alpha\sin\beta \right)
\end{align}
and the gravitational coupling strength
\begin{equation}
  g_\mathrm{Grav} = \frac{G M_A M_B t}{\hbar} .
\end{equation}
Eigenvalues of $\rho^{\Gamma}$ are
\begin{equation}
  \left\{ \sin^2\left(\frac{\phi_{AB}}{2}\right), \cos^2\left(\frac{\phi_{AB}}{2}\right), \frac{\sin(\phi_{AB})}{2}, -\frac{\sin(\phi_{AB})}{2}\right\}
\end{equation}
and thus the logarithmic negativity is given by the sum of negative eigenvalues as
\begin{equation}
  E_N = \log_2\left\{ 1 + \abs{\sin(\frac{G M_A M_B \Delta x_A \Delta x_B t}{8\hbar L^3}\left[\sin\alpha\sin\beta-\frac{1}{2}\cos\alpha\cos\beta\right])} \right\} .
\end{equation}
Due to Casimir interactions between the particles and the shield, in first order only the additional term ($\delta = \alpha, \beta$)
\begin{equation}
  \phi_\mathrm{Cas.,\,A(B)} = \frac{c \pi^3}{720}\left(\frac{\varepsilon_r - 1}{\varepsilon_r + 1}\right)\varphi(\varepsilon_r) R \frac{2 \Delta x_{A(B)} \sin\delta}{(L - R - d/2)^3} t
\end{equation}
is added to the phase $\phi_{A(B)}$, leaving the resulting logarithmic negativity unchanged.


\section{Density matrix with stochastic placement variations}\label{apx:placement-average-density-matrix}
Expanding the separations $1/(\mathscr{L}^{(i)}_{A(B)})^2$ for the PFA from eq. \eqref{eq:4:L-casimir} in first order in $\theta_{A(B)}$ and $L_{A(B)}$ (here exemplary for mass $A^{(1)}$):
\begin{multline}
  \frac{1}{\left(\mathscr{L}^{(1)}_{A}\right)^2} \approx \frac{4}{(d-2L+2R)^2} + \frac{8 \Delta x_{A(B)}\sin\delta}{(d-2L+2R)^3}
  + \theta_{A(B)}\left(\frac{8\Delta x_{A(B)} \cos\delta}{(d-2L+2R)^3}\right) \\
  + L_{A(B)}\left(\frac{16}{(d-2L+2R)^3} + \frac{48\Delta x_{A(B)} \sin\delta}{(d-2L+2R)^4}\right) + \theta_{A(B)}L_{A(B)}\frac{48\Delta x_{A(B)} \cos\delta}{(d-2L+2R)^4}
\end{multline}
Similar results can be obtained for expanding the gravitational couplings $1/L^{(ij)}$ from eq. \eqref{eq:4:L-gravity}:
\begin{multline}
  \frac{1}{L^{(11)}} \approx \frac{1}{2L} + \frac{\Delta x_B \sin\beta - \Delta x_A \sin\alpha}{8L^2} - \theta_A\frac{\Delta x_A\cos\alpha}{8L^2} + \theta_B\frac{\Delta x_B\cos\beta}{8L^2} \\
  + L_A \left(-\frac{1}{4L^2} + \frac{\Delta x_A \sin\alpha-\Delta x_B \sin\beta}{8L^3}\right)
  + L_B \left(-\frac{1}{4L^2} + \frac{\Delta x_A \sin\alpha-\Delta x_B \sin\beta}{8L^3}\right) \\
  + L_A \theta_A \frac{\Delta x_A \cos\alpha}{8L^3} - L_A \theta_B \frac{\Delta x_B \cos\beta}{8L^3}
  + L_B \theta_A \frac{\Delta x_A \cos\alpha}{8L^3} - L_B \theta_B \frac{\Delta x_B \cos\beta}{8L^3} \\
  + L_A L_B \left(\frac{2}{4L^3} + \frac{3\Delta x_B \sin\beta - 3 \Delta x_A \sin\alpha}{16L^4}\right) \\
  - L_A L_B \theta_A \frac{3 \Delta x_A \cos\alpha}{16L^4} + L_A L_B \theta_B \frac{3 \Delta x_B \cos\beta}{16L^4}
\end{multline}
The resulting average over $\theta_{A(B)}$ and $L_{A(B)}$ can be computed by
\begin{equation}
  \int_{-\infty}^{\infty} \dd \theta_A \dd \theta_B \dd L_A \dd L_B \, p(\theta_A) p(\theta_B) p(L_A) p(L_B) e^{i \phi}
\end{equation}
where $p(\,\cdot\,)$ is a Gaussian probability distribution in the form of
\begin{equation}
  p(x) = \frac{1}{\sqrt{2\pi}\Delta x} e^{-\frac{x^2}{2(\Delta x)^2}}
\end{equation}
and $\phi$ is the accumulated dynamic phase dependent on the two series expansions above.
The mixed terms consisting of combinations of $\theta$ and $L$ can be neglected in first order because in the final result, they appear in the form of (notation: $\Delta A,\Delta B$ for either $\Delta\theta$ or $\Delta L$; $a,b$ are constants)
\begin{equation}
  \sim \exp{-\frac{a^2(\Delta A)^2}{2b^2(\Delta A)^2(\Delta B)^2 + 2}} \xrightarrow{\Delta A,\Delta B \ll 1} 1 ,
\end{equation}
decaying much faster than the relevant terms $\sim \exp{-(\Delta A)^2}$.
Each averaged element of the density matrix can therefore be analytically calculated using 
\begin{equation}
  \prod_{\Delta A = \{\Delta \theta_{A(B)}, \Delta L_{A(B)}\}} \int\limits_{-\infty}^{\infty} \dd A \, \frac{1}{\sqrt{2\pi}\Delta A} e^{-\frac{A^2}{2 (\Delta A)^2}} e^{i\xi A} e^{i\tilde{\phi}} = \prod_{\Delta A} e^{-\frac{\xi^2 (\Delta A)^2}{2}} e^{i\tilde{\phi}} ,
\end{equation}
where $\tilde{\phi}$ are phase components independent of $A$, in particular representing the gravitational interaction considered in \cref{apx:entanglement-orientation}.

The resulting averaged density matrices are given by the following for variations in the angle:
\begin{equation}
  \mean{\rho_{\theta}} = \begin{pmatrix}
    1 & e^{-\xi_B^2\frac{(\Delta \theta_B)^2}{2}t^2} & e^{-\xi_A^2\frac{(\Delta \theta_A)^2}{2}t^2} & e^{-\xi_A^2\frac{(\Delta \theta_A)^2}{2}t^2}e^{-\xi_B^2\frac{(\Delta \theta_B)^2}{2}t^2}\\
    & 1 & e^{-\xi_A^2\frac{(\Delta \theta_A)^2}{2}t^2}e^{-\xi_B^2\frac{(\Delta \theta_B)^2}{2}t^2} & e^{-\xi_A^2\frac{(\Delta \theta_A)^2}{2}t^2}\\
    \multicolumn{2}{c}{\multirow{2}{*}{c.c}} & 1 & e^{-\xi_B^2\frac{(\Delta \theta_B)^2}{2}t^2}\\
    & & & 1
  \end{pmatrix}
\end{equation}
with the abbreviations ($\delta = \alpha,\beta$)
\begin{equation}\label{eq:apx:definition-xi}
  \xi_{A(B)} = \left(\frac{c \pi^3}{720} \left(\frac{\varepsilon_r - 1}{\varepsilon_r + 1}\right)\varphi(\varepsilon_r) R \frac{2\Delta x_{A(B)} \cos\delta}{(L-R-d/2)^3} + \frac{G M_A M_B}{\hbar} \frac{\Delta x_{A(B)} \cos\delta}{4L^2}\right)
\end{equation}
Similar results for variations in the particle-shield separation $L$ can be obtained:
\begin{multline}
  \mean{\rho_{L}} = \left(\begin{matrix}
    1 & e^{-\chi_B^2\frac{(\Delta L_A)^2}{2}t^2}e^{-(\zeta_B + \chi_B)^2\frac{(\Delta L_B)^2}{2}t^2} & e^{-(\zeta_A+\chi_A)^2\frac{(\Delta L_A)^2}{2}t^2}e^{-\chi_A^2\frac{(\Delta L_B)^2}{2}t^2} & \dots \\
    & 1 & e^{-(\chi_A+\zeta_A)^2\frac{(\Delta L_A)^2}{2}t^2}e^{-(\chi_B+\zeta_B)^2\frac{(\Delta L_B)^2}{2}t^2} & \dots\\
    \multicolumn{2}{c}{\multirow{2}{*}{c.c}} & 1 & \dots\\
    & & & \dots
  \end{matrix}\right. \\
  \left.\begin{matrix}
    \dots & e^{-(\chi_A+\zeta_A)^2\frac{(\Delta L_A)^2}{2}t^2}e^{-(\chi_B+\zeta_B)^2\frac{(\Delta L_B)^2}{2}t^2} \\
    \dots & e^{-(\zeta_A+\chi_A)^2\frac{(\Delta L_A)^2}{2}t^2}e^{-\chi_A^2\frac{(\Delta L_B)^2}{2}t^2} \\
    \dots & e^{-\chi_B^2\frac{(\Delta L_A)^2}{2}t^2}e^{-(\zeta_B + \chi_B)^2\frac{(\Delta L_B)^2}{2}t^2} \\
    \dots & 1
  \end{matrix}\right)
\end{multline}
with 
\begin{align}
  \chi_{A(B)} =& \frac{G M_A M_B}{\hbar}\frac{\Delta x_{A(B)}\sin\delta}{4L^3} \\ \label{eq:apx:definition-zeta}
  \zeta_{A(B)} =& \frac{c\pi^3}{720}\left(\frac{\varepsilon_r - 1}{\varepsilon_r + 1}\right)\varphi(\varepsilon_r) R \frac{6\Delta x_{A(B)} \sin\delta}{(L-R-d/s)^4}
\end{align}
The combined mean density matrix is therefore given by
\begin{equation}
  \mean{\rho} = \rho_\mathrm{Phase} \odot \mean{\rho_{\theta}} \odot \mean{\rho_{L}}
\end{equation}
where the symbol $\odot$ represents the element-wise matrix product (Hadamard product).
In the special case where $\Delta \theta_A = \Delta \theta_B$ and $\Delta L_A = \Delta L_B$ as well as where both particles are identical, i.e. $\Delta x_A = \Delta x_B$ and $M_A=M_B$ and $\alpha=\pm\beta$, the logarithmic negativity can be obtained analytically.
Using the approximation $\chi_{A(B)} \ll \zeta_{A(B)}$, which is justified for all separations $R < L \lesssim 9.7 \si{m}$.
\begin{equation}
  \mean{\rho} = \rho_\mathrm{Phases} \odot \begin{pmatrix}
    1 & e^{-\gamma} & e^{-\gamma} & e^{-2\gamma} \\
    & 1 & e^{-2\gamma} & e^{-\gamma} \\
    \multicolumn{2}{c}{\multirow{2}{*}{c.c}} & 1 & e^{-\gamma} \\
    & & & 1
  \end{pmatrix}
\end{equation}
with 
\begin{equation}\label{eq:apx:stochastic-decoherence}
  \gamma = \left( \xi^2 \frac{(\Delta \theta)^2}{2} + \zeta^2 \frac{(\Delta L)^2}{2} \right) t^2
\end{equation}
The logarithmic negativity is given by ($\phi_{AB}$ from eq. \eqref{eq:apx:phi-AB})
\begin{align}\label{eq:apx:stochastic-variations-logarithmic-negativity}
  E_N &= \max\left\{0,\ \log_2\left(e^{-\gamma}\left(\cosh\gamma + \abs{\sin\phi_{AB}}\right)\right)\right\} \\
  &= \log_2\left\{ \frac{1}{2} e^{-\gamma} \left(\abs{\sin\phi_{AB}-\sinh\gamma} + \abs{\sin\phi_{AB}+\sinh\gamma} + 2\cosh\gamma\right) \right\} .
\end{align}
For general combinations of $\Delta L_A, \Delta L_B, \Delta \theta_A, \Delta \theta_B$ and for results without any approximations, the logarithmic negativity was computed numerically. 




\section{Density matrix for particles in front a vibrating plate} \label{apx:density-matrix-vibrating-plate}
The separations between the shield and the Particle state $A(B)_i$ in the parallel configuration are given by
\begin{equation}
  d^i_{A(B)} = L \pm_{A(B)} z \left(\abs{u} \mp_i \abs{\nabla u} \frac{\Delta x}{2}\right)
\end{equation}
where the first $\pm$ distinguishes between particle $A$ and $B$ and the second one between $i=1$ and $i=2$. The gravitational interaction is given as before in \cref{cha:first-look}.
After averaging over $z$ (normally distributed around $\mean{z} = 0$ and std. $\Delta z$) the resulting density matrix is now given by
\begin{equation}
  \mean{\rho} = \frac{1}{4} \begin{pmatrix}
    1 & e^{i \Delta \phi} e^{- \frac{1}{2} (\xi_\mathrm{Cas})^2 (\Delta z)^2} & e^{i \Delta \phi} e^{- \frac{1}{2} (\xi_\mathrm{Cas})^2 (\Delta z)^2} & 1 \\
    & 1 & e^{- \frac{1}{2} (2 \xi_\mathrm{Cas})^2 (\Delta z)^2} & e^{-i \Delta \phi} e^{- \frac{1}{2} (\xi_\mathrm{Cas})^2(\Delta z)^2} \\
    \multicolumn{2}{c}{\multirow{2}{*}{c.c}} & 1 & e^{-i \Delta \phi} e^{- \frac{1}{2} (\xi_\mathrm{Cas})^2 (\Delta z)^2} \\
    & & & 1
  \end{pmatrix}
\end{equation}
with
\begin{align}
  \Delta \phi &= \frac{G M_A M_B}{\hbar} \left(\frac{1}{4L^2} - \frac{1}{\sqrt{2L + (\Delta x)^2}}\right) t \\
  \xi_\mathrm{Cas} &= \begin{cases}
    \frac{c \pi^3 R}{720} \left(\frac{\varepsilon_r - 1}{\varepsilon_r + 1}\right)\varphi(\varepsilon_r)\cdot \frac{2 \abs{\nabla u} \Delta x}{\mathscr{L}^3} t & \text{in the PFA for }L / R \approx 1 \\
    \frac{3c R^3}{8\pi} \left(\frac{\varepsilon_r - 1}{\varepsilon_r + 2}\right)\frac{4\abs{\nabla u}\Delta x}{L^5}t & \text{in the LSL for } L / R \gg 1
  \end{cases}
\end{align}
which is only dependent on the gradient of the shape $\abs{\nabla u}$.
The logarithmic negativity is given by
\begin{equation}\label{eq:apx:en-thermal-shield}
  E_N(\mean{\rho}) = \log_2\left\{\frac{1}{4}\left(3 + e^{-4 \gamma}+\sqrt{(1-e^{-4\gamma})^2 + 16e^{-2\gamma}\sin^2\Delta\phi}\right)\right\}
\end{equation}
with 
\begin{equation}\label{eq:apx:decoherence-naive-vibration}
  \gamma = \frac{1}{2}(\xi_\mathrm{Cas})^2(\Delta z)^2 .
\end{equation}
It follows, that for large $\gamma$ and thus $(\Delta z)_\mathrm{crit} \sim 1/\xi_\mathrm{Cas}$, no entanglement is observable.


\section{Time evolution of two particles in front of a thermal plate}\label{apx:thermal-shield-time-evolution}
The time evolution operator $\op{U}=e^{-i\op{H}t/\hbar}$ of the hamiltonian eq. \eqref{eq:5:hamiltonian} can be calculated in the interaction picture using the \q{Magnus expansion} \cite{Blanes_2009}.
In the following calculations, the direct gravitational interactions between the two particles are ignored as they don't depend on the shield vibrations at all. The final evolution due to these couplings were already studied in \cref{cha:entanglement-generation} and can just be added in the end. 
The interaction picture hamiltonian in the $\{\ket{\psi^1_A\psi^1_B},\ket{\psi^1_A\psi^2_B},\ket{\psi^2_A\psi^1_B},\ket{\psi^2_A\psi^2_B}\}$-basis is given by
\begin{equation}
  \op{H}_\mathrm{int} = \sum_{m\in\{(k,l)\}} \begin{pmatrix}
    g^1_\mathrm{A,m} + g^1_\mathrm{B,m}& & & \\
    & g^1_\mathrm{A,m} + g^2_\mathrm{B,m}& & \\
    & & g^2_\mathrm{A,m} + g^1_\mathrm{B,m}& \\
    & & & g^2_\mathrm{A,m} + g^2_\mathrm{B,m}
  \end{pmatrix}(\op{a}e^{-i\omega_m t}+\op{a}^\dagger e^{i\omega_m t})
\end{equation}
The 4-dimensional operator at the beginning of the equation is referred to as $\op{G}$ in the following.
The time evolution in the Magnus expansion here given by \cite{Blanes_2009}
\begin{equation}
  \op{U}(t) = \exp{-\frac{i}{\hbar}\int_{0}^{t}\dd t_1 \op{H}_\mathrm{int}(t)-\frac{1}{2\hbar^2}\int_{0}^{t}\dd t_1 \int_{0}^{t_1}\dd t_2 [\op{H}_\mathrm{int}(t_1),\op{H}_\mathrm{int}(t_2)]} .
\end{equation}
All higher order terms vanish, so this is an exact result.
After substitution, the result is given by
\begin{align}
  \op{U}(t) &= \exp{\op{G}(f_1\op{a}^\dagger - f_1^*\op{a}) + i \op{G}^2 f_2} \\ 
  &= \op{D}\left(f_1(g^1_\mathrm{A,m} + g^1_\mathrm{B,m})\right) \exp{i f_2 (g^1_\mathrm{A,m} + g^1_\mathrm{B,m})^2} \ketbra{\psi^1_A\psi^1_B} + \dots
\end{align}
with
\begin{equation}
  f_1 = \frac{(1-e^{i\omega_m t})}{\hbar \omega_m}
  \quad \text{and} \quad
  f_2 = \frac{t\omega_m - \sin(t \omega_m)}{\hbar^2 \omega_m^2}
\end{equation}
and the displacement operator $\op{D}(\alpha)=\exp{\alpha \op{a}^\dagger - \alpha^*\op{a}}$. The evolved state $\rho(t)=\op{U}(t)\rho_0\op{U}^\dagger(t)$ is now given by
\begin{align}
\begin{split}
  \rho(t) =& \bigotimes_{m\in\{(k,l)\}} \op{D}\left(f_1(g^1_A + g^1_B)\right)\rho_\mathrm{th,m}\op{D}^\dagger\left(f_1(g^1_A + g^1_B)\right) \otimes \frac{1}{4} \ketbra{\psi^1_A\psi^1_B} \\
  +& \op{D}\left(f_1(g^1_A + g^1_B)\right)\rho_\mathrm{th,m}\op{D}^\dagger\left(f_1(g^1_A + g^2_B)\right) \otimes \frac{1}{4} e^{if_2(g^1_A + g^1_B)^2} \ketbra{\psi^1_A\psi^1_B}{\psi^1_A\psi^2_B} e^{-if_2(g^1_A + g^2_B)^2} \\
  +& \dots \\
  +& \op{D}\left(f_1(g^2_A + g^2_B)\right)\rho_\mathrm{th,m}\op{D}^\dagger\left(f_1(g^2_A + g^1_B)\right) \otimes \frac{1}{4} e^{if_2(g^2_A + g^2_B)^2} \ketbra{\psi^2_A\psi^2_B}{\psi^2_A\psi^1_B} e^{-if_2(g^2_A + g^1_B)^2} \\
  +& \op{D}\left(f_1(g^2_A + g^2_B)\right)\rho_\mathrm{th,m}\op{D}^\dagger\left(f_1(g^2_A + g^2_B)\right) \otimes \frac{1}{4}\ketbra{\psi^2_A\psi^2_B}
\end{split}
\end{align}
We are interested in the evolution of the two-particle system. This is given by tracing out the thermal shield $\rho_\mathrm{sys.} = \tr_{th}\left\{\rho(t)\right\}$. Using $\tr{A \otimes B} = \tr{A}\tr{B}$, it follows:
\begin{equation}
  \rho_\mathrm{sys.} = \frac{1}{4}\begin{pmatrix}
    1 & \prod_{m}\tr{\op{D}\left(f_1(g^1_A + g^1_B)\right)\rho_\mathrm{th,m}\op{D}^\dagger\left(f_1(g^1_A + g^2_B)\right)} e^{i f_2((g^1_A + g^1_B)^2 - (g^1_A + g^2_B)^2)} & \dots \\
    \vdots & \ddots & \\
    & & 
  \end{pmatrix}
\end{equation}
To calculate $\tr{\op{D}(\zeta_i)\rho_\mathrm{th}\op{D}^\dagger(\zeta_j)}$, we expand $\rho_\mathrm{th}$ into coherent states \cite{Steiner_2024}
\begin{equation}
  \rho_\mathrm{th} = \int \dd \alpha^2 \, \frac{1}{\bar{n}\pi} e^{-\frac{\abs{\alpha}^2}{\bar{n}}} \ketbra{\alpha}
\end{equation}
and calculate the required trace \cite{Steiner_2024}:
\begin{equation}
  \tr{\op{D}(\zeta_i)\rho_\mathrm{th}\op{D}^\dagger(\zeta_j)} = \exp{\phi - \abs{\Delta \zeta}^2 \left(\frac{1}{2} + \bar{n}\right)}
\end{equation}
where $\Delta \zeta = \zeta_i - \zeta_j$ and $\phi = (\zeta_j^*\zeta_i - \zeta_j\zeta_i^*)/2 = 0$.
The final decoherence elements of the evolved state therefore all have the form
\begin{equation}
  e^{-\gamma_{1,2}} = \exp{- \sum_{m} \abs{(g^1_{A,m} + g^1_{B,m})-(g^1_{A,m} + g^2_{B,m})}^2 f_1f^*_1 (\frac{1}{2} + \bar{n}_m)}
\end{equation}