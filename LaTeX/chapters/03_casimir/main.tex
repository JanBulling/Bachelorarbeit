\chapter{Casimir effect}\label{cha:casimir-effect}

Casimir forces can be viewed in a very similar way to the \textit{van der Waals forces}. In fact, both phenomena describe just two different sides of the same coin. They define the so-called \emph{dispersion forces} between neutral atoms or bodies.
The quantum theory of van der Waals forces between two neutral atoms was developed by London in 1930 who found the attractive potential $\propto 1/r^6$ for small separations \cite{London_1930}.
Casimir and Polder showed in 1948, that for separations larger than the resonance wavelength of the atoms, retardation effects need to be taken into account and the potential decays by a power law of $1/r^7$ \cite{Casimir_1948a}. 
Additionally, they calculated the interaction with an atom or molecule and a perfectly conducting plate, showing that macroscopic objects could experience these \emph{Casimir-Polder interactions} as well.
It becomes evident, that a full description of dispersion forces cannot be given by classical electrodynamics alone. Additional considerations regarding relativistic effects and quantum electrodynamics have to be made \cite{Bordag_2001,Klimchitskaya_2009,Lamoreaux_2004}.
Casimir, following a suggestion by Bohr \cite{Bordag_1999}, found a simple derivation using the zero-point energy of the vacuum to calculate the attraction between two conducting plates.
In quantum electrodynamics each point in the electromagnetic field can be described by an quantized harmonic oscillator with ground state energy $E_0 = \hbar\omega/2$.
The total \textit{zero-point energy} of the ground state (the vacuum) of the field is therefore given by summing over the energies $E_0$ for each possible mode $n$
\begin{equation}
  E_\mathrm{vacuum} = \frac{\hbar}{2} \sum_n \omega_n.
\end{equation} 
These sums are clearly divergent since there are infinitely many possible excitations.
Electrostatic boundary conditions require the field to be zero at the surface of conductors restricting the possible modes between two parallel plates.
Precisely the finite difference between the infinite vacuum energy with and without the macroscopic plates give rise to the \emph{Casimir forces}.
Often, this divergence is simply dropped, motivated by the fact that energy is usually only defined up to a constant \cite{Bordag_2001}. 
Casimir was able to use regularization techniques to deal with the infinite quantities and arrived at his famous formula \cite{Casimir_1948}
\begin{equation}\label{eq:3:casimir-energy-pp-conducting}
  E_\mathrm{Casimir} = -\frac{\hbar c \pi^2}{720 L^3} A
\end{equation}
for the attractive Casimir-potential between two plates with surface area $A$ and separation $L$.
The attractive force $F=-\nabla E$ between the plates can now be simply expressed as
\begin{equation}\label{eq:3:casimir-force-pp-conducting}
  F_\mathrm{Casimir} = - \frac{\hbar c \pi^2}{240 L^4} A .
\end{equation}
It is remarkable, that such a simple relation arises out of the infinities of the vacuum.
To this day, these Casimir forces are a major topic of modern scientific research. They are generally very difficult to calculate for geometries other than two infinitely large plates or for real materials with dielectric properties. 
For simple geometries, even the sign of the force is not always intuitively clear: As an example, the Casimir force for an ideal conducting spherical shell leads to an expansion of the sphere \cite{Boyer_1968,Klimchitskaya_2009}.
Between other simple and important objects like the sphere-plane or sphere-sphere geometry, no universally valid closed formula for any separation $L/R$ between the bodies exists. This is discussed in more detail in \cref{sec:3:casimir-plate-sphere}.

Almost ten years after the discovery of Casimir and Polder, Lifshitz was the first to find an expression for the Casimir force between two dielectric plates with arbitrary relative permittivity $\varepsilon_{r,\,1(2)}$ for separations larger than the resonant wavelength \footnote{The \q{resonance wavelength} for a macroscopic body in this case can be understood as e.g. the plasma frequency in the Drude model \cite{Ford_1998}. Different models for light-matter interaction result in slightly different resonant wavelength. The Lifshitz formula however holds true for the cases of separations in the micro-meter regime for all practical materials \cite{Kamp_2020}.} \cite{Lifshitz_1956}.
The expression he found facilitates the general complexity of the Casimir interactions and is only expressible as a complicated integral \cite{Lifshitz_1956}
\begin{multline}\label{eq:3:lifshitz-general-integral}
  F/A = -\frac{\hbar c}{32 \pi^2 L^4} \int\limits_0^\infty \dd x \int\limits_1^\infty \dd p \ \frac{x^3}{p^2}\biggl\{ \left[ \frac{(s_1+p)(s_2+p)}{(s_1-p)(s_2-p)} e^x - 1 \right]^{-1} + \\
  \left[ \frac{(s_1+ \varepsilon_{r,\,1} p)(s_2 + \varepsilon_{r,\,2} p)}{(s_1 - \varepsilon_{r,\,1} p)(s_2 - \varepsilon_{r,\,2} p)} e^x - 1 \right]^{-1} \biggr\}
\end{multline}
with
\begin{equation*}
  s_{1(2)} = \sqrt{\varepsilon_{r,\,1(2)} - 1 + p^2} .
\end{equation*}
In the limit of two perfectly conducting plates ($\varepsilon_{r,\,1} = \varepsilon_{r,\,2} \rightarrow \infty$), the integral can be solved analytically resulting in the same expression already obtained by Casimir
\begin{equation}
  F_\mathrm{cond.}/A = -\frac{\hbar c}{16 \pi^2 L^4} \int\limits_0^\infty \dd x \int\limits_1^\infty \dd p \ \frac{x^3}{p^2 (e^x - 1)} = -\frac{\hbar c \pi^2}{240 L^4} .
\end{equation}
Lifshitz determined the Casimir force between a conducting metal plate and a dielectric plate (denoted $\mathrm{DM}$) as well as the force between two dielectric plates with the same dielectric constant $\varepsilon_r$ ($\mathrm{DD}$) as
\begin{align} \label{eq:3:casimir-pp-F-DM-lifshitz}
  F_\mathrm{DM} &= -\frac{\hbar c \pi^2}{240 L^4} \frac{\varepsilon_r - 1}{\varepsilon_r + 1} \varphi(\varepsilon_r) \\
  F_\mathrm{DD} &= -\frac{\hbar c \pi^2}{240 L^4} \left( \frac{\varepsilon_r - 1}{\varepsilon_r + 1} \right)^2 \varphi(\varepsilon_r)\label{eq:3:casimir-pp-F-DD-lifshitz}
\end{align}
where $\varphi(\varepsilon_r)$ is a numerical function obtained by solving eq. \eqref{eq:3:lifshitz-general-integral}, which approaches $1$ for a perfect conductor \cite{Lifshitz_1956}. The function was numerically calculated and the result is shown in \cref{fig:3:lifshitz-function}.
\begin{figure}[!htbp]
  \centering
  \includegraphics[width=0.8\textwidth]{./../figures/casimir/casimir-lifshitz-function.pdf}
  \caption{Numeric calculations of the function $\varphi(\varepsilon)$ used in the Lifshitz formula eq. \eqref{eq:3:casimir-pp-F-DM-lifshitz} and \eqref{eq:3:casimir-pp-F-DD-lifshitz}. The function was calculated for a dielectric and a metal plates \textbf{(blue)} and two dielectric plates \textbf{(orange)}. The function approaches unity for $\varepsilon_r\rightarrow\infty$ and a finite value for $\varepsilon_r\rightarrow 1$.}
  \label{fig:3:lifshitz-function}
\end{figure}
For a dielectric and metal plate, $\varphi$ approaches the finite value $\varphi(\varepsilon_r \rightarrow 1) \approx 0.46$ for small dielectric constants. This limit is practically already reached at $\varepsilon_r \approx 4$ and $\varphi$ stays approximately constant for smaller $\varepsilon_r$.

\section{Proximity force approximation}\label{sec:3:pfa}

The Casimir-Polder force cannot be computed easily for different shapes. There even exists no analytic expression for the simple (and for this thesis relevant) plate-sphere geometry for all ratios $L/R$ and plate-sphere separations.
For a general shape, even the sign of the force, i.e. whether it is attractive or repulsive, is often unknown.
Fortunately, approximation methods exist and in particular the \emph{proximity-force-approximation (PFA)} can be calculated very easily \cite{Hartmann_2018,Emig_2007a,Bulgac_2006}.
The PFA is only valid for small separations ($L/R \approx 1$) between the considered smooth bodies.
The idea of this approximation is to divide the surfaces of the two bodies into infinitesimal small parallel plates with area $\dd A$ and summing over the forces $\dd F$ (or the Casimir-energy $\dd E$) between them (see \cref{fig:3:PFA}):
\begin{equation}\label{eq:3:pfa}
  E_\mathrm{PFA} = \iint_A \dd A \, \frac{E_\mathrm{plate-plate}}{A}
\end{equation}
where for the casimir energy per unit area $E_\mathrm{plate-plate}/A$ either eq. \eqref{eq:3:casimir-energy-pp-conducting} or any of the Lifshitz equations \eqref{eq:3:casimir-pp-F-DD-lifshitz}, \eqref{eq:3:casimir-pp-F-DM-lifshitz} can be chosen.
\begin{figure}[!htbp]
  \centering
  \def\svgwidth{0.55\textwidth}
  \input{./../figures/proximity-force-approximation.pdf_tex}
  \caption{In the proximity force approximation the sphere is divided into infinitesimal plane areas $\dd A$ which all exert a force $\dd F$ according to eq. \eqref{eq:3:casimir-force-pp-conducting}. All the contributions are added up together.}
  \label{fig:3:PFA}
\end{figure}
For the following calculations, it is important to distinguish between the distance between the plates center and the spheres center $L$ (like used before) and the edge-to-edge distance $\mathscr{L} = L - R$.

The problem with this approximation is, that it is ambiguous, what surface the area element $\dd A$ represents. For the plate-sphere geometry, the element can be either chosen tangential to the sphere or parallel to the plate (or in theory any other fictitious surface somewhere in between) \cite{Bulgac_2006}.
For the plate-sphere geometry, in the limit of the validity of the PFA $\mathscr{L} \ll R$ both methods yield the same result.
For the following calculations, I choose $\dd A$ parallel to the plate and the area can be parameterized with $r\in [0, R]$ and $\varphi \in [0, 2\pi]$ resulting in a distance $z$ between the infinitesimal area elements $z(r) = \mathscr{L} + R - \sqrt{R^2 - r^2}$ \footnote{Taking $\dd A$ tangential to the sphere, it can be parameterized with $\theta \in [0, \pi/2]$ and $\varphi \in [0, 2\pi]$ resulting in $z(\theta) = \mathscr{L} + R - R\cos\theta$. The PFA eq. \eqref{eq:3:pfa} yields with $\dd A = R^2\sin\theta\dd\theta\dd\varphi$ the result $\propto \frac{\pi R^2(R + 2\mathscr{L})}{\mathscr{L}^2(R+\mathscr{L})^2}$ which in the limit of $\mathscr{L} \ll R$ results in the same expression as eq. \eqref{eq:3:PFA-sphere-plate}.}. The PFA eq. \eqref{eq:3:pfa} then yields for a dielectric sphere against a perfectly conducting plate
\begin{align}
  E_\mathrm{plate-sphere} &= -\frac{\hbar c \pi^2}{720} \left(\frac{\varepsilon_r - 1}{\varepsilon_r + 1}\right) \varphi(\varepsilon_r) \int_0^R \dd r \int_0^{2\pi} r\dd \varphi \frac{1}{z(r)^3} \\
  &= -\frac{\hbar c \pi^3}{360} \left(\frac{\varepsilon_r - 1}{\varepsilon_r + 1}\right) \varphi(\varepsilon_r) \frac{R^2}{2\mathscr{L}^2(R + \mathscr{L})} \\
  &\approx -\frac{\hbar c \pi^3}{720} \left(\frac{\varepsilon_r - 1}{\varepsilon_r + 1}\right) \varphi(\varepsilon_r) \frac{R}{\mathscr{L}^2} \label{eq:3:PFA-sphere-plate}
\end{align}


\section{Casimir forces between a conducting plate and a dielectric sphere} \label{sec:3:casimir-plate-sphere}

There does not exist a closed form expression for the Casimir energy between a dielectric sphere with radius $R$ an dielectric constant $\varepsilon_r$ in front of a conducting plate, that is applicable at all sphere-plate separations $L/R$.
In the limit of small separations, the proximity force approximation from \cref{sec:3:pfa} is valid and yields for dielectric or conducting spheres
\begin{align}\label{eq:3:casimir-sphere-plate-PFA}
  &E_\mathrm{PFA} = -\frac{\hbar c \pi^3}{720} \left(\frac{\varepsilon_r - 1}{\varepsilon_r + 1}\right)\varphi(\varepsilon_r) \frac{R}{\mathscr{L}^2} \sim \frac{1}{(L-R)^2} \quad\quad \text{for}\ L/R \approx 1 \\ \label{eq:3:casimir-sphere-plate-PFA-conducting}
  &E_\mathrm{PFA,\,cond.} = E_\mathrm{PFA}(\varepsilon_r \rightarrow \infty) = -\frac{\hbar c \pi^3}{720} \frac{R}{\mathscr{L}^2} .
\end{align}
For arbitrary separations, the Casimir energy can only be expressed as an infinite series \cite{Emig_2007,Emig_2007a} or in terms of an integral \cite{Ford_1998}
The integral form reads
\begin{multline}\label{eq:3:ford-integral}
  F = - \frac{\hbar c}{4 \pi L^4} \int_{0}^{\infty} \dd \omega \, \alpha(\omega) \left[3\sin 2 \omega L - 6L\omega \cos 2 \omega L \right. \\ 
  \left. - 6L^2\omega^2 \sin 2 \omega L + 4L^3\omega^3 \cos 2 \omega L\right].
\end{multline}
where $\alpha$ is the electric polarizability of the sphere and the integration is performed over all possible interaction frequencies $\omega$ of the electromagnetic field with the materials.

In the \emph{large-separation-limit (LSL)}, where the sphere-plate separation are much larger the the resonant wavelength of the material, the polarizability can be taken as a static constant \cite{Ford_1998,Kamp_2020}.
In this case, the integral eq. \eqref{eq:3:ford-integral} can be solved analytically by using an exponential convergence factor
\begin{equation}
  F = -\frac{6 \hbar c}{4 \pi L^5} \alpha .
\end{equation}
The polarizability of a uniform dielectric sphere with a dielectric constant $\varepsilon_r$ is calculated in \cref{apx:polarizability-sphere} and is given by
\begin{equation}\label{eq:3:polarizability-sphere}
  \alpha_\mathrm{sphere} \propto \left(\frac{\varepsilon_r - 1}{\varepsilon_r + 2}\right) R^3
\end{equation}
resulting in a Casimir energy of
\begin{align}\label{eq:3:casimir-sphere-plate-LSL}
  &E_\mathrm{LSL} = -\frac{3}{8} \frac{\hbar c}{\pi L^4} \left(\frac{\varepsilon_r - 1}{\varepsilon_r + 2}\right)R^3 \sim \frac{1}{L^4} \quad\quad\quad \text{for}\ L/R \gg 1 \\ \label{eq:3:casimir-sphere-plate-LSL-conducting}
  &E_\mathrm{LSL,\,cond.} = E_\mathrm{LSL}(\varepsilon_r \rightarrow \infty) = -\frac{3}{8} \frac{\hbar c R^3}{\pi L^4} .
\end{align}
This matches precisely the leading-order term in the series expansion from Ref. \cite{Emig_2007a} and Ref. \cite{Pirozhenko_2013}.
A comparison of the PFA and LSL approximations across all separations is shown in \cref{fig:3:casimir-behavior}, alongside numerical results from Ref \cite{Emig_2007a}.
\begin{figure}[!ht]
  \centering
  \includegraphics[width=\textwidth]{./../figures/casimir/casimir-behavior.pdf}
  \caption{Behavior of the Casimir energy for different sphere-plate separations $L/R$. For close separations ($L/R \approx 1$), the PFA eq. \eqref{eq:3:casimir-sphere-plate-PFA} is valid whereas for large separations ($L/R \gg 1$) the LSL eq. \eqref{eq:3:casimir-sphere-plate-LSL} can be used. Additionally the numeric series expansion from Ref. \cite{Emig_2007a} is shown, which converges to the PFA and LSL in each limit.}
  \label{fig:3:casimir-behavior}
\end{figure}

The scaling of $1/L^4$ for large separations can be motivated empirically. Casimir and Polder calculated the potential between two atoms separated by a distance $L$ with polarizability $\alpha_i$ as \cite{Casimir_1948a} \footnote{For two macroscopic spheres, the casimir potential looks identical to eq. \eqref{eq:3:casimir-polder-two-atoms}. The polarizability $\alpha$ is given by eq. \eqref{eq:3:polarizability-sphere}, resulting in a Casimir potential between two identical dielectric spheres in the large separation limit of $-\frac{23 \hbar c}{4\pi L^7}\left(\frac{\varepsilon_r - 1}{\varepsilon_r + 2}\right)^2R^6$ \cite{Emig_2007}.}
\begin{equation}\label{eq:3:casimir-polder-two-atoms}
  E = -\frac{23 \hbar  c \alpha_1 \alpha_2}{4 \pi L^7} .
\end{equation}
If both atoms are approximated as spheres with $\alpha \sim R^3$, and one of them is increased to the size of $R \sim L$, the total Casimir-Polder potential between them effectively scales with $\sim R^3/L^4$.
This approximation corresponds to the limit $L/R \gg 1$ and aligns with the actual scaling of the macroscopic Casimir potential for large separations in eq. \eqref{eq:3:casimir-sphere-plate-LSL}.

The series expansion in \cref{fig:3:casimir-behavior} suggests, that the proximity-force-approximation is an upper bound for the actual Casimir interaction at all separations. In fact, it can be proven, that the PFA for a superconducting sphere and a plate always predicts a stronger force $\abs{\nabla E}$ than the LSL.
\begin{theorem}
  The Casimir force in the PFA-model eq. \eqref{eq:3:casimir-sphere-plate-PFA} between a superconducting sphere ($\varepsilon_r \rightarrow \infty$) and a perfectly conducting plate is an upper bound for the LSL eq. \eqref{eq:3:casimir-sphere-plate-LSL}.
\end{theorem}
\begin{proof}
  The proof is given in the following steps: \textbf{(a)} first it is shown that $\abs{\nabla E_\mathrm{PFA}} > \abs{\nabla E_\mathrm{LSL}}$ for arbitrary dielectric spheres, then it will be shown \textbf{(b)} that $\abs{\nabla E_\mathrm{PFA,\,cond.}} \geq \abs{\nabla E_\mathrm{PFA,\,diel.}}$.

  \textbf{(a)} By directly comparing the gradients of eq. \eqref{eq:3:casimir-sphere-plate-PFA} (PFA) and eq. \eqref{eq:3:casimir-sphere-plate-PFA} (LSL),  one can find
  the inequality
  \begin{align*}
    &\qquad\qquad\qquad\qquad\qquad\ \ \abs{\nabla E_\mathrm{PFA}} > \abs{\nabla E_\mathrm{LSL}} \\
    &\Longleftrightarrow \quad  \frac{2\hbar c \pi^3}{720}\left(\frac{\varepsilon_r - 1}{\varepsilon_r + 1}\right)\varphi(\varepsilon_r)\frac{R}{\mathscr{L}^3} > \frac{12 \hbar c}{8\pi L^5}\left(\frac{\varepsilon_r - 1}{\varepsilon_r + 2}\right)R^3 \\
    &\Longleftrightarrow \quad \frac{\pi^4}{540}\left(\frac{\varepsilon_r + 2}{\varepsilon_r + 1}\right)\varphi(\varepsilon_r) > \frac{(L-R)^3 R^2}{L^5} = \left(\frac{R}{L}\right)^2 - 3\left(\frac{R}{L}\right)^3 + 3\left(\frac{R}{L}\right)^4 - \left(\frac{R}{L}\right)^5
  \end{align*}
  One can easily convince oneself that the right-hand side (for $R/L \leq 1$) is upperbounded by $\approx 0.0346$ (at $R/L = 0.4$). By remembering that $(\varepsilon_r + 2)/(\varepsilon_r + 1) > 1$ and $\varphi(\varepsilon_r) \gtrsim 0.46$ one can put a lower bound on the left-hand side by $0.0830 > 0.0346$. Therefore, $\abs{\nabla E_\mathrm{PFA}} > \abs{\nabla E_\mathrm{LSL}}$.

  \textbf{(b)} By using eq. \eqref{eq:3:casimir-sphere-plate-PFA} and eq. \eqref{eq:3:casimir-sphere-plate-PFA-conducting} for the PFA of a dielectric and conducting sphere, it follows quickly that $\abs{\nabla E_\mathrm{PFA,\,cond.}} \geq \abs{\nabla E_\mathrm{PFA}(\varepsilon_r)}$, because $\varphi(\varepsilon_r)$ as well as $(\varepsilon_r - 1)/(\varepsilon_r + 1)$ are monotonically increasing with $\varepsilon_r$. 

  Combining steps \textbf{(a)} and \textbf{(b)} results in
  \begin{equation}
    \abs{\nabla E_\mathrm{PFA,\,cond.}} \geq \abs{\nabla E_\mathrm{PFA,\,diel.}} > \abs{\nabla E_\mathrm{LSL}} .
  \end{equation}
  Thus, the PFA provides an upper bound for the Casimir force at all separations.
\end{proof}
\begin{remark}
  For later calculations, only the difference in the Casimir energy for slightly different separations $L$ and thus effectively the gradient $\nabla E = \dd E / \dd L$ is required. Thus, the proof was given in terms of the Casimir force.
\end{remark}

For subsequent calculations, the PFA is therefore used as a worst-case approximation of the Casimir energy. Whenever possible, results are cross-verified anc compared with the LSL model.

\section{Imperfect plate and spheres}
\label{sec:3:imperfect-plates}

In practice, the surfaces of the sphere and plate are not perfectly flat and contain imperfections, leading to small, localized variations in the sphere-plate separation and, consequently to slight changes in the Casimir energy. 
While, in reality, both the sphere and the plate have rough surfaces, we limit ourselves to the case where the plate is rough and the sphere is smooth, as we do not expect any fundamental changes.
Under the PFA, the Casimir interaction solely depends on the surface-to-surface separation $\mathscr{L}$ and thus, all irregularities on the sphere's surface effectively be modeled as an equivalent roughness on the plate.
To quantify and estimate the impact of uneven surfaces on the Casimir energy, several representative types of plate imperfections with characteristic amplitude $\Delta \mathscr{L}$ shown in \cref{fig:3:imperfect-plates} have been studied with numerical methods.
\begin{figure}[!htbp]
  \centering
  \includegraphics[width=\textwidth]{../figures/casimir/imperfect-plates-advanced.pdf}
  \caption{A selection of imperfect plates. \textbf{(a)} A simple gaussian deformation in the same size as the sphere. \textbf{(b)} Linearly inclining plate or a tilted flat plate. \textbf{(c)} Uneven and noisy but uniformly random surface realized using \textit{Perlin noise} \cite{Perlin_1985}. \textbf{(d)} A cross-shape in the center of the plate.}
  \label{fig:3:imperfect-plates}
\end{figure}
\begin{enumerate}
  \item[\textbf{(a)}] A \textit{gaussian shaped bump or dip} in the plate can be used to describe a range of possible local deformations comparable in size to the sphere. 
  For a small shield ($r_s \approx R$), thermal vibrations resemble these deformations, as discussed in \cref{cha:the-shield}.
  Displacements with positive or negative amplitudes $\pm \Delta \mathscr{L}$ following a Gaussian profile were studied.

  \item[\textbf{(b)}] If the characteristic length scale of imperfections is much larger than the sphere's radius and the sphere is sufficiently close to the plate, it experiences a nearly linear gradient in the plate's surface height, effectively behaving as though the plate was tilted. These \textit{linear deflection} can describe thermal vibrations for larger shields $r_s \ll R$. At small gradients, variations in the Casimir potential cancel out in first order since the potential in the PFA $1/(\mathscr{L} \pm \Delta \mathscr{L})^2 \sim 1/\mathscr{L}^2 \mp 2\Delta \mathscr{L}/\mathscr{L}^3$ depends linearly on the deflection. As a result, no significant change in the total attraction force is expected.
  
  \item[\textbf{(c)}] Similarly negligible are \textit{random noisy but uniformly distributed deformations}, provided the typical length scale of the noise is smaller than the sphere's radius. Here, the noise was modeled using \textit{Perlin noise} \cite{Perlin_1985}, which produces smooth pseudo-random surface textures commonly used in computer science to imitate surface roughness. Equidistant grid-points are defined, each of which is assigned with a pseudo-random gradient. The noise function follows this gradient in the vicinity of a grid-point and the interpolation between points generates smooth transitions. Due to the uniformness, no large deviations from an ideal flat plate are expected.
  
  \item[\textbf{(d)}] Structural features on the plate, such as a \textit{centered cross}, may enhance the stability and rigidity of the shield, potentially reducing thermal vibrations. However, the effects of such features, including amplification of the Casimir interaction, must be investigated further.
\end{enumerate}
The resulting Casimir potentials between a macroscopic sphere and the imperfect surfaces were numerically calculated in the PFA and are shown in \cref{fig:3:casimir-imperfect-plates}.
\begin{figure}[!htbp]
  \centering
  \includegraphics[width=\textwidth]{../figures/casimir/casimir-potential-imperfect-plates-relative.pdf}
  \caption{Casimir energy between a sphere and plates with surface imperfections shown in \cref{fig:3:imperfect-plates}. 
  The gaussian deformation (blue) was calculated for displacements with amplitude $\pm\Delta\mathscr{L}$. The shaded region bounds all imperfections and represents the Casimir energy between a flat plate moved $\pm\Delta\mathscr{L}$ closer or farther to the sphere. In the limit $\Delta \mathscr{L} \ll \mathscr{L}$, all imperfections are negligible.}
  \label{fig:3:casimir-imperfect-plates}
\end{figure}
All imperfections are bounded by the potential between a sphere and a perfectly flat ideal plate moved by a distance $\Delta \mathscr{L}$ closer or farther.
This is symbolized by the gray region in \cref{fig:3:casimir-imperfect-plates}.
For the gaussian distributions, this overestimation is not particularly large and especially for large structures, like the cross, this bound is practically reached.
As expected, the uniformly distributed noise as well as a slightly tilted plane do not increase the Casimir potential substantially even at small separations.
For small imperfections or large separations, plate imperfections are negligible as the relative effect decreases with $\Delta \mathscr{L}/\mathscr{L} \rightarrow 0$.
However, the considerations made in this section are particularly important for small shields the size of the particles and close distances.