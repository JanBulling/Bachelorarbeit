\section{Proximity force approximation}\label{sec:3:pfa}

While the macroscopic Casimir force has an analytical description for two plates, it is not possible to find such an expression for arbitrary geometries. There even exists no analytic expression for the
simple (and for this thesis relevant) plate-sphere geometry for all separations of the bodies.
Fortunately, approximation methods exist and in particular the \emph{proximity-force-approximation (PFA)} can, in many cases, be calculated easily as long as the involved surfaces are smooth \cite{Hartmann_2018,Emig_2007a,Bulgac_2006}.
The PFA is only valid for small separations ($L/R \approx 1$) where $R$ is the typical length scale of the bodies and $L$ the distance between the two bodies.
In the sphere-plate geometry, $R$ would be the radius of the sphere and $L$ the center-to-plate distance.
In the PFA, the surfaces of the two bodies are divided into infinitesimal small parallel segments with area $\dd A$ as depicted in \cref{fig:3:PFA}.
Then, one sums over the forces each of the surface elements experiences to estimate the force on the whole body. This is given by
\begin{equation}\label{eq:3:pfa}
  E_\mathrm{PFA} = \iint_A \dd A \, \frac{E_\mathrm{plate-plate}}{A}
\end{equation}
where for the Casimir energy per unit area $E_\mathrm{plate-plate}/A$ either eq. \eqref{eq:3:casimir-energy-pp-conducting} or alternatively any of the Lifshitz equations eq. \eqref{eq:3:casimir-pp-F-DD-lifshitz} or eq. \eqref{eq:3:casimir-pp-F-DM-lifshitz} can be used.
\begin{figure}[!htbp]
  \centering
  \def\svgwidth{0.55\textwidth}
  \input{./../figures/proximity-force-approximation.pdf_tex}
  \caption{In the proximity force approximation the sphere is divided into infinitesimal plane areas $\dd A$ which all exert a force $\dd F$ according to eq. \eqref{eq:3:casimir-force-pp-conducting}. All the contributions are added up together.}
  \label{fig:3:PFA}
\end{figure}
For the following calculations, it is important to distinguish the distance between the plate and the spheres center of mass donated by $L$ and the edge-to-edge separation $\mathscr{L} = L - R$.

The problem with this approximation is that it is ambiguous what surface the area element $\dd A$ represents. 
In general, the segment $\dd A$ is tangential to only one of the surfaces (or in theory to any other fictitious surface somewhere in between) \cite{Bulgac_2006} and is thus not uniquely defined.
In the limit of the validity of the PFA $\mathscr{L} \ll R$, all definitions usually yield the same result.
For the following calculations, $\dd A$ is chosen tangential to the plate and can be parametrized with $r\in [0, R]$ and $\varphi \in [0, 2\pi]$ resulting in a distance $z$ between the infinitesimal area elements\footnote{Taking $\dd A$ tangential to the sphere, it can be parametrized with $\theta \in [0, \pi/2]$ and $\varphi \in [0, 2\pi]$ resulting in $z(\theta) = \mathscr{L} + R - R\cos\theta$. The PFA eq. \eqref{eq:3:pfa} yields with $\dd A = R^2\sin\theta\dd\theta\dd\varphi$ the result $\propto \frac{\pi R^2(R + 2\mathscr{L})}{\mathscr{L}^2(R+\mathscr{L})^2}$ which in the limit of $\mathscr{L} \ll R$ results in the same expression as eq. \eqref{eq:3:PFA-sphere-plate}.} $z(r) = \mathscr{L} + R - \sqrt{R^2 - r^2}$. The PFA eq. \eqref{eq:3:pfa} yields for a dielectric sphere and a perfectly conducting plate
\begin{align}
  E_\mathrm{PFA} &= -\frac{\hbar c \pi^2}{720} \left(\frac{\varepsilon_r - 1}{\varepsilon_r + 1}\right) \varphi(\varepsilon_r) \int_0^R \dd r \int_0^{2\pi} r\dd \varphi \frac{1}{z(r)^3} \\
  &= -\frac{\hbar c \pi^3}{360} \left(\frac{\varepsilon_r - 1}{\varepsilon_r + 1}\right) \varphi(\varepsilon_r) \frac{R^2}{2\mathscr{L}^2(R + \mathscr{L})} \\
  &\overset{\mathscr{L}\ll R}{\approx} -\frac{\hbar c \pi^3}{720} \left(\frac{\varepsilon_r - 1}{\varepsilon_r + 1}\right) \varphi(\varepsilon_r) \frac{R}{\mathscr{L}^2} \label{eq:3:PFA-sphere-plate}
\end{align}
