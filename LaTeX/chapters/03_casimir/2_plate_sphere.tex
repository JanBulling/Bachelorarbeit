\section{Casimir forces between a conducting plate and a dielectric sphere}
\label{sec:3:casimir-plate-sphere}

An empirical derivation for power law of the casimir energy between a sphere and a conducting plate can be made directly from the energy between two atoms with static polarizability $\alpha_i$ given by Casimir and Polder \cite{Casimir_1948a}. They derived an expression for the Casimir-Polder potential of \footnote{For two macroscopic spheres, the casimir potential looks very similar to eq. \eqref{eq:3:casimir-polder-atom-atom}. The polarizability of a sphere is given by eq. \eqref{eq:3:polarizability-sphere}. Using this result, the Casimir energy between two identical dielectric spheres is given as $-\frac{23\hbar c}{4\pi L^7}\left(\frac{\varepsilon_r-1}{\varepsilon_r+2}\right)^2R^6$. !!!CITATION!!!}
\begin{equation}\label{eq:3:casimir-polder-atom-atom}
  E = -\frac{23\hbar c \alpha_1 \alpha_2}{4 \pi L^7} .
\end{equation}
The polarizability of a sphere with radius $R$ is derived in appendix \ref{apx:polarizability-sphere} and is given for an dielectric with $\varepsilon_r$ as 
\begin{equation} \label{eq:3:polarizability-sphere}
  \alpha_\mathrm{sphere} = \left(\frac{\varepsilon_r-1}{\varepsilon_r+2}\right)R^3 .
\end{equation}
If one atom is now replaced by a conducting sphere ($\varepsilon_r \rightarrow 1$) of radius $\sim L$ (much larger than the atom) with a polarizability of $L^3$, it get obvious that between this big sphere and the atom, the energy is given by a power law of $1/L^4$.
It is therefore natural to assume, that for a macroscopic sphere and a macroscopic plate, the Casimir energy behaves similar to a $1/L^4$ law - at least for the \textit{large separation limit} (LSL).
The exact calculation for this problem is very hard. In fact, no analytic solution is known.

Ford was able to determine an integral expression using a macroscopic approach in 1998 \cite{Ford_1998}:
\begin{multline}
  F = - \frac{\hbar c}{4 \pi L^4} \int_{0}^{\infty} \dd \omega \, \alpha(\omega) \left[3\sin 2 \omega L - 6L\omega \cos 2 \omega L \right. \\ 
  \left. - 6L^2\omega^2 \sin 2 \omega L + 4L^3\omega^3 \cos 2 \omega L\right].
\end{multline}
The expression depends on the polarizability, which is generally not constant for a dielectric. Especially for small separations between the sphere and the plate, this dependence and the non-constant polarizability make this integral nearly unsolvable.
For large separations, much larger than the absorption wavelength of the dielectric or much larger than the wavelength corresponding to the plasma frequency in the Drude-Model, the polarizability can be assumed to be static $\alpha=\mathrm{const}$ \cite{Ford_1998,Kamp_2020}. In this simplifying case, the integral can be solved using an exponential convergence factor and results in
\begin{equation}
  F = -\frac{6 \hbar c}{4 \pi L^5} \alpha
\end{equation}
and thus 
\begin{equation}
  E = -\frac{3}{8}\frac{\hbar c}{\pi L^4} \left(\frac{\varepsilon_r - 1}{\varepsilon_r + 2}\right)R^3 .
\end{equation}
For the large separation limit, the Casimir interaction between a sphere and a plate behaves like expected considering the motivation of the $1/L^4$-law above in this section.


- Series expansion

- Comparison to PFA

- Showing PFA is always the maximum

\begin{theorem}
  The PFA model predicts a stronger casimir potential for all separations $L/R$ than the LSL or the exact series expansion.
\end{theorem}
\begin{proof}
  a) Series expansion in the one paper. $LSL/PFA \leq 1$ for all $L/R$ \cite{Emig_2007a}.

  b) Better than LSL (proof)
\end{proof}
\begin{remark}
  This is rather unintuitive, because one would expect it to be the other way around. The $1/L^4$ dependence in the LSL should - for small separations - tend much faster to zero than the $1/\mathscr{L}^2$ dependence in the PFA. However considering the differences between $L$ and $\mathscr{L} = L-R$, this is no longer surprising.
\end{remark}