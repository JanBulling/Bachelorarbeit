\section{Time evolution under a gravitational potential}\label{sec:2:time-evolution}

The time evolution of a quantum system is governed by the Schrödinger equation
\begin{equation}
  i\hbar \pdv{t} \ket{\psi(t)} = \op{H} \ket{\psi(t)}
\end{equation}
where in this case the interaction Hamiltonian responsible for the entanglement dynamics is given by $\op{H} = \op{V}$ in eq. \eqref{eq:2:newtonian-potential}.
The eigenbasis of $\op{V}$ is given by $\left\{ \ket{\psi_A^{(1)}}, \ket{\psi_A^{(2)}} \right\}\otimes \left\{ \ket{\psi_B^{(1)}}, \ket{\psi_B^{(2)}} \right\}$, as all these states are eigenstates of the distance operator $\op{L}$
\begin{equation}
  \op{V} \ket{\psi^{(i)}_A} \otimes \ket{\psi^{(j)}_B} = -\frac{G M_A M_B}{2L^{(ij)}} \ket{\psi^{(i)}_A} \otimes \ket{\psi^{(j)}_B} .
\end{equation}
The Schrödinger equation for the diagonal Hamiltonian $\op{H}$ can be directly solved for the initial state eq. \eqref{eq:2:initial-state} with the solution given by 
\begin{equation}
  \ket{\psi(t)} = \frac{1}{2} \sum_{i,j\in\{1,2\}} \exp{\frac{i}{\hbar} \frac{G M_A M_B}{2L^{(ij)}} t} \ket{\psi^{(i)}_A \psi^{(j)}_B}
\end{equation}
where the tensor product $\otimes$ was omitted.
It is possible to express the state using the dynamically accumulated phases $\phi^{(ij)}$ which build-up after a mutual interaction as
\begin{equation}\label{eq:2:evolved-state}
  \ket{\psi(t)} = \frac{1}{2}\bigl(
    e^{i\phi^{(11)}} \ket{\psi_A^{(1)} \psi_B^{(1)}} 
    + e^{i\phi^{(12)}} \ket{\psi_A^{(1)} \psi_B^{(2)}}
    + e^{i\phi^{(21)}} \ket{\psi_A^{(2)} \psi_B^{(1)}} 
    + e^{i\phi^{(22)}} \ket{\psi_A^{(2)} \psi_B^{(2)}} \bigr) .
\end{equation}
The phases $\phi^{(ij)}$ in the specific setup shown in \cref{fig:2:simple-problem} are given by
\begin{equation}
  \phi \equiv \phi^{(11)} = \phi^{(22)} = \frac{G M_A M_B}{2\hbar L}t 
  \qquad \text{and} \qquad 
  \phi^{(12)} = \phi^{(21)} = \frac{G M_A M_B}{\hbar \sqrt{4L^2 + (\Delta x)^2}}t .
\end{equation}
By expanding the phases for small superposition sizes $\Delta x \ll L$, the global phase $\phi$ can be factored out of the evolved state
\begin{equation}\label{eq:2:definition-delta-phi}
  \phi^{(12)} = \phi^{(21)} \approx \frac{GM_AM_B}{\hbar} \left[ \frac{1}{2L} - \frac{(\Delta x)^2}{16 L^3} \right] t \equiv \phi - \Delta\phi ,
\end{equation}
which ultimately can be written in the form
\begin{equation}\label{eq:2:evolved-state-factored}
  \ket{\psi(t)} = e^{i\phi}\frac{1}{\sqrt{2}}\left[ 
    \ket{\psi_A^{(1)}} \otimes \frac{\ket{\psi_B^{(1)}} + e^{-i\Delta\phi} \ket{\psi_B^{(2)}}}{\sqrt{2}}
    + \ket{\psi_A^{(2)}} \otimes \frac{e^{-i\Delta\phi} \ket{\psi_B^{(1)}} + \ket{\psi_B^{(2)}}}{\sqrt{2}} \right] .
\end{equation}
It can be seen that in general, the resulting state is not expressible as a product state, hence it is entangled.
This is of course only the case, if $\Delta \phi \neq k\pi, \ k\in\mathbb{N}$.

In order to assess quantitatively how entangled the state $\ket{\psi(t)}$ is after time $t$, a more sophisticated measure is required. 
One possible measure is the \q{logarithmic negativity}, which is introduced in the next section and used in the rest of this work.