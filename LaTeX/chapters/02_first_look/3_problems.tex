\section{Issues with the idealized experimental procedure}\label{sec:2:experimental-problems}

For the practical realization of an experiment on measuring gravitationally induced entanglement of masses, other forms of direct or indirect interaction between the particles must be suppressed such that the measured entanglement ultimately arises only due to their gravitational interaction.
In particular, the short-range \emph{Casimir interactions} \cite{Casimir_1948} discussed in \cref{cha:casimir-effect} have to be shielded as they exert a much greater attraction force on the particles at small separations than gravity.
It is still a hot topic of discussions whether Casimir interactions can even entangle macroscopic bodies at all, as it is not even clear if it is a conservative force in the first place - although most researchers believe it is \cite{DeBiase_2012,Yi_2023}.
To estimate the minimal particle-particle separation $2L$ while requiring that the gravitational interaction $V_\mathrm{Gravity}$ is stronger than the Casimir interactions $V_\mathrm{Casimir}$ \cite{Emig_2007} by a factor $\chi > 1$, the following inequality can be stated:
\begin{align}
  \chi \abs{V_\mathrm{Casimir}} &\leq \abs{V_\mathrm{Gravity}} \\
  \Longleftrightarrow \quad \chi \frac{23 \hbar c}{4 \pi (2L)^7} \left(\frac{\varepsilon_r - 1}{\varepsilon_r + 2}\right)^2 R^6 &\leq  \frac{G M^2}{2L} .
\end{align}
Using $M = 4/3 \pi R^3\rho_\mathrm{Silica}$, the minimum separation distance is independent of the size of the particle and is given by
\begin{equation}
  L \geq \left(\frac{207}{4096} \frac{\hbar c}{\pi^3 G \rho_\mathrm{Silica}^2}\right)^{1/6} \sqrt[6]{\chi} \approx 69\si{\mu m} \sqrt[6]{\chi} .
\end{equation}
For the same particle as used before, the time for a single measurement, i.e. the coherence time $t_\mathrm{max} \approx 30\si{s} \sqrt{\chi}$ is very large.
The field of levitated particles is promising for these experiments as it offers an isolated, noise-reduced environment while still allowing for exceptional force sensitivity as well as precise quantum control and thus long coherence times \cite{Aspelmeyer_2024,GonzalezBallestero_2021}.
Nevertheless, it would be beneficial to reduce the separation distance between the particles for a shorter measurement time.
For this, usually a conducting \emph{Faraday shield} between the particles is proposed \cite{Kamp_2020}.
Such a shield would simultaneously suppress all other forms of electromagnetic interactions such as Coulomb forced, if the particles are happened to be charged.
Coulomb forces have the ability to entangle the particles as well\footnote{In fact, the Aspelmeyer group is currently working on an experiment trying to measure entanglement due to Coulomb interactions \cite{Rudolph_2022}.} and due to the similar distance behavior for Coulomb and gravitational interactions as well as the stronger coupling, these interactions could potentially be problematic.

This thesis is focused around the problems which arise in the generation of entanglement in the presence of the Faraday shield.
Reconstructing the position states of the masses requires many experimental runs and small variations in the initial setup between measurements introduce effective decoherence.
Casimir interactions between the particles and the newly placed Faraday shield can degrade entanglement in the final averaged measurement.
In \cref{cha:entanglement-generation} this effect is analyzed in depth, narrowing the range of viable parameters for particle-shield separation, superposition size, and particle mass.
Additionally, thermal vibrations and shield-induced noise are explored in \cref{cha:the-shield}.
